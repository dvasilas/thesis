\documentclass[a4paper,usenames,dvipsnames]{report}
\usepackage{fancyhdr}
\usepackage[ddmmyyyy,hhmmss]{datetime}
\usepackage{tcolorbox}
\usepackage[numbers,square,sort&compress]{natbib}
\usepackage[
  colorlinks=true,
  breaklinks=true,
  allcolors=blue]{hyperref}
\usepackage{chapterbib}
\usepackage{listings}
\usepackage{graphicx}
\usepackage[colorinlistoftodos]{todonotes}
\usepackage[]{algorithm}
\usepackage{algpseudocode}
\usepackage{amssymb}
\usepackage{subcaption}
\usepackage{float}
\usepackage[]{xcolor}
\usepackage{array,longtable,calc}

\def\UrlBreaks{\do\/\do-}

\setcounter{tocdepth}{3}
\setcounter{secnumdepth}{4}

\usepackage[a4paper,top=3cm,bottom=2cm,left=3cm,right=3cm,marginparwidth=1.75cm]{geometry}

\usepackage{booktabs}% http://ctan.org/pkg/booktabs
\newcommand{\tabitem}{~~\llap{\textbullet}~~}
\usepackage{adjustbox}
\usepackage{multirow}
\usepackage{tabularx}
\newcolumntype{M}[1]{>{\centering\arraybackslash}m{#1}}
\usepackage{afterpage}

\newcommand\blankpage{%
    \null
    \thispagestyle{empty}%
    \addtocounter{page}{-1}%
    \newpage}

\colorlet{punct}{red!60!black}
\definecolor{delim}{RGB}{20,105,176}
\colorlet{numb}{magenta!60!black}

\lstdefinelanguage{qpulang}{
  basicstyle=\normalfont\ttfamily,
  captionpos=b,
}

\lstdefinelanguage{json}{
  basicstyle=\normalfont\ttfamily,
  breaklines=true,
  commentstyle=\color{gray},
  morecomment={[l]{//}},
  literate=
    {\#}{{{\color{numb}{\#}}}}{1}
    {:}{{{\color{punct}{:}}}}{1}
    {,}{{{\color{punct}{,}}}}{1}
    {\{}{{{\color{delim}{\{}}}}{1}
    {\}}{{{\color{delim}{\}}}}}{1}
    {[}{{{\color{delim}{[}}}}{1}
    {]}{{{\color{delim}{]}}}}{1},
    captionpos=b,
}

\lstset{float,
  basicstyle=\normalfont\ttfamily,
  language=SQL,
  % frame=tb,
  keywordstyle=[1]\color{NavyBlue},
  keywordstyle=[2]\color{PineGreen},
  keywords=[2]{attributeName, tableName, stringValue, floatValue, intValue, dateTimeValue, timestampValue},
  captionpos=b,
  mathescape=true,
  % tabsize=2,
  % xleftmargin=\parindent,
  escapeinside={(*@}{@*)},
  breaklines=true,
  morekeywords=[1]{SYNC, ASYNC, TO, SYSTEM, START, LATEST},
  deletekeywords={VALUE},
  literate=
    *{=}{{{\color{gray}=}}}{1}
    {<}{{{\color{gray}<}}}{1}
    {>}{{{\color{gray}>}}}{1}
    {!}{{{\color{gray}!}}}{1}
  }

\lstdefinelanguage{pseudo}{
  basicstyle=\normalfont\ttfamily,
  keywordstyle=\color{NavyBlue},
  captionpos=b,
  breaklines=true,
  keywords={type, class, function},
}

% \pagestyle{fancy}
% \lhead{}
% \rfoot{Compiled on \today\ at \currenttime}
% \cfoot{}
% \lfoot{Page \thepage}

\makeindex

\title{A flexible and decentralised approach to query processing for geo-distributed data systems}
\author{Dimitrios Vasilas}
\date{}

\begin{document}
\pagenumbering{roman}

\afterpage{\blankpage}

\thispagestyle{empty}

% \newgeometry{textheight=25cm,textwidth=16cm}

\includegraphics[scale=.08]{./figures/Logo_Sorbonne_Universite.png}
\vspace*{1.5cm}
\begin{center}
  {\Huge{\bf Sorbonne Universit\'e}}

  \vspace*{1cm}

  {\Large\'Ecole Doctorale Informatique, T\'el\'ecommunications et \'Electronique}

  \vspace*{0.3cm}

  {\Large Scality}

  \vspace*{0.3cm}

  {\Large Laboratoire d’Informatique de Paris 6}

  \vspace*{0.3cm}

  {\Large Inria}

  \vspace*{0.3cm}

  {\Large \'Equipe DELYS}

  \vspace*{1cm}

  {\LARGE {\bf A flexible and decentralised approach to query processing for geo-distributed data systems}}

  \vspace*{1.5cm}

  {\Large Par Dimitrios Vasilas}

  \vspace*{1cm}

  {\Large Th\`ese de doctorat en informatique}

  \vspace*{1cm}

  {\Large Dirig\'ee par Marc Shapiro et Bradley King}

  \vspace*{1cm}

  {\large Pr\'esent\'ee et soutenue publiquement le 19 f\'evrier 2021}
\end{center}

\vspace*{1cm}
\begin{flushleft}
{\large Devant un jury compos\'{e} de :  }\\
\resizebox{\textwidth}{!}{\begin{tabular}{r@{\ }ll}
  & {\large \textbf{M. Bernd} {\sc\textbf{Amann}}, Professeur, Sorbonne Universit{\'e}} & {\large Examinateur}\\
  & {\large \textbf{Mme. Bettina} {\sc\textbf{Kemme}}, Associate Professor, McGill University} & {\large Examinateur}\\
  & {\large \textbf{M. Bradley} {\sc \textbf{King}}, Co-founder \& Field CTO, Scality} & {\large Encadrant}\\
  & {\large \textbf{M. S\'{e}bastien} {\sc\textbf{Monnet}}, Professeur, Universit{\'e} Savoie Mont Blanc} & {\large Rapporteur}\\
  & {\large \textbf{M. Themis} {\sc\textbf{Palpanas}}, Professeur, Universit\'{e} de Paris} & {\large Examinateur}\\
  & {\large \textbf{M. Nuno} {\sc\textbf{Pregui\c{c}a}}, Associate professor, Universidade Nova de Lisboa} & {\large Rapporteur}\\
  & {\large \textbf{M. Masoud} {\sc\textbf{Saeida Ardekani}}, Software Engineer, Google} & {\large Examinateur}\\
  & {\large \textbf{M. Marc} {\sc\textbf{Shapiro}}, Directeur de Recherche \'Em\'erite, Sorbonne Universit{\'e} \& Inria} & {\large Directeur de th\`{e}se}\\
\end{tabular}}
\end{flushleft}
% \restoregeometry

% \maketitle

\abstract{
Query processing is an essential component of today's data serving systems.
Query processing involves a variety of metrics that are in tension and create trade-offs.
Because of these trade-offs, application developers need to tune query engines to the characteristics and needs of each application.
Today's query engines often handle requests from users around the world, accessing data spread across geographically distributed sites.
This thesis studies how to support efficient query processing in contexts in which users and data are distributed across multiple geographic locations.

We present an analysis of the design decision and trade-offs in geo-distributed query processing.
In particular, we study how the placement of derived state used by the query engine to accelerate query processing (indexes, materialized views)
and the communication patterns involved in query processing and state maintenance affect three metrics: query performance, query result freshness, and cross-site network resource consumption.
We propose a query engine architecture that, as opposed to current state-of-the-art approaches, allows application developers to make derived state placement decisions in a case-by-case basis.

The enabling technique that this thesis presents is composition-based design:
a query engine architecture can be constructed by composing building block components that encapsulate primitive query processing tasks into a directed acyclic graph that provides higher-order query processing capabilities.
We introduce a query processing component abstraction, the Query Processing Unit (QPU), that defines a uniform interface and interaction semantics for query processing architecture building blocks.
This uniform interface and interaction semantics allows us to expose design decisions about the query engine’s architecture and placement to application developers.

Finally, we present an implementation of the proposed approach, in the form of a framework for constructing and deployment application-specific query engines, called Proteus.
Proteus consists of an extensible library of Query Processing Unit implementations, and mechanisms for facilitating the definition and deployment of QPU-based query engines.

The experimental evaluation supports the theoretical analysis of the trade-offs involved in query processing state placement, and suggests that Proteus can effectively occupy multiple different points in the design space of geo-distributed query processing.
}

\afterpage{\blankpage}

\chapter*{}
\thispagestyle{empty}

\begin{flushright}
\begin{minipage}{8cm}
\begin{center}
{\Large To my family.}
\end{center}
\end{minipage}
\end{flushright}
\normalsize\vfill

\afterpage{\blankpage}

\chapter*{Acknowledgements}

One of my primary drivers for this work has been to contribute something that would be useful to other people.
If I am even a bit successful in this, it is thanks to the collaboration, discussions and support from my
advisors, colleagues, friends, and family.
This section is my attempt in thanking each and every one of them.

\small
It feels right to start by thanking my family. I want to thank my father, Kostas, for passing on his life values to me from an early age,
and by that making me who I am today.
To my mother, Georgia, and my sister, Christina, thank you for your care and support at my every step.

\bigskip
This work would not have been possible without the guidance, help and support of my advisors, Marc Shapiro and Brad King.
I am thankful to them for dedicating much of their time for this work,
and for always being available and helpful when I needed help.
Marc, thank you for inspiring me and improving this work by always pointing out some previously unidentified
design flaw or incorrect assumption at my every idea.
Brad, thank you for teaching me something new about systems, their use, research, and even fluid dynamics at our every
discussion.
Thank you to both for teaching me how to do systems research and how to communicate this research so that it can useful
to others.

\bigskip
I would like to thank the reviewers, S{\'e}bastien Monnet and Nuno Pregui\c{c}a for dedicating the time
to evaluate this thesis.
I also thank the examiners, Bernd Amann, Bettina Kemme, Themis Palpanas and Masoud Saeida Ardekani for accepting to be
part of my thesis committee.
A special thanks to S{\'e}bastien and Themis for also accepting to be part of the thesis monitoring committee,
and for offering their valuable feedback.

\bigskip
Research is not a solitary endeavour.
Good research is only made possible through discussion and collaboration.
I am fortunate to have been working among two great teams: Scality and the LIP6 laboratory.

I want to thank my colleagues at the Delys team.
Alejandro, Beno\^it, Francis, Ilyas, Jonathan, Laurent, Michal, Paolo, Saalik, Sara, Sreeja, Vincent, Vinh, thank you
for all the interesting discussions we have had over the years, and the time spent together.

Special thanks to Sara Hamouda for all her inputs, help and support with this work, and for being a great co-author.

// too much?
Some special nods to
Alejandro for the lunches at the Chinese restaurant,
Ilyas for the Wednesday night runs,
Jonathan whose the never-ending ``I have something interesting to show you'' have been the source of interesting and fun conversations,
Saalik for the interesting discussions on technology, football and pop culture over lunch and coffee,
and Sreeja for being someone I look up to as a researcher and a human being.
//

I would also like to thank the other Ph.D. students and interns at the LIP6 laboratory
Alexandre, Antoine, Arnaud, C\'edric, Damien, Daniel, Darius, Florent, Guillaume, Hakan, Jo\~ao Paulo, Lucas, Ludovic, Marjorie, Maxime, Redha and S\'ebastien,
as well as the senior researchers, Jonathan, Julien, Luciana and Pierre, with whom I have shared my days at the laboratory.

I am grateful to the Scality family. I want to thank the Ironman team for welcoming me early on, and Vinh Tao and
Vianney Rancurel for guiding me in my first steps in this thesis.
I would like to thank the File Squad for welcoming me more recently, and for supporting and teaching me about working within
an engineering team, and building a great product.
A special thanks to Aline and Marina for supporting me through the challenges of moving to France.
Thank you to Charles, Dogara, Duhamel, Florent, Greg, Imane, Jordi, Romain, Roxanne and St\'ephanie for the workout
sessions.
Also thanks to Alexandre, Anurag, Jordi and Partick for introducing me to bouldering.

In addition, I would like to thank all the members of the LightKone European project for the exchanges and collaboration.

\bigskip
Georgios Goumas, Vangelis Koukis and Nectarios Koziris from the National Technical University of Athens in Greece inspired
me and sparked my interest for understanding and building computing systems.
I also want to thank Stefanos Gerangelos for introducing me to the path of research, and guiding me through the first steps.

\bigskip
I am grateful to my close friends for accompanying me through this journey, and for supporting me,
each in their own way,
Aggeliki, Alexandros, Andreas, Argyro, Faidra, Giorgos K., Giorgos S., Giorgos V., Iliana, Kostis, Margarita, Maria and Thanos.

Finally, I owe a great deal to Kalliopi for her patience and understanding, for always being there to listen and for
her ability to lift my spirits no matter the circumstances.

% \listoftodos

\pagenumbering{arabic}

\tableofcontents

\afterpage{\blankpage}

\chapter{Introduction}
\label{ch:intro}
The goal of this chapter is to motivate and define the research problem that
this thesis addresses, and to introduce its contributions.

\section{Problem Statement}

\section{State-of-the-art}

For motivating the work, we will state the relevance and timeliness of two
application needs / market trends / ... : attribute-based data retrieval, and
geo-distribution of data.

Attribute-based data retrieval: \\
Retrieving data based on descriptive attributes attached to them (as opposed to
retrieval based on primary key or content-based retrieval).

Data geo-distribution: \\
At the same time, data is increasingly geo-distributed (and federated).

Problem definition: \\
This thesis studies how attribute-based retrieval can be performed efficiently
on geo-distributed data.


\section{Contributions}
\subsection{A design pattern for configurable query processing architectures}
Chapter~\ref{ch:design_pattern}

\subsection{Proteus: Towards application-specific query processing systems}
Chapter~\ref{ch:proteus}

\chapter{Preliminaries}
\label{ch:models}
\begin{tcolorbox}
\textbf{Outline Description - Internal}
The goal of this chapter is to prepare the ground so that the reader has the mental framework needed to understand the
rest of this document.

This includes:
\begin{itemize}
  \item Establishing the terminology to be used throughout the document.
  \item Establishing a ''world view'' by describing the system and data model that we consider, and the assumptions that
  we make
  \item Establishing concepts that will guide our design decisions in the following chapters (performance metrics etc.)
\end{itemize}
\end{tcolorbox}


\section{Models}
We consider a data serving system with a two-tiered architecture: a data storage tier and a query processing tier.

\begin{itemize}
  \item The data storage tier is responsible for storing and providing access to data.
  We use the terms \textit{storage system} or \textit{data store} for the system that implement this tier, and the terms
  \textit{base data} or \textit{corpus} for its data.

  \item The query processing tier is responsible for providing the functionality to identify and retrieve data using
  queries on secondary attributes (Section~\ref{subsec:query_prcessing_tier}).
  We use the terms \textit{query engine} or \textit{query (processing) system} for the system that implements this tier.
\end{itemize}

Disaggregating query processing from the storage engine is an approach used by various systems and cloud
services such as Amazon Athena \cite{aws:athena}, Aurora \cite{aws:aurora}, and Google BigQuery
\cite{google:bigquery}.
In these systems, query processing is performed by a query engine independent from the storage system.

This model provides several benefits:
\begin{itemize}
  \item Storage and query processing resources can scale independently (elasticity).
  \item It enables ad-hoc, one-time queries on already existing data without the need to migrate data.
  For example, performing log forensic queries to investigate an incident.
  \item It enabling cloud providers to implement fine-grained pricing for querying services.
\end{itemize}

In this work, we focus on the design of the query processing tier and consider the data storage tier as ``imposed'':
we consider the data storage tier's functionalities, guarantees, and data distribution schemes as input parameters in
our design.

\bigskip

\noindent
\textbf{Data and control flow} \\

\noindent
Data and control flow between users and the system's tiers as follows: (TODO: need figure here.)
\begin{itemize}
  \item A user submits a corpus data update to the data storage tier.
  As a result, data flows from the user to the data storage tier, and then optionally to query processing tier.
  The query processing tier may use this data to update data structures such as indexes or materialized views.

  \item A user submits a query to the query processing tier.
  As a result, control messages linked to the query execution flow through the query processing tier, and potentially
  for the query processing to the data storage tier (a query processed using cached data does not require control flow
  between tiers, while a query that requires reading from base tables does).
  Data --- in the form of intermediate or final query results flow from the storage tier to the query processing tier to
  the user.
\end{itemize}

The bidirectional flow of data and control messages the fundamental property of the presented system model, and guides
our query processing tier design, as discussed in section TODO.

\subsection{Data storage tier}

The data storage tier is a broad abstraction that can include any system that can be used to store and retrieve data.
This can include databases, file systems, cloud object storage systems.

As described above, in this work we view this tier as an external, underlying system that our design can build upon.
However, any efficient query processing tier design needs to take into account the properties and characteristics of the
data storage tier, most importantly its data distribution scheme.
To address this, we adopt a narrower data storage tier definition.
More specifically, we model a data storage tier as a federation of geo-replicated storage systems.
In the following section, we present our data storage tier model in detail.


\subsubsection{System model}
The data storage tier is implemented by a storage system (a database, file system or cloud storage system) or a
federation of multiple, potentially heterogeneous storage systems.
Each system is responsible for storing a large corpus dataset $D$, which is updated by a stream of small changes.
In a federated system, the dataset of each storage system is independent, but the application logic may implement
replication across different systems (for example an application that replicates data across multiple cloud providers).
In addition, storage systems implement replication:
each storage system maintains one or more full replicas of its dataset, and may also maintain partial replicas.
Finally, storage systems may employ sharding:
each replica partitions its data in non-overlapping parts, and a shard is responsible
for a part of the data.
Replication and sharding are independent, orthogonal techniques, but are often both used in large-scale systems.

We model the infrastructure on which the data storage tier runs as a collection of \textit{sites}.
A site is a group of nodes (servers, user devices) with the following characteristics:
\begin{itemize}
  \item Network communication latency between nodes in different sites is significantly higher (typically an order of
  magnitude higher) \cite{pbailis:hats} compared to communication latency within a site.
  \item Network bandwidth is high within a site and more limited and costly across sites.
  This is reflected in the pricing for cross-region data transfer in public cloud platforms.
  Using the AWS Pricing Calculator \cite{aws:costcalc} we see that data transfer to different AWS data centers
  (regions) costs double the price of intra-DC data transfer (0.02 and 0.01 USD per GB respectively).
\end{itemize}

A site can correspond to a data center, a group of servers serving as an edge Point of Presence \cite{google:infra}, or
a group of user devices in close proximity (in the same room or building).


\subsubsection{Data model}
The corpus dataset is a collection of \textit{data items}, organized in \textit{tables}.
We use the term \textit{data item} to refer to the unit of stored data:
Depending on the storage system, a data item may correspond to a file, an object, or a database record.
Tables may be organized in a hierarchical structure (file system directories), a flat namespace (object store buckets),
or in a relational schema.

A data item is composed of a primary key and a set of attributes.
The primary key can be used to efficiently identify and retrieve the data item, without requiring a scan.
Attributes are key-value pairs: each data item is associated with a map of attribute keys and values
$\{AttrKey: AttrVal\}$.

Our main focus is on data item attributes: we consider that updates and queries refer to attributes.
We do not assume a strict schema for attributes: the attributes of each data item are independent of the attributes of
others.
Finally, the data item may contain a value (content), such a an image, video, or PDF file.
We treat this content as one of the data item's attributes.
(
TODO: should we differentiate it? having the content as a normal attribute creates the assumption that it can be part of
query processing as any other attribute. We haven't considered in our design consideration of image / video / ... search.)

This data model can express the data model of multiple different types of storage systems:
\begin{itemize}
  \item Object stores, such as AWS S3: \\
  The data storage tier is implemented as an object store.
  In that case, a data item corresponds to an object and a tables to a bucket.
  In addition, a data item's ``content'' attribute corresponds to the object's content, and the rest of the attributes
  correspond to object tags \cite{awss3:tagging}.

  \item Wide-column stores, such as Amazon DynamoDB, Google's Bigtable or Apache Cassandra: \\
  In a data storage tier implemented by a wide-column store:
  \begin{itemize}
    \item Data is organized in tables.
    \item Data items correspond to table rows and attributes correspond to columns.
  \end{itemize}
  \item Relational databases: \\
  In a data storage tier implemented by a relational database each table record corresponds to a data item, and each
  table column corresponds to an attribute.
  The relational schema can be represented by enforcing a corresponding schema for the attributes of data items in each
  table.

  \item Document-oriented databases, such as MongoDB: \\
  Document-oriented databases (or document stores) are one of the categories of the group of databases termed NoSQL
  databases.
  The are designed to manage information in the form of documents (semi-structured) data.
  Documents encode data in some standard format such as XML, YAML, JSON or BSON.
  TODO: now introduced in the background chapter. Maybe remove general description from here.

  The described data model is partially compatible with the document store data model:
  tables correspond to document collections, and data items correspond to documents.
  A document's identifier can be represented as a data item's primary key, while document attributes can be represented
  as attributes.
  However, the described data model cannot express complex attribute types such as lists and maps, and nested attributed,
  which are used in the document formats used in document stores.

  \item File systems: \\
  In a data storage tier implemented by a file system, data items correspond to files and tables correspond to file
  system directories.
  A data item's primary key corresponds to the corresponding file's path, the ``content'' attribute to the file's content,
  and rest of attributes correspond to extended file attributes.
\end{itemize}

This general data model, which is able to express different existing data models, satisfies the design goal mentioned
above:
Allowing the query processing tier to be independent from the underlying data storage tier, and be compatible with
different storage tier implementations. \\

\noindent \textbf{Timestamps} \\
We assume that objects are ``versioned''.
Each object is associated with a timestamp, which is one of its attributes.
A timestamp can be implemented by any data type that provides a partial of total order,
such as Unix time or vector timestamps.


\subsubsection{Data storage tier API}
One of our design goals
(TODO: link to design goals in the next chapter)
is that the query processing tier should be agnostic of the data storage tier.
A query system should be able to interoperate with multiple different data storage tier implementations.
To achieve that, we model the interconnection between the data storage and query processing tier as a set of well-defined APIs.
This allows the query processing tier to be agnostic of the data storage tier, and compatible with any storage system
that exposes the described APIs.

% As an example, the query processing tier can work on top of a streaming system that does not persist data but exposes the $Subscribe$ API.

To be compatible with our design, a data storage tier need to expose some the following APIs:
\begin{itemize}
  \item An API for iterating over the corpus data (List).
  This API enables the query processing tier to access corpus data.
  It can be used to implement query processing tasks such as iterating over a table's data items and filtering those that
  match a given predicate, or performing a join over two tables.
  \item An API for subscribing to notification for changes to the corpus data (Subscribe).
  This API enables the query processing tier to receive a constant stream of notifications for corpus data changes,
  which can be used to incrementally maintain data structures such as indexes and materialized views.
  \item An API for querying the corpus data (Select).
  This API allow the query processing tier to make use of the querying capabilities of the data storage tier.
\end{itemize}

List and Subscribe are complementary, and are both needed for a fully featured query processing tier.
Select is operational, as it can be viewed as a generalization of List (a List is a Select without predicates)
Our query processing tier design is compatible with data storage systems that expose only one of the described APIs,
albeit with some limitations.

In short:
\begin{itemize}
  \item When only the List API is available, the query processing tier can perform any querying task, but the use of
  derived data for optimizations is limited.
  The query processing tier can build indexes and materialized views by scanning the corpus data, but updating those data
  structures to reflect corpus data changes requires a full rebuild.

  \item When only the List API is available, the query processing tier can perform query processing tasks only by
  building a derived view of the corpus data (build materialized views).
  Depending on the semantics of Subscribe, the query processing tier may only be able to receive information for corpus
  updates with timestamps starting from the invocation of the Subscribe API, making the query processing tier not
  possible to deploy over already existing data.
\end{itemize}

In chapter
TODO,
we discuss the implications of the data storage tier exposing only one of the APIs to the query processing tier
functionality.
In addition we discuss the implications of $List$ and $Subscribe$ not supporting a range of timestamps but rather
returning results about the most recent timestamp, and how the query processing tier can make use of the $Select$ API.
For the rest of this document, we assume, unless stated otherwise, that the data storage tier exposes the $List$ and
$Subscribe$ APIs.

\bigskip

\noindent
\textbf{API 1: List}

\noindent
This API provides a mechanism for retrieving the primary key and attributes of all objects in a given table:

\[
  List(Table, [Timestamp_{low}, Timestamp_{high})) \rightarrow [ListResponse]
\]

\noindent
\begin{sloppypar}
Given a table name ($Table$) and a range of timestamps ($[Timestamp_{low}, \linebreak Timestamp_{high})$),
$ListResponse$ contains all data items in $Table$ with $Timestamp_{low} \leq Timestamp < Timestamp_{high}$.
Each data item in $ListResponse$ is represented as a tuple $(ID, \{AttrKey: AttrVal\})$, containing the data item's
primary key ($ID$), and attributes ($\{AttrKey: AttrVal\}$).
\end{sloppypar}

$ListResponse$ may be implemented in different ways, such a single response containing a set of
data items, or a stream in which each data item is sent as a record, or an iterator in which calling a $Next()$ method
returns the following data item.
Finally, we don't assume any ordering in $ListResponse$.

The $List$ API may be provided as an explicit API method or implemented as a combination of a \textit{list} and a
\textit{read} (get) operation.

Depending on its versioning mechanism, the storage system implementing the data storage tier may not support the
listing API for any range of timestamps.
For example, in a storage system that does not provide multi-versioned storage, $List$ will return the latest version of
each data item.
For simplicity, we assume the $List$ API as specified above.
In Section
TODO
we describe how our design can use the $List$ API without support for Timestamp ranges.

\bigskip

\noindent
\textbf{API 2: Subscribe}

\noindent
This API provides a mechanism for subscribing to notification for updates to data items in a given table:

\[
  Subscribe(Table, [Timestamp_{low}, Timestamp_{high})) \rightarrow [SubscribeResponse]
\]

\noindent
\begin{sloppypar}
Given a table name ($Table$) and a range of timestamps ($[Timestamp_{low}, Timestamp_{high})$),
$SubscribeResponse$ contains a record for each update performed in a data item $d$ in $Table$, with
$Timestamp_{low} \leq Timestamp < Timestamp_{high}$.
$SubscribeResponse$ records have the form $(ID, \{AttrKey: (AttrVal_{old}, AttrVal_{new})\})$:
each record contains the update data item's attributes before and after the update.
If an operation updates more than one data item, then $SubscribeResponse$ contains a record for each updated data item.
If an attribute is created by the update, then its old value has the special value $null$.
Conversely, if an attribute is deleted by the update, then its new value has the $null$ value.
\end{sloppypar}

$SubscribeResponse$ can be implemented as a stream:
An invocation of $Subscribe$ returns a stream handler, and a record is sent to stream for each corresponding update.

Various systems provide mechanisms that can be used to implement the $Subscribe$ API.
Examples include triggers in traditional database management systems \cite{mariadb:triggers}, and event notification
mechanisms in cloud storage services \cite{awss3:notifications}.

\bigskip

\noindent
\textbf{API 3: Select}

\noindent
This API provides a mechanism for identifying data items based on their attributes:

\begin{sloppypar}
\[
  Select(Table, Predicate, [Timestamp_{low}, Timestamp_{high})) \rightarrow [SelectResponse],
\]
where $Predicate$ is a map containing attribute keys and ranges:
$\{AttrKey: [AttrVal_{low}, AttrVal_{high})\}$.
\end{sloppypar}

\begin{sloppypar}
Given a table name ($Table$), a range of timestamps ($[Timestamp_{low}, Timestamp_{high})$), and
a predicate consisting of attributes keys and ranges of values
($\{AttrKey: [AttrVal_{low}, AttrVal_{high})\}$), $Select$ returns $SelectResponse$.
$SelectResponse$ contains all data items in $Table$ that have all attributes contained in $Predicate$ and
their values are within the ranges specified in $Predicate$.
\end{sloppypar}

$Select$ can be expressed as an SQL query with the form: \\
\noindent
$SELECT$ $AttrKey_1$, $AttrKey_2$, ..., \\
$FROM$ $Table$ \\
$WHERE$ \\
$AttrKey_1$ $IS$ $NOT$ $NULL$ $AND$ $AttrVal_{low}$ $\leq$ $AttrKey_1$ $<$ $AttrVal_{high}$
$AND$ $AttrKey_2$ $IS$ $NOT$ $NULL$ $AND$ $AttrVal_{low}$ $\leq$ $AttrKey_2$ $<$ $AttrVal_{high}$
$AND$ ... \\
$AND$ $Timestamp_{low}$ $\leq$ $Timestamp$ $<$ $Timestamp_{high}$

$Select$ may be provided as a subset of a more expressive query language, such as in relational database
systems that support a full SQL query language. \\

\subsection{Query processing tier}
\label{subsec:query_prcessing_tier}
The design decisions and trade-offs involved in the design of the query processing tier is the main focus of this work.
In this Section we present an overview of the query processing tier's role and functionality.
The following chapters present our query processing tier design in more detail.

The query processing tier is responsible for providing attribute-based data retrieval:
Identifying and retrieving data items based on queries on their attributes.

As the query language, we consider the subset of SQL that supports expressions of the form: \\

\noindent
$SELECT$ $projection$ $FROM$ $Table$ $WHERE$ $predicate$,

\noindent
where
% \begin{itemize}
%   \item $projection$ is a list of attributes, $AttrKey_1$, $AttrKey_2$, ..., $AttrKey_N$.
%   \item $predicate$: has the form: \\ $rangePredicate_1$ $AND$ $rangePredicate_2$ $AND$ ... $AND$ $rangePredicate_N$, \\
%   where $rangePredicate_i$ an expression of the form $AttrVal_{low}$ $\leq$ $AttrKey$ $<$ $AttrVal_{high}$.
% \end{itemize}

TODO: this needs some work to define exactly what is and is not supported (we now support more than previous iterations, for example some simple joins).

Given a query $Q$ = ($projection$, $table$, $predicate$)
and a data item $d$ = ($table$, $id$, $attributes$), $d$ satisfies $Q$ if:
\begin{itemize}
  \item $Q.table$ = $d.table$: the data item belongs in the table referred by the query, and
  \item $\forall$ ($AttrVal_{low}$ $\leq$ $AttrKey$ $<$ $AttrVal_{high}$) $\in$ $Q.predicate$:
  $attrKey$ $\in$ $d.attributes$ and $AttrVal_{low}$ $\leq$ $AttrVal$ $<$ $AttrVal_{high}$:
  the data item contains all attributes included in the query's predicate, and their values are within the ranges
  specified by the predicate.
\end{itemize}

$Predicate$ may contain expressions that refer to the $Timestamp$ predicate.
In that case, a data item satisfies $Q$ if, in addition to the above conditions, its $Timestamp$ satisfies the
$Timestamp$ -related expressions in $predicate$.
If no $Timestamp$ -related predication is given, then only the latest version of each data item is considered for the
query.

We define the set of data items that satisfy a query $Q$ as $Q_r$.
In the ideal case, $Q$'s response is $Q_r$, however, as we explain in the following section a query response may
differ from $Q_r$. \\

\noindent
\textbf{Derived data} \\
Query systems often maintain data such as indexes, materialized views, and caches.
We refer to this data as \textit{derived} data.
Derived data, in the general case, is obtained by performing a computation over the base data.
When base data changes, the query system need to update derived data to reflect theses changes.
After a base data updates, indexes and materialized views need to be update so that the reflect the new state of the corpus,
and obsolete cache entries need to be invalidated.
As we describe in the following section, inconsistencies between base and derived data may result to divergence between
query responses and $Q_r$, as $Q_r$ is defined using the most up-to-date state of the corpus at the time of a query, but
query processing may be based on an outdated view of the corpus data.

\section{Query processing system performance evaluation}
\label{sec:requirements}

% An important first step in the process of designing a query processing system (and any system in general) is determining
% the factors and metrics that will be used to evaluate how well the design achieves its goals.

The aspects of a query processing system's performance can be categorized in two groups:
\textit{efficiency} and \textit{effectiveness} \cite{buttcher:informationretrieval}.
We can measure efficiency with metrics such as response time, throughput, and scalability.
Effectiveness is a measure of how well a query processing system achieves its intended purpose.
It involves metrics such as precision (the fraction of useful information returned by the query) and recall
(the fraction of data items in the corpus that satisfy a query returned by the query).

Finally, two other important factors in the design of query processing systems are availability and operational cost.
Availability is important, due to the negative effects of downtime in user serving systems.
It is especially relevant in the design of distributed systems due to the many different faults that can impact the
operation of a distributed system \cite{kleppmann:designing}.

\todo{Cost ...}

\subsection{Evaluating Efficiency}

The most visible aspect of efficiency is the \textit{response time} experienced by a user between issuing a query and
receiving the corresponding response.
Since the query system may need to support many simultaneous users, \textit{query throughput}, measured in queries served
per unit of time, is an important performance metric for the query processing system.
In addition, as response time is the measure that directly affects user experience, an important efficiency metric is
how response time scales with query throughput.
This measures whether the system can maintain low query response time for individual users as the number of users
increases.
The relations between response time and query throughput, and between throughput and system load (number of users),
characterize the system's \textit{scalability}.

\subsubsection{Response time}
Response time --- the amount of time between making a request and receiving the corresponding response ---
is among the most important metrics for the quality of a user-facing service.

A number of studies and experiments have studied the effects of response time to user experience.
Results show that response time is among the factors that have the largest effect users' subjective perception of the
quality of a system.
Users have been shown to perceive websites that load faster as more interesting \cite{ramsay/retrievaltimesinvestigation}.
On the other hand, long response times increase user frustration \cite{ceaparu:userfrustration} and even compromise
user's conceptions of the security of the system \cite{bouch:qualityeyebeholder}.

Industry reports have indicated that even small increases in user-perceived response times can result in drops in web
traffic, and therefore sales.
Experiments by the Google and Bing search engines have shown that longer page loading times have a significant impact on
metrics such as time to click, repeated site usage, and queries per visit \cite{schurman:rerformanceuserimpact}.
A study from Akamai on the impact of travel site performance on consumers showed that more than half of the users will
wait three seconds or less before abandoning the site \cite{akamai:travelsiteperformance}.
Finally, a comparison shopping service (Shopzilla) has reported that a website re-engineering project that achieved a
speedup in page load time from 6-9 seconds down to 1.2 seconds resulted in 25\% increase in page views and 5-12\%
increase in revenue \cite{dixon:shopzillasiteredo}.

\subsection{Evaluating Effectiveness}

Effectiveness is a measure of how well a query processing system achieves its intended purpose.

\subsubsection{Recall and precision}

In the field of information retrieval, which covers the problems associated with searching human-language data,
the key notion linked with effectiveness is \textit{relevance} \cite{buttcher:informationretrieval}.
In information retrieval, given a user's information need, represented by a search query, the \textit{search engine}
(the system responsible for query processing) computes a relevance \textit{score} for each document (e-mail message,
webpage, news article), and returns a ranked list of results.

Recall and precision are metrics often used to measure the query results' relevance:
\begin{itemize}
  \item Recall is the fraction of the relevant that are returned by the query.
  A recall value equal to 1 indicates that all relevant documents are returned by the query
  A recall value of less that 1 indicates that some relevant documents are not returned (``false-negatives'').

  \item Precision is the fraction of relevant documents among the documents contained in the query result.
  A precision value equal to 1 indicates that all documents returned by the query are relevant.
  A precision value of less than 1 indicates that some of the returned documents are not relevant (``false-positives'').
\end{itemize}

The difference between the information retrieval query model, and the query model that we consider in this work is the
notion of relevance.
In information retrieval, relevance is a spectrum: documents can be more or less relevant to a given query, which
represented by their \textit{score} for that query.
Here, relevance is binary: a data item is either relevant (satisfies the given query) or it is not.
However, as mentioned in Section \ref{subsec:query_prcessing_tier}, similarly to information retrieval, query results
can include non-relevant data items (false positives), or not include relevant data items (false-negatives).
Therefore we argue that the recall and precision are useful metrics in evaluating the effectiveness of the query
processing tier.

\subsubsection{Freshness}

As mentioned in Section \ref{subsec:query_prcessing_tier}, inconsistencies between base and derived data can lower query
processing effectiveness by introduction false-positives and false-negatives.

Traditional database systems often keep derived data consistent with base data by updating both in a single transaction.
For example, when executing an $UPDATE$ statement, MariaDB updates a table's secondary indexes in the same transaction
as the table rows \cite{innodb:writepaths}.
However, in systems that implement asynchronous (lazy) derived data maintenance policies \cite{tan:diffindex,
qi:secondaryindexconsistency, shukla:schemaagnostic} derived data can become stale with respect to base data.

Stale derived data may introduce false-positives and false-negatives to query results:
\begin{itemize}
  \item \textbf{False-positives}: A data item $d$ has been deleted (or updated so that it matches the given query),
  but the corresponding derived data has not yet been updated to reflect this change.
  A query execution that uses the stale derived data will include $d$ in the query response, introducing a false-positive.

  \item \textbf{False-negatives}: A data item $d$ has been created (or updated so that it does not match the given query),
  but the corresponding derived data has not yet been updated to reflect this change.
  A query execution that uses the stale derived data will not include $d$ in the query response,
  introducing a false-negative.
\end{itemize}
This can result in false-positives and false-negatives, affecting the recall and precision of these systems.

We use the notion of \textit{freshness} to refer to the measure of consistency between base and derived data due to
asynchronous derived data updates.

A number of metrics for measuring data freshness have been proposed in the literature \cite{bouzeghoub:datafreshness}:
\begin{itemize}
  \item \textbf{Currency} measures the time between a change in the source data, and that change between reflected in
  the derived data.
  In caching systems, the terms recency \cite{bright:latencyrecency} and age \cite{cho:dbfreshness}
  have been used to describe this metric.
  \item \textbf{Obsolescence} measures the number of updates to source data since derived data was last updated.
  Work on query systems has defined the \textit{obsolescence cost} \cite{avigdor:obsolescent}, of a query to represent
  the penalty of basing a query result on obsolescent materialized view.
  This cost is computed as a function of the number of insertions, updates, and deletions that cause deviation between
  the materialized view and and the base table.
  \item \textbf{Freshness-rate} measures the percentage of derived data entries that are up-to-date with the source
  data.
  This metric has been used to quantify the freshness of web pages \cite{labrinidis:balancingperfomancefreshness} and
  local databases copies \cite{cho:dbfreshness}.
\end{itemize}

\subsection{Other aspects of query processing system design}

\subsubsection{Availability}
The importance of availability becomes apparent when considering the negative effects of service downtime.

A study on user behavior in the Web \cite{nah:waitingtime} found that users abandon a nonworking hyperlink after
5-8 seconds.

Operators of global services understand that ``even slightest outage has significant financial
consequences and impacts customer trust'' \cite{deCandia:dynamo}.
A service 49 minute service outage in January 2013 cost Amazon an estimated \$4 million or more in lost sales
\cite{infoworld:cloudoutages}.

\subsubsection{Operational Cost}
% Operating a query processing system requires computation, memory, network and storage resources.
Another important parameter that drives system design parameters is the system's operational cost.

Traditionally, the system is deployed on dedicated infrastructure, and the operational cost is the cost of owning and
operating that infrastructure.

More recently, the infrastructure and platform as a service cloud computing models providing flexible, fine-grained
provisioning of computing, storage and networking resources.
In addition, these services provide fined-grained, ``pay-as-you-go'' pricing mechanisms.
As a result, system designers can have more control over the system's operational cost.

A typical cloud pricing model \cite{aws:pricing} has distinct pricing for (1) computation and memory resources
(vCPUs and memory), (2) persistent storage, and (3) data transfer.

In chapter
TODO
we describe how system design decisions can affect the system's resource utilization, for example limiting the cross-DC
communication, and therefore its operational cost.

\bibliographystyle{plainnat}
\bibliography{refs}

\chapter{Background}
\label{ch:background}

Query processing refers to the range of activities involved in extracting data from a database.
These activities include the translation of queries from high-level languages into formats that can be processed
by the database, query-optimizing transformations, and actual evaluation of queries.

Query processing has been studied extensively in the context of relational database systems.
Relational databases provide sophisticated querying capabilities and require complex query processing techniques
to connect declarative query languages to efficient query execution.

On the other hand, databases that belong in the class of nonrelational database systems (also referred to as NoSQL)
in general make design decisions that favor scalability (very large datasets or very high write throughput) over rich
query models, often supporting only primary key lookups.
There, query processing is mainly linked to data distribution schemes and involves the use of secondary indexes.

The purpose of this chapter is to present an overview of several ``textbook'' query processing techniques.
Our work is influenced by and builds on top of the techniques presented in this chapter.

\section{Query Processing in Relational Database Systems}

The database component responsible for query processing is the \textit{query processor}.
The role of a relational query processor is, given a declarative high-level statement (often SQL), to validate it,
optimize it into a procedural dataflow execution plan, and execute that dataflow program.

Query processing consists of the following phases \cite{hellerstein:databasearchitecture, kossmann:distqeuryprocessing}:

\bigskip
\noindent
\textbf{Query Parsing.}
The goal of the query parsing is to translate a given query into an internal representation that can
be processed by later phases, commonly a tree of relational operators \cite{silberschatz:dbbook}.
In generating the internal form of the query,
the query processor checks the syntax of the given query,
verifies that the relation names that appear in the query are valid names of relation in the database,
and verifies that the user is authorized to execute the query.

\bigskip
\noindent
\textbf{Query Rewrite.}
In the query rewrite phase, the query processor transforms the query in order to carry out optimizations that are
beneficial regardless of the physical state of the system
(the size of tables, presence of indices, locations of copies of tables etc.)

Typical transformations include:
\begin{itemize}

  \item \textbf{Elimination of redundant predicates and simplification of expressions:}
  This includes the evaluation of constant arithmetic expressions,
  short-circuit evaluation of logical expressions via satisfiability tests,
  and using the transitivity of predicates to induce new predicates.
  Adding new transitive predicates increases the ability of the following phase (query optimization) to construct query
  plans that filter data early in execution, and make use of index-based access methods.

  \item \textbf{View expansion, sub-query un-nesting:}
  For each view reference that appears in the query, the query processor retrieves the view definition and rewrites the
  query to replace that view with its definition.
  In addition, this phase flattens nested queries when possible.

  \item \textbf{Semantic optimization:}
  In many cases, integrity constrains defined by the schema can be used to simplify queries.
  An important such optimization is redundant join elimination (for example, a query that joins two tables but does not
  make use of the columns of one of the tables).

\end{itemize}

\bigskip
\noindent
\textbf{Query Optimization.}
In the query optimization phase, the optimizer (the query processor component responsible for query optimizations)
transforms the internal query representation into an efficient plan for executing the query.

A query plan can be thought of as a dataflow diagram that specifies how the query is to be executed.
A common approach to represent a query plan is using a tree:
the tree's nodes are operators
--- each operator carrying out a particular operation such as join, group by, sort, scan, etc. ---
and edges represent consumer-producer relationships between operators.

The optimizer is responsible for decisions such as which indices to use to execute a query,
which methods to use to execute the operators of a query,
and in which order to execute a query's operations.
In a distributed system, the optimizer also decides at which node each operation is to be executed.

The foundational paper by Selinger et al. on System R \cite{selinger:systemr} decomposes the problem of query optimization
into three distinct sub-problems: cost estimation, relational equivalences that define a search space, and cost-based
search.
The optimizer assigns a cost estimate to the execution of each component of a query, measured in terms of I/O and CPU
cost.
To do so, the optimizer relies on pre-computed statistics about the contents of each relation, and heuristics for
determining the cardinality of the query output.
It then uses a dynamic programming algorithm to construct a query plan based on these cost estimates.

\bigskip
\noindent
\textbf{Query Execution}
Query execution is carried out by the \textit{query execution engine}.
The query execution engine is the query processor's component that provides implementations for every operator.

The most common approach used to implement query operators is the iterator model \cite{graefe:queryevaluation}.
Iterators can be described using the object-oriented paradigm.
Figure~\ref{lst:iterator} shows a simplified definition for a general iterator class.

\begin{lstlisting}[caption={Iterator class pseudocode \cite{hellerstein:databasearchitecture}},label={lst:iterator},captionpos=b,language=Java]
class iterator {
  iterator &inputs[];
  void init();
  tuple get_next();
  void close();
}
\end{lstlisting}

The iterator model has certain useful properties for modeling query execution:
\begin{itemize}

  \item All iterators have the same interface.
  As a result, a consumer-producer relationship can exist between any two iterators, and thus, any plan can be executed.
  In addition, the common interface means that each iterator's logic is independent of its children and parents in the
  graph.

  \item Τhe iterator paradigm supports pipelining of results from one operator to another.

\end{itemize}

\subsection{Materialized Views}

An important element of the relational model is the \textit{view}.
A view is a ``virtual relation'' defined by a query that conceptually contains the result of that query.
Views are not precomputed and stored; the database stores only the query defining the view.
Each time the view is used, the database expands it on the fly into the underlying query,
and then processes the expanded query.

In contrast, a materialized view is a view whose contents are pre-computed and stored (cached?) by the database.
In many cases reading the contents of a materialized view is much more efficient than computing the contents of the view
by executing the query that the view defines.
Essentially, a materialized view is a precomputed cache of query results.
The use of materialized views is a common technique for reducing query response time.

\subsubsection{View maintenance}

An important aspect of materialized views is that when the data referred in the view definition changes,
the view must be kept up-to-date.
A simplistic way of achieving this to recompute the materialized view on every update.
A better option is to, on each update, modify only the affected parts of the view.
This approach is known as incremental view maintenance.
There is considerable research on incremental view maintenance for in relational databases
\cite{larson:outerjoinviewmaintenance, lee:multiplejoinviewmaintenance, zhuge:viewmaintenance}.

Another design decision when incremental view maintenance is used, is when to perform the maintenance:
in \textit{synchronous} view maintenance, view maintenance is performed as soon as an update occurs,
as part of the updating transaction,
while in \textit{asynchronous} or lazy view maintenance,
updating the view is deferred to a later time \cite{zhou:lazymvMaintenance}.
Materialized views with deferred view maintenance may be somewhat out-of-date with the underlying data.

\subsubsection{Query Optimization and Materialized Views}

Materialized views add further consideration to query optimization:

\begin{itemize}

  \item Rewriting queries to use materialized views.
  The query processor may produce a more efficient query plan by rewriting the query to make use of an available
  materialized view.

  \item Replacing the use of a materialized view with its definition.
  Sometimes, using the query that defines a materialized view instead of directly reading from its pre-computed contents
  may offer more optimization options.
  For example, consider a case in which a materialized view does not include indexes that can be used to speed up the query,
  but the underlying relations do.
  Using the views definition instead of its contents enables query execution to take advantage of those indexes.

\end{itemize}

\subsubsection{Materialized View Selection}

Materializing an appropriate set of views and processing queries using these views can significantly speed up
query processing since the access to materialized views can be much faster than recomputing the views.
In principle, materializing all queries that a system may receive can achieve the optimal query response time.
However, maintaining a materialized view incurs a maintenance cost.
In addition, query results may be too large to fit in the available storage space.
There is therefore a need for selecting a set of views to materialize by taking into account query processing cost,
view maintenance cost and storage space.
The problem of choosing which views to materialize in order to achieve a desirable balance among these three
parameters is known as the view selection problem \cite{gupta:viewselection, mami:viewselection}.

\subsection{Caching}
Another technique for speeding up queries, besides the use of indexes and materialized views, is caching:
storing the result of a costly computation so that repeated reads of this result can be performed without re-executing
the computation, if the underlying data has not change.

A common approach is to deploy a \textit{caching tier} between the database system and the clients.
In-memory key-value stores such as Redis \cite{redis:cache} and memcached \cite{memcached:wiki} are often used for this
purpose.

% However, this approach requires application logic to invalidate or replace cache entries, which may be complex and error prone.

\subsection{Distributed Query Processing}

So far we covered query processing from the perspective of a single-node database, without considering data distribution.
However, data is inherently distributed \cite{bacon:spanner, cockroachdb:docs} and therefore query processing needs to
efficiently operate on distributed data.
In addition, query processing computations need to be able to be distributed and run in parallel on multiple nodes to
achieve better scalability.

In Ingres \cite{epstein:ingres}, relations can be distributed across a collection of ``sites''.
Query processing is based on \textit{decomposing} queries into sub-queries that can be processed on a single site.
The database uses query decomposition heuristics based on two optimization criteria:
minimizing response time and minimizing network traffic.

Spanner \cite{bacon:spanner} is sharded, geo-replicated relational database system.
Spanner uses a \textit{distributed union} operator in the query tree to represent query distribution.
Distributed union is used to a sub-query to each shard of a table, and concatenate the results.
It provides a building block for more complex distributed operators such as distributed joins between independently
sharded tables.

When a query tree is initially created, a distributed union operator is inserted immediately above every table.
In the query optimization phase, where possible, query tree transformations may pull the distributed union operator up
the tree in order to push the maximum amount of computation to the servers.
In the query execution phase, distributed union routes a sub-query request addressed to a shard, to one of the nearest
replicas of that shard in order to minimize latency.

CockroachDB employs a mechanism for distributed query processing \textbf{computation} \cite{cockroachdb:distsql}
(for example join, aggregation, or sorting) on multiple nodes in order to improve performance.
In CockroachDB, a query plan is a tree of operators, termed \textit{aggregators}:
each aggregator consumes an input stream of records and produces an output stream or records.
The key idea is that an aggregator splits the input stream into \textit{groups}:
the computation for each group is independent of the computation for other groups; the output stream is the
concatenation of computation result for all groups.
Since results for each group are independent, different groups can be processed on different nodes.


\section{Query Processing in Non-Relational Database Systems}

The querying capabilities of a non-relational database mainly follow from their distribution model and data model.
Thus different non-relational databases have varying querying capabilities.

To further discuss query processing in non-relational databases,
we first briefly introduce the data models and data distribution techniques used in these systems.

\subsection{Non-relational Database Data models}
\noindent
\textbf{Key-Value Stores.}
A key-value store's data model is a map/dictionary of key-value pairs.
As the structure of values is usually opaque to the database system, this data model only supports get and put operations
(requesting and writing value using a key).
Key-value stores in generally favor scalability over a richer data model and more complex query capabilities:
the simple key-value model makes partitioning and locating data efficient, thus enabling these systems to achieve low
latency and high throughput.

\bigskip
\noindent
\textbf{Document Stores.}
A document store is a key-value store that restricts values to semi-structured formats such as
XML, YAML, JSON or BSON \cite{bson:spec}.
This enables more sophisticated data access capabilities:
apart from retrieving an entire document from its key, documents stores support predicate queries
(retrieving the keys of all documents that match a given predicate), and joins.

\bigskip
\noindent
\textbf{Wide-column Stores.}
The data model of wide-column stores is often depicted as a relational table with many sparse columns.
More accurately, this data model can be described as a distributed, multi-level, sorted map.
The first-level keys identify rows (row keys) and the second-level keys identify columns (column keys).
% In some wide-column stores multi-versioning is implemented by adding third-level, timestamp keys.

\subsection{Partitioning}
\label{sec:partitioning}

Partitioning is a technique for dividing a logical database into smaller distinct parts, called partitions, and spreading
across several nodes.
Partitioning divides both the database's content (corpus) as well as its computations.
Each partition effectively acts as a database of its own, although there may be operations that involve multiple partitions.
Different database systems use different terms to refer to what we here call partition, including \textit{shard}
\todo{MongoDB, elasticSearch}, \textit{region} \todo{HBase}, \textit{tablet} \todo{BigTable} and \textit{vnode} \todo{Cassandra, Riak}.

The goal of partitioning is to spread data and load evenly across nodes.
When implemented efficiently, it enables horizontal scaling:
doubling the number of nodes in the system should make the system able to handle double the volume of data, and
should double the system's read and write throughput.

The partitioning techniques commonly used in non-relational databases are range and hash partitioning.

\bigskip
\noindent
\textbf{Range partitioning.}
Range partitioning assigns an interval of keys to each partition.
These ranges of key are not necessarily evenly spaced, because data might be unevenly distributed.
Partition boundaries might be chosen manually by an administrator, or automatically by the database management system.

Within each partition keys are kept in sorted order.
This has the advantage that range queries on the partitioning key are efficient:
it is easy to determine which partitions contain keys of a given range, and within each partition the key can be treated
as an index.

The downside of this partitioning scheme is that certain access patterns can lead to hotspots.
Therefore, systems that use range partitioning need mechanisms for detecting and resolving hotspots.

Range partitioning is used by Bigtable and its open source equivalent HBase \cite{hbasebigtable:comparison},
RethinkDB, and MongoDB before version 2.4.

\bigskip
\noindent
\textbf{Hash partitioning.}
An alternative approach that avoids the risks of skew and hotspots is to use a quasi-random hash function to determine the partition
for a given key.
Hash partitioning assigns each partition a range of hashes --- rather than a range of keys --- and every key whose hash
falls within a partition's range is handled by that partition.

This partitioning scheme is efficient at distributing keys fairly among partitions.
The downside of this approach is that does not allow for efficient range queries,
as adjacent keys are scattered across multiple partitions.

Hash partitioning is used in Amazon's Dynamo, MongoDB since 2.4 \cite{mongo:hashpartitioning}, Riak, CouchBase,
and Voldemort.

\subsection{Replication}
\label{sec:replication}

Partitioning is usually combined with replication so that copies of each partition are stored on multiple nodes.
Replication improves availability by allowing the systems to continue working even if some of its parts have failed,
and increases read throughput by increasing the number of machines that can serve read queries.

There are multiple different replication strategies.
These strategies can be categorized based on two design decisions \cite{gray:replication}:
\begin{itemize}
  \item How are updates regulated. Is there single ``master'' replica responsible for processing updates to a given data item,
  or can any replica with a given data item update its copy?
  \item Are updates propagated between replicas eagerly or lazily?
\end{itemize}

\bigskip
\noindent
A common approach to the first design decision is called \textit{leader-based} replication.
One of the replicas is designated the \textit{leader} (also termed \textit{master} or \textit{primary}).
Every write is sent to the leader.
The leader determines the order in which writes should be processed,
and sends the corresponding data changes to the other replicas
(termed \textit{followers}, \textit{slaves} or \textit{secondaries}),
Followers apply those changes in the same order.
Reads can be performed from any replica.

This approach is used in MongoDB, RethinkDB, and Espresso.

Leader-based replication has one main downside:
as there is only one leader (when replication is combined with partitioning there is one leader per partition),
and all database writes must go through it, if the leader is unreachable writes cannot be performed.

An extension of leader-based replication is to allow more than one replica to accepts writes.
In \textit{multi-leader} replication there are multiple leaders,
each processing writes and forwarding the corresponding data changes to all other replicas.
Each leader acts also as a follower to the other leaders, accepting writes from them.

An alternative approach, termed \textit{leaderless} replication,
is to allow any replica directly accept writes from clients.
Each write is sent (either by the client, or by a coordinator) to $W$ replicas, and each read is sent to $R$ replicas,
where $W$ and $R$ are configuration parameters.
In order to ensure that eventually all data is propagated to every replica,
leaderless replication implementations often employ two mechanisms:
(1) read repair, a way to detect and update stale values during reads,
and (2) anti-entropy, having a background process that replicates missing data between replicas.

This approach was popularized by Amazon's Dynamo.
Riak, Cassandra, and Voldemort are datastores with leader replication models inspired by Dynamo.

\bigskip
\noindent
There are two approaches for the design decision ``when the leader propagates data changes to followers''.
In \textit{synchronous} (or \textit{eager}) replication the leader propagates changes synchronously
and waits for acknowledgements from followers before reporting success to the user.
In \textit{asynchronous} (or \textit{lazy}) replication the leader propagates changes
and does not wait for responses from followers.

The advantage of synchronous replication is that followers are guaranteed to have copies of the data that are
up-to-date with the leader.
Its disadvantage is that if followers do not respond (due to a crash, network fault or other reasons)
writes cannot be processed.

\bigskip
\noindent
Geo-replication (replication across geographically distributed data centers) can protect the system against data center
failures and network problems,
and improve read latency for clients distributed across multiple geographic locations.
Synchronous geo-replication, as implemented in Google's Megastore \cite{baker:megastore}
and Spanner \cite{corbett:spanner, bacon:spanner},
achieves strong consistency at the cost of high write latency.
In asynchronous geo-replication, as used in Dynamo \cite{deCandia:dynamo}, PNUTS \cite{cooper:pnuts08, cooper:pnuts19},
Walter \cite{sovran:walter}, COPS \cite{lloyd:cops}, Cassandra \cite{lakshman:cassandra}, and Bigtable \cite{chang:bigtable}
the inter-data center network delays are hidden from clients,
and the system remains available during partitions.
The downside of asynchronous geo-replication is that the same data may be concurrently modified in different data centers
creating conflicts that then need to be resolved.

\subsection{Query Processing}

Non-relational database systems in general support two types of queries:

\bigskip
\noindent
\textbf{Primary key lookups.}
In a primary key lookup, a data item is retrieved using its primary key.
This is the main data access method in non-relational databases.
It can be efficiently supported as it is compatible with both hash and range partitioning.

\bigskip
\noindent
\textbf{Filter (predicate) queries.}
A filter query returns all data items from a database table that meet a predicate specified over their properties.
In their simplest form, filter queries can be performed as filtered full-table scans.

\subsubsection{Secondary indexes}
For databases that use hash partitioning a full-table scan implies a scatter-gather operation where each shard
performs a filtered scan, and results from all shard are merged.

A common technique used to support efficient filter queries is the use of secondary indexes (secondary indexing).

A secondary index is a structure that is derived from the primary data, and organizes data in a form that
provides a way to efficiently access database records by means other than the primary key.

Essentially, a secondary index is a key-value structure where the key is a \textit{term} (an attribute or key of a
database record other than the primary key),
and the value is a list of primary keys of all the records that contain that term (a \textit{posting list}).

Secondary indexing is an instance of a general system design pattern:
having the same data represented in different formats to address different access patterns.
Database tables are the primary copy of data.
Derived copies of the data transform the primary copy differently in order to satisfy certain access patterns.
Adding a secondary index does not affect the contents of the database;
it only affects the performance of read and write operations.
Writes go to the primary data and all of the other data copies are derived from it.
The other copies only serve read requests.
(TODO: address write-through?)

The following data structures are commonly used as secondary indexing structures:

\medskip
\noindent
\textbf{B-Tree.}
The B-tree is the most widely used indexing structure.
Its purpose is to keep key-value pairs sorted by key, which allows efficient key lookups and range queries.
The B-tree breaks the indexed key-value pairs into fixed-size \textit{pages} (traditionally 4 KB in size);
reads and writes are performed in the granularity of a page.
Pages can be identified using an address, which allows one page to refer to another, in disk instead of in memory.
The B-tree uses these references to construct a tree of pages.
Each page contains multiple keys and references to child pages.
Each child is responsible for a continuous range of keys; keys between child page references indicate the boundaries
of those ranges.

To update the value of an existing key in a B-tree, one must search for the leaf page that contains that key,
change the value in that page, and write the page back to disk.
Adding a new key consists of finding the page whose range contains the new key, and adding it to that page.

The B-tree algorithm ensures that the tree remains balanced: a B-tree with $n$ keys always has a depth of $O(log n)$

\medskip
\noindent
\textbf{Log-Structured Merge Tree.}
Like the B-tree, the log-structured merge (LSM) tree is a key-value structure that keeps keys sorted.
An LSM-tree is composed of two or more tree-like component data structures.
A smaller component (for example a red-black or AVL tree), sometimes called a \textit{memtable},
resides entirely in memory.
The rest of LSM tree's components are persisted on disk as Sorted String Table (\textit{SSTables}).
An SSTable is a sequence of key-value pairs, sorted by keys.

Write operations are performed on the memtable.
When the memtable reaches some size threshold, the system writes it out to disk as an SSTable file.
To serve a lookup, the LSM tree algorithm first tries to find the requested key in the memtable,
then in the most recent on-disk segment, then in the next-older segment etc.
A background process periodically merges SSTables by removing redundant and deleted keys and creating compacted SSTables.

LSM-trees are typically able to sustain higher write throughput that B-trees, partly because they sequentially write
compact SSTable files to disk rather than having to potentially overwrite several pages for each write \cite{lsm:vsbtree}.

Originally the log-structured merge tree index structure was described by O'Neil et al. in \cite{oneil:lsmtree}.
The terms \textit{memtable} and \textit{SSTable} were introduced by Google's Bigtable paper \cite{chang:bigtable}.
LSM trees are used in data stores such as LevelDB \cite{leveldb:implnotes} and RocksDB \cite{rocksdb:history},
and similar storage engines are used in Cassandra and HBase \cite{hbase:hfile}.

\bigskip
\noindent
The B-tree and LSM-tree can be both used as primary or secondary index structures.

In this work, we focus on the aspects of employing secondary indexes on distributed data.
We consider these aspects orthogonal to the index implementation;
we abstract index implementation details by modeling a secondary index as a system component that provides the following
API:
\begin{itemize}

  \item An efficient range query operation $query(key_1, key_2) \rightarrow [value]$,
  where $key_1$ and $key_2$ are the boundaries of a range of keys.
  Using this API, a key lookup is a special case in which $key_1 == key_2$.

  \item Operations for inserting, updating, and deleting keys.

\end{itemize}

We argue that our results hold true for any index implementation with the above specification.
In our prototype
(TODO: ref to implementation chapter)
we use of-the-shelf state of the art index data structure implementations.

\subsubsection{Partitioning and Secondary Indexes}
\label{sec:index_partitioning_background}
The partitioning schemes discussed in \ref{sec:partitioning} rely on a key-value data model.
Secondary indexes do not neatly map to these partitioning techniques:
a secondary index usually does not uniquely identify a data item, but rather provides a way of searching for occurrences
of a particular value.

There are two main approaches to partitioning a secondary index:
document-based partitioning and term-based partitioning.

The terminology used in the rest of this section comes from the literature of full-text indexes
(a particular kind of secondary index):
a document is a self-contained piece of information, is composed of terms.

\begin{figure}[t]
  \centering

  \includegraphics[width=\textwidth]{./figures/background/index_partitioning.png}

  \caption{Index partitioning schemes (temporary; borrowed)
  TODO: redo}

  \label{fig:index_partitioning}

\end{figure}

\bigskip
\noindent
\textbf{Partitioning Indexes by Document.}
In this approach, each partition is separate:
each partition maintains its own secondary indexes, covering only documents in that partition.

In a document-partitioned index
each database write (adding, removing, or updating a document) is handled only by the partition that contains the
corresponding document.
Reading from a document-partitioned index requires a scatter/gather approach:
sending the query to all partitions and combining the returned results.
This can make index lookups quite expensive.
Even if index lookup requests are sent to partitions in parallel, response time depends from the latency of the slowest
index partition.

This approach is commonly used in commercial systems, including
MongoDB \cite{coubase:mongoindexes}, Riak \cite{riakv:secondaryindexes}, Cassandra \cite{cassandra:secondaryindexing}
Elasticsearch \cite{elastic:docrouting}, Solr \cite{solr:indexsharding}.

An index partitioned using this approach are commonly referred to as a \textit{local index}.

\bigskip
\noindent
\textbf{Partitioning Secondary Indexes by Term.}
An alternative approach is to construct a logical ``global'' index that covers data in all partitions.
A global index, however, needs to be partitioned itself, as storing it on one node would likely become a bottleneck.

To partition a global index, the indexed terms can be used as the partition key (thus the term \textit{term-partitioned}
index).
Same as in base data partitioning, the index partitioning scheme can use the terms themselves, which can be useful for
range scans, or a the terms' hashes, which results to a more even load distribution.

The advantage of a term-partitioned index is that it can make reads more efficient:
rather than requiring a scatter/gather over all partitions, a lookup for a given term only needs to make a request to the
partition containing that term.
The downside of this approach is that writes are more complicated and slower:
a write to a single document may affect multiple partitions as the corresponding terms may correspond to multiple
different partitions.

HBase \cite{hbase:secondaryindexes} uses this approach:
Secondary indexes are stored in regular HBase tables, using the indexed attribute as primary key.
Term-partitioned indexes have also been used in the research systems such as SLIK \cite{kejriwal:slik}
and Diff-Index \cite{tan:diffindex}.


An index partitioned using this approach are commonly referred to as a \textit{global index}.

\medskip
\noindent
DynamoDB \cite{dynamodb:secondaryindexes} and Apache Phoenix \cite{phoenix:secondaryidnexing} support both local and global secondary indexes.

\subsubsection{Query Planning and Execution}

Most non-relational databases have simple query models that do not support complex operations such as aggregation and
joins.
However, some document-oriented databases like MongoDB \cite{mongodb:joins}, RethinkDB \cite{rethinkdb:joins},
and CouchDB \cite{couchdb:joins} support join operations.
Query planning in NoSQL databases mainly deals with the database's distribution model:
a query execution plans consist of routing query requests to the appropriate data or index partitions.

\bibliographystyle{plainnat}
\bibliography{refs}


\chapter{The design space of geo-distributed query processing}
\label{ch:design_space}
% The goal of this chapter is to draw the design space of designing a query engine.

% Present the different design decisions,
% and the trade-offs the create (how each decision affects the metrics )

% This includes:
% \begin{itemize}
%   \item Listing the axes of the design space, and the options for each axis.
%   \item Describing the expected performance and efficiency characteristics
%   of each point in the design space.
%   \item Describe the trade-offs that occur from the above analysis.
%   \item Discuss the additional challenges/constraints that occur from the geo-
%   distribution of data.
%   \item Present state-of-the-art approaches and their limitations.
% \end{itemize}

% \section{Design choices and trade-offs}
% Describe the design choices involved in design a query engine.
% For each, list the possible options and comment on their effect on the engine's
% performance and efficiency.

% It is well understood that efficient query processing requires tuning on a case-by-case and even query-by-query basis.
% Databases systems allow database administrators to select which indexes to materialize, and choose between different
% index types.
% Relational database systems in particular have a long history of aiming to provide access to data via the use of
% optimizers, components that automatically construct query execution plans, select query operator algorithms etc. based
% on statistics on the characteristics of data.
% The common characteristics of these techniques is that query processing systems are designed as general purpose systems
% and provide mechanisms for optimizing query processing to the characteristics of different use cases \textit{at runtime}.

As discussed in chapter~\ref{ch:background}, secondary indexing, the use of materialized views, and caching are techniques
crucial to providing efficient query processing.
They all involve the use of \textbf{derived state}, specialized representations of the base data that speed up a
particular read access to data.
The use of derived state for efficient query processing involves a number of design decisions.
These decisions are often in tension and create trade-offs.

In this chapter, we describe the design decisions and trade-offs involved in the use of derived state for query processing,
and discuss how difference choices affect the behavior of the query processor.
In addition, we present a framework for reasoning about these design choices and their interactions.

\section{The use of derived state in query processing}

Indexes, materialized views, and caches are all instances of a common technique:
creating derived state by applying a transformation to the data stored in the database in order to speed up a particular
read access to the data.

At an abstract level, derived state can be described by a \textbf{write path} and a \textbf{read path}
\cite{kleppmann:designing}.
The write path is the process of updating the derived state to reflect a change to the base data.
The read path is the process serving a query using the derived state.
% Note that in the case of caches, the read path modifies the derived data (inserting new cache entries after a cache miss).
In other words, the write path is the pre-computation that takes place as soon as a change to the base data occurs,
regardless if the results are going to be consumed by a query;
the read path is the computation that occurs when one reads uses the derived state to process query.

\medskip

We can illustrate how the notion of the read and write path applies on different derived state structures using an example.
Consider a database that stores images.
Each image can be associated with user-defined metadata tags, and the database provides the functionality of
images via queries on those tags.
Consider the query \\

\noindent
$SELECT$ $*$ $FROM$ $photoAlbum$ \\
$WHERE$ $tags.predominantColor$ $BETWEEN$ $\#0a6fb6$ $AND$ $\#52aca2$ \\

\noindent
(this query could for example be part of a service that automatically creates slideshows from the image dataset).

A secondary index on the $predominantColor$ tag can be used to speed up this type of queries.
In that case, the write path updates the index (when images are inserted or deleted), and the read path searches the
index for color values in the range specified by the query, and combines the results.
If an index has not been created, processing the query would involve scanning all the images in $photo-album$:
no work is required in the write path, and significantly more work is needed in the read path.
Using an index, therefore, shifts an amount of computation from the read to the write path.

Another option is to maintain pre-computed search results for the most common queries, so that they can be served
without needing to access the index or scan the dataset.
This can be implemented either as a \textit{cache} of common queries, or as a \textit{materialized view}.
In the case of a cache, the read path either reads from the cache, or falls back to using the index or performing a scan
and then writes to the cache; the write path may invalidate cache entries, depending on the policy that is used.
In the case of a materialized view, the read path read the query results from the view, while the write path updates the
view to reflect updates to the dataset that should be included in the results of the most common queries.

From this example it is clear that indexes, materialized views, and caches can be viewed as techniques for shifting the
boundaries between the read and the write path.
They allow the system to perform more work on the write path, in order to reduce the work needed or speed up the
processing on the read path.
Database systems provide mechanisms for managing this trade-off at runtime, by allowing the database administrator
to select indexes and materialized views to be created, and by providing mechanisms to automate this selection
\cite{valentin:db2advisor, chaudhuri:decadeselftuning}.

\section{Design decisions and trade-offs in derive state based query processing systems}

Designing a query processor that maintains derived state involves a number of design decisions:
\begin{itemize}

  \item Given a change in the base data, when is the derive state updated to reflect this change?

  \item How is the derived state distributed across the system's nodes?

  \item In a system composed of multiple geographically distributed sites, how is the derived state placed across sites?

\end{itemize}

The approaches taken in these questions affect different aspects the query processors characteristics, including query
performance, overhead to write operation, relevance of query results, and operation cost.

These design decisions and their effects to the system's characteristics can be analyzed through the lens of the
read and write path computations to derived data:
\begin{itemize}
  % \item Selecting which indexes and materialize views to create and maintain shifts the \textbf{balance between
  % read-path and write-path computation} (section~ \ref{sec:index_view_selection}).
  % Materializing more views and indexes reduces the amount of work to be done on the read-path, speeding up query
  % processing; it also, however, increases the amount of work on the write-path.
  % This results in a trade-off between the amount of read-path and write-path computation

  \item Selecting between maintaining derived data synchronously or asynchronously, changes the
  \textbf{impact of write-path computation} (section~ \ref{sec:sync_async_maintenance})
  Maintaining derived data synchronously incurs an overhead to write operations, as derived data need to be updated in
  the critical path of writes.
  On the other hand, asynchronous maintenance does not cause an overhead to write operations, but it means that derived
  data are not strongly consistent with the base data.
  Therefore the design decisions on derived state maintenance involve a trade-off between the overhead to write operations
  and the consistency between base and derived data.

  \item In distributed databases, derived data structures are often partitioned,
  and therefore read path and write path computations are distributed computations.
  The partitioning scheme determines the \textbf{communication patterns} of read and write path computations.
  In section~ \ref{sec:index_partitioning} we describe how the choice of index partitioning scheme affects query
  performance and the overhead to write operations (assuming synchronous maintenance).
  The two main index partitioning approaches can be viewed as a trade-off in the volume of communication (number
  of round trips, and number of entities to be contacted) between the read and the write path.

  \item In a database system that is distributed across multiple geographically distributed sites, the placement of
  derived data across sites affects the \textbf{communication latency} of the read-path and write-path computations.
  (section~ \ref{sec:placement})
  Since both clients and base data are distributed across sites, placement decisions involve a trade-off between
  requiring cross-site communication on the read path or the write path.

\end{itemize}

TODO: here we need a visualization for this analysis. a table? a decision tree? both?

It is important to note that the framework of read and write path computations does not include the memory and storage
costs associated with maintaining derived state.

% \subsection{Index and materialized view selection}
% \label{sec:index_view_selection}
% As discussed in chapter~\ref{ch:background} indexes are crucial to query processing performance in most database system.
% The use of indexes provides fast access to data,
% but also complicates updates operations since indexes need to be updated to reflect changes to the indexed data.
% Hence, there is a tradeoff involved in selecting which indices to materialize.
% Having too few or not having the appropriate indexes may force many queries to scan large parts of the dataset;
% Having too many indexes incurs high update costs.

% A similar tradeoff lies in the problem of selecting materialized views.
% Using an appropriate set of materialized views can significantly reduce query response time as processing a query by
% accessing a materialized view can be much faster than processing the query from the base data.
% On the other hand, materializing a view incurs an additional maintenance and storage space cost.

% Therefore, both the index and view selection problem involve a tradeoff between the reduced query response time
% from the one size,
% and write latency and memory/storage space overhead from the other.

% The index and view selection problems have been studied extensively.
% In particular, various commercial and research systems have focused on the problem of automated index selection
% \cite{valentin:db2advisor, chaudhuri:decadeselftuning}

% TODO: methods for finding a rewriting of a query using a set of materialized views -> orthogonal

\subsection{Derive state maintenance policies}
\label{sec:sync_async_maintenance}

Given a database write that updates a data item $d$ by sets the value of an attribute $attr$ $v_1$,
updating derived state (and index or materialized view) that is affected by this update consists of the following steps:
(1) inserting $d$ to the state entries or materialized query results that correspond to $attr$ $=$ $v_1$,
(2) retrieve the value of $attr$ in $d$ before this update, say $v_0$,
(3) and remove $d$ from the state entries or materialized query results that correspond to $attr$ $=$ $v_0$,

The derived state maintenance policy scheme defines when these steps are performed.
In \textbf{synchronous} maintenance, all steps are performed in the critical step of a write operation, usually bundled
as a transaction.
This incurs overhead in write latency, but ensures that derived state is always up-to-date with the base data.

An alternative approach is to perform some of the maintenance step asynchronously: deferring them for after the write
operation has been acknowledged to the client.
This reduces the overhead to write operations, but has the implication that the derived state may be temporarily
stale, representing a previous state of the base data.
If stale derive state is used for query processing, then query results may be inconsistent with the state of the base
dataset, having false-positives and false-negatives.
Therefore, asynchronous state maintenance reduces the query processor's \textit{effectiveness}.

\bigskip

We can view derived state maintenance schemes through the lens of read and write path computation as follows.
The write path computation can be broken down into synchronous and asynchronous computation;
synchronous write path computation incurs an overhead on write operations, while asynchronous write path computation
relaxes the consistency of the derived state.
The state maintenance scheme determines the boundary between synchronous and asynchronous write path computation.
Because reading from inconsistent derived state is possible to results that are not up-to-date with the base data,
maintenance scheme design decisions involve a trade-off between \textit{write overhead} and query result
\textit{effectiveness}.

\bigskip

\noindent
One asynchronous maintenance scheme consists of inserting new derived state entries synchronously (step 2),
and removing the old state entries asynchronously in the background (steps 2 and 3).
The literature often refers to this scheme as \textit{sync-insert}.

The implications of sync-insert is that it reduces the work needed to be done at the critical path of write operations,
but it temporarily leaves stale entries in the derived state, until steps 2 and 3 are performed.
Queries that read from read the stale entries will include false positives in their results.
A common complementary mechanism used with sync-insert is read-repair:
The system validates query results by reading from the base data, and removing false-positive.
Therefore, the read-repair mechanism can be seen as a way to shift an amount of work from the write to the read path.

Another asynchronous policy, termed \textit{async-simple} consists of acknowledging the write operation to the client as
soon as the base data is updated, and performing all derived state maintenance steps in the asynchronously.
In practice, async-simple is implemented using an asynchronous update queue: write operation are acknowledged as soon
as they are logged in the queue; a background process ingests the queue and performs derived state maintenance.

This scheme incurs no overhead to write operations.
However, this approach only provides eventually consistency;
for each write operation there is a time window in which a data item has been updated, but this update has not been
reflected in the derived state.
Reading from the state in this time window, it is possible that a data item that has already been updated in the database
appears with an old value in the query results (neither step 1 nor 3 have been applied), appears to have been remove
from the database (step 3 has been applied, but not step 1), or appears to be associated with two values (step 1 has
been applied but not step 3).

This scheme is used by Amazon's DynamoDB
TODO: cite
% https://docs.aws.amazon.com/amazondynamodb/latest/developerguide/GSI.html
.
Secondary indexes in DynamoDB are update in an eventually consistent fashion.
Because of this applications need to
``anticipate and handle situations where a query on a secondary index returns results that are not up to date''

Diff-Index \cite{tan:diffindex} has proposed an asynchronous index maintenance scheme that provides sessions guarantees.
The technique used to achieve this is to track additional state in the client library:
this state is used to guarantee that any index look-up contains updates to the base data that were made in the same
session.
This guarantees the read-your-writes property.

% So far we have assumed that index updates are synchronous:
% index entries are updated inside the critical path of write operations.
% However, as we discussed in the previous section, in the case of a term-partitioned index,
% index updates are distributed operations.
% This may result in significant overhead to write operations.
% As a result, many databases do not update their indexes synchronously, but rather use asynchronous maintenance schemes.

% -> effect on efficiency : false-positives



\subsection{Index partitioning}
\label{sec:index_partitioning}
Maintaining indexes and materialized views in a distributed database is complex because the partitioning schemes of
derived state do not map to the partitioning of base data, and as a result read path and write path computations may
involve communication among multiple nodes.

The are two main approaches for partitioning an index over a partitioned dataset, presented in chapter~\ref{ch:background},
are:
\begin{itemize}

  \item In \textbf{partitioning by document}, index entries are co-located in the same node as the corresponding
  data items.

  \item In \textbf{partitioning by term}, a global index is partitioned using the same partitioning scheme as the base
  tables, and using the indexes value as the partitioning key.

\end{itemize}

In the partitioning-by-document approach, an index lookup requires broadcasting a read request to every index partition
and then gathering the returned results.
On the other hand, when reading from a term-partitioned index only the index partitions with index entries that are
relevant for the query need to be contacted.
In this approach, however, an additional round of messages is required for retrieving the base data items, as they may
be located on different nodes that the corresponding index entries.
This is not the case for the document-partitioned index, since data items are by-design co-located with the corresponding
index entries.

An advantage of the document-partitioned index is that updating the index upon a write to the database does not require
communication between nodes.
On the other hand, updating a term-partitioned index may involve significant communication overhead, as an update to data
item may involve updating index entries located on different nodes.

\bigskip

These observations can be grouped under the framework of read and write path computations:
partitioning-by-document guarantees local-only communication on the write path, but the most amount of communication on
the write path (a scatter/gather operation involving all index partitions);
partitioning-by-term involves communication across nodes on the write path, but requires less communication on the read
path in the general case (index partitions known do not include relevant index entries are not contacted).

TODO: figures

\bigskip

Guided by this analysis we can reason about the performance characteristics of the two approaches.

Partitioning-by-document is more suitable at a small scale (with a small number of nodes),
while partitioning-by-term becomes preferable as the number of nodes increases and has been shown to provide better
scalability \cite{kejriwal:slik}.

In addition to the system's characteristics, which approach is more suitable for a certain case depends on various
workload characteristics.
In \cite{dsilva:tworings}, D`silva et al. perform an extensive experimental comparison of the two approaches,
implemented in HBase.
Their results show how the performance characteristics of the two approaches is affected by various characteristics of
the workload.

The factors that affect index lookup performance are:
\begin{itemize}

  \item The distribution of values of the indexes attribute:
  As the number of data items per index value increase, the partitioning-by-document approach performs better.
  This is because of the additional round trip required to retrieve the base data items:
  as the number of data items per index value increases, a term-partitioned index is likely to have to contact
  more and more remote nodes, while in a document-partitioned index data items are always co-located on the same node as
  the index entries.

  \item Concurrency:
  As the number of concurrent index lookups increases, the partitioning-by-term approach performs better.
  This can be attributed to the overhead of the scatter/gather operation used by the document-partitioned index:
  As the volume of concurrent index lookups increases, the overhead of broadcasting to every node become more significant.

  \item The selectivity of queries:
  The document-partitioned index outperforms the term-partitioned index as the range of range queries increases.
  This is in accordance with the observation that the partitioning-by-document approach is beneficial where there are
  more records to be returned.

\end{itemize}

In addition, this work assumes synchronous index maintenance, and evaluates the impact of the two approach to write
operations.
The document-partitioned index generally outperforms the term-partitioned for writes, incurring less overhead
and providing better scalability.

From the above analysis it is evident that neither of the two approaches is suitable for all needs.
The choice of which approach to use should be guided by factors including the scale of the system, the properties of the
dataset (distribution of indexed values over the data items), and the characteristics of the workload (query/write ratio,
concurrency, query selectivity):
\begin{itemize}

  \item Document-partitioned indexes are more suitable for: (1) smaller scale systems with a small number of nodes and
  limit concurrency in index lookups, (2) workloads with less selective queries that return large result sets, (3)
  skewed data distribution where a large number of data items have the same indexed value, or (4) write-intensive
  workloads.

  \item Term-partitioned indexes are more suitable for: (1) larger scale systems with a greater number of nodes, (2)
  query-intensive workloads with a large query load, (3) workloads consisting of more selective queries with smaller
  result sets, or (4) data with normal distribution in the indexed value.

\end{itemize}

Clearly, the decision of which index partitioning approach to be used needs be taken in a case-by-case basis.
Database systems should to support both index partitioning schemes, and expose the partitioning scheme selection as a
configuration parameter at the time of creating an index.
We are not aware of any database system that provides this functionality.
In existing systems, this design decision is made at design time, and every index uses the same partitioning scheme.
\cite{}
  TODO: cite.
In Chapter~\ref{ch:case_studies} we demonstrate how a distributed database can provide both index partitioning
schemes.



\subsection{Placement}
\label{sec:placement}



% -> This can apply to any query processing operator, not only stateful ones.

% \subsection{Query processing techniques}

% The choice between
% (1) performing query processing by iterating over the corpus (filtering, aggregation without materialized views)
% (2) maintaining derived state (indexes, materialized views) updated in response to changes to the corpus data
% (3) using caches on top of that.

% \subsubsection{Secondary indexing}

% General system design pattern:
% Having the same data represented in different formats to address different access patterns.

% We have a primary copy of the data in one system (that might be called the source of truth)
% and different derived data systems which they take their data as a copy from this
% primary system.
% Transform it in some way and then represent it differently
% in order to satisfy certain read access patterns.
% Writes go to the primary storage system and all of the other systems are derived from it.
% TODO: address that some systems support writes to a view.
% The other systems only serve read requests.

% \begin{itemize}
%   \item
%   \item it’s just another way of saying we’re going to take the same data and have multiple
%   copies of it represented in different ways, either sorted in a different way, or with a different
%   storage layout.  That allows us to access the data in ways depending on what you’re trying to do. 
%   \item In any relational database, if you have a secondary what you’re really saying is,
%   I want to construct this additional structure on the side (typically a B-tree) and every time you write to some table to update
%   that index to also allow me to find records based on values of a column other than the primary key.
%   (It just happens to be done internally by the database.)
% \end{itemize}

% \subsubsection{Distributed Query planning}

% \subsubsection{Caching}

% \subsection{State maintenance}
% When maintaining derived state, the choice between synchronously versus
% asynchronously updating it for each change to the corpus.

% State maintenance schemes have two characteristics:
% \begin{itemize}
%   \item How much of the state is materialized.
%   When maintaining an index we assume that every index entry is materialized.
%   In contrast, when maintaining a cache, only a subset of entries are
%   materialized
%   (a subtlety here is how an ``entry'' is defined, it might be different between
%   index and cache).
%   There is a finite storage area for materializing entries.
%   When it is filled, some entries are evicted to make space for new entries to
%   be materialized.
%   \item When is the state being updated.
%   The options for this can be:
%   \begin{itemize}
%     \item Update-triggered - Synchronously. When an update to the corpus
%     occurs, it is blocked until the derived state is up-to-date.
%     \item Update-triggered - Asynchronously. Updating the derived state is done
%     outside of the critical path of the corpus update.
%     \item Read-triggered. The derived state mays updated as a response to the
%     state being read (cache misses).
%     This is used in caching.
%   \end{itemize}
% \end{itemize}
%   Based on the above analysis, we can argue that caching is just another state
%   maintenance scheme, and therefore could be treated as a configuration
%   parameter when deploying a QPU.
%   (Note: this is just a way to unify and simplify things by merging caching
%   QPUs with index QPUs, materialized view QPUs etc.)

% \subsection{Component placement}
% There are two types of computational query processing operator can be
% involved in:
% \begin{itemize}
%   \item Update-triggered: For example, updating an index as a result to a corpus
%   update
%   \item Query-triggered: For example, index/cache lookup as a result to a query.
% \end{itemize}
% This can apply to any query processing operator, not only stateful ones.

% The placement determines how close (in terms of network round trip time) the
% operator is placed related to its sources of updates and queries.

% Deciding on placement schemes can lead to trade-offs cause by:
% \begin{itemize}
%   \item The benefits and costs of placing operators close to update sources
%   versus close to query sources.
%   \item An operator might have multiple update and/or query sources.
% \end{itemize}

% We can define ``levels'' of placement: on the same physical machine, on the same
% cluster, in the same DC.

% \section{Trade-offs}
% Summarize how the previously described design choices result to performance and
% efficiency trade-offs.

% Note: This may be redundant depending on how detailed those trade-offs are
% discussed in the previous section, but it is central to the thesis to make the
% argument that it is because of these trade-offs that we propose the QPU
% approach.

% An idea is to here focus on examples of applications that might choose different
% points in these trade-offs.

% \subsection{Response time vs Freshness}

% \subsection{Response time \& Freshness vs Cost}


% \section{Constraints of geo-distributed data}
% Discuss the implication of performing query processing on geo-distributed data,
% and of replicating the derived data used for query processing.

% \subsection{Implications of replicated data in derived state}
% Implication of query processing over AP storage systems:
% \begin{itemize}
%   \item If not carefully designed, concurrent non-conflicting updates may
%   result to conflicts on the derived state.
%   Example: write1: recordX:attributeA=1, write2: recordY:attributeA=1.
%   This results to: index\_update1: add recordX to (attributeA, 1),
%   index\_update2: add recordY to (attributeA, 1).
%   Therefore a ``set CRDT'' behavior is needed for indexes.
%   Generalize to state other than indexes.
%   \item Conflicting updates to the corpus data.
%   Ensure that conflict resolution on corpus data is reflected on derived data.
% \end{itemize}

% \subsection{CAP}
% Describe how query processing performance-availability is constraint by
% CAP in the case of replicated derived state.

\bibliographystyle{plainnat}
\bibliography{refs}

\chapter{A design pattern for flexible query processing architectures}
\label{ch:design_pattern}

In this chapter we present our main contribution,
a \textit{design pattern} for building query processing systems with the goal of evolving query processing
system's architecture from monolithic and static to modular and flexible.

Traditional static query processing systems are not able to cater to the needs of modern applications in which users and
data are geo-distributed across the globe.
Our vision is to enable query processing systems that are designed and deployed on a case-by-case basis,
with the workload characteristics, data and access distribution patterns, and requirements of specific applications in
mind.

As a first step towards this vision, we focus on the \textit{mechanisms} required for enabling a flexible and configurable
query processing system architecture.

The design pattern that we propose is based on the following objectives:
\begin{itemize}
  \item \textbf{Decoupling between storage and query processing architecture.}
  The design pattern should be based on architecture, mechanism or interface of a specific data storage tier.
  Instead, it should enable query processing system to work as middleware systems on top of existing database systems.

  \item \textbf{Independence from corpus partitioning and distribution schemes.}
  The design pattern should not be based on specific a partitioning and distribution scheme for the corpus,
  but rather enable the construction of query processing systems compatible with multiple different
  data partitioning or distribution schemes.

  \item \textbf{Tunability.}
  The query processing should be able to be configured on the following dimensions:
  \begin{itemize}
    \item Index and view materialization:
    The query processing system should enable database to create secondary indexes and materialized views.
    Additionally, the index data structure used for indexes should be configurable.
    \item Caching:
    Similarly, the use of caches should be configuration-based.
    \item State maintenance scheme:
    The mode (synchronous or asynchronous) of incrementally applying corpus update to indexes and materialized views,
    should be configurable in a per index/view basis.
    \item State partitioning scheme:
    Similarly, the partitioning scheme used for secondary indexes and materialized views should be configurable in a per
    index/view basis.
  \end{itemize}

  \item \textbf{Flexible state and computation placement.}
  The design pattern should enable fine-grained control over the placement of the query processing system's derived state
  and computations.

\end{itemize}

\section{Overview: composable query processing system architecture}

The key idea for achieving the objectives described above is \textit{assembly-based modularity}
\cite{leclercq:dream, bouget:pleiades}.
A query processing system is constructed by interconnecting composable building blocks
that encapsulate derived state structures, such as indexes, materialized views, and caches,
as well as relational operators such as filters, aggregations, and joins.
In that way, configuration choices, such as index and view materialization, cache, and derived state partitioning,
translate to \textit{composition choices}: which building blocks are used and how they are interconnected.
For example, adding a caching layer to an existing query processing system can be done by extending an existing
architecture with additional ``caching'' building blocks.

Moreover, the query processing architecture's building blocks separate interface from implementation.
Because the query processing system's components communicate through well-defined interfaces,
they can be flexibly placed across the system infrastructure.
As a result, a given query processing system architecture can support multiple different component placement schemes.

\bigskip
\noindent
This design pattern defines a \textbf{modular} and \textbf{composable} component-based query processing system
architecture.
The principal element of this architecture, is its building block, which we term Query Processing Unit (QPU).

A QPU is a \textit{system component} that combines properties of a streaming operator and a microservice.
Similarly to a streaming operator, a QPU receives one or more input streams, performs a computation over these streams,
and emits an output stream.
Similarly to a microservice, a QPU provides its functionality by exposing an interface for receiving requests
(called \textit{query requests})
and can interoperate with other QPUs by sending query requests to them.
The response to query request is a stream:
every input stream that one QPU $A$ receives is the output stream of another QPU $B$, and is the response to a query request
that $A$ has sent to $B$.

\medskip
\noindent
A query processing unit can encapsulate a \textit{relational operator}.
For example, a ``join operator'' QPU receives input streams that represent the tables to be joined,
and emits an output stream that represents the results of the join operation on these tables.

Moreover, a QPU can encapsulate a \textit{derived state structure}, such as an index, a materialized view, or a cache.
For example, a ``secondary index'' QPU receives an input stream that represents notifications for updates to that table
at the corpus.
It maintains a secondary index data structure, which it stores as internal state.
When it receives a query request, it reads from its internal state and emits the result as an output stream.

Finally, a QPU can encapsulate a responsible for managing access to components of the query processing system, such as
a partition or replica manager, or a load balancer.
For example ``partition manager'' QPU, responsible for managing access to the partitions of a partitioned secondary index
(each index partition being encapsulated by a QPU), sends query requests to the appropriate index partitions for a given
query, and merges the resulting streams.

Our key insight is that all three of the described QPU types --- relational operators, derived state, and routing operators ---
can be generalized to a system component with common semantics, and thus can be composed in order to implement higher-level
query processing computations.

\medskip
\noindent
QPUs are organized in a directed acyclic graph.
The edges of the graph represent potential query request - response steam relations.
For example, an edge from a QPU $A$ to $B$ indicates that $A$ can send query requests to
--- and therefore establish input streams from --- $B$.
We call $B$ a \textit{downstream connection} of $A$, and a query request sent from $A$ to $B$ a \textit{downstream query}.
Base data are the leaves of the graph (nodes with only incoming edges),
and client queries enter the graph through its root nodes (nodes with outgoing edges).

When a QPU receives a query, it can either process it by reading from its internal state, or establish input streams
by sending query requests to its some of its downstream connections and produce query results by performing a
computation over these streams,
or a combination of the two.
As each QPU send query requests to its downstream connections,
this process is repeated at each QPU that receives a query request.
In that way, a client query results to query requests flowing downwards through the query processing system's QPU graph,
defining a \textit{query execution sub-graph}.
Query result stream flow upwards through the edge of this sub-graph.


\section{Query processing unit: a building block for composable query processing architectures}

\subsection{The Query Processing Unit component model}

The key requirement for the query processing unit is \textit{composability}:
QPUs should be able to be interconnected in various topologies,
and to interoperate in the execution of query processing tasks.

To achieve this, we define a common set of properties that every query processing unit should conform to.
This properties include the QPU's interface for receiving query requests, and the query request - response stream semantics.
We call this set of properties the query processing unit \textit{component model} (QPU model for short).
Using object-oriented programming terminology, the QPU model can be viewed as an \textit{abstract class}
that defines a set of method signatures, but not their implementation.
Implementations of the QPU model define \textit{QPU classes} with specific functionalities, for example
filter, join or materialized view QPU classes.
Using the object-oriented programming analogy, QPU classes can be viewed as classes that implement the QPU
abstract class.
Finally, specific \textit{QPU instances} can be viewed as objects of a specific QPU class, for example a secondary
index QPU for the attribute $predominantColor$ of the table $photoAlbum$.
According to the proposed design pattern, query processing systems are composed of QPU instances.

In the rest of this thesis we use the terms query processing unit, QPU, and unit interchangeably to refer to QPU instances.

\bigskip
\noindent
In this section we present an overview of the QPU model, with the goal of introducing the main concepts.
In section~\ref{ref:specification} we present the detailed specification of the QPU model.

The query processing unit component model defines a \textit{system component} with the following properties:

\begin{figure}[t]
  \centering
    \includegraphics[width=0.5\textwidth]{./figures/design_pattern/qpu_abstraction.pdf}
  \caption{A conceptual depiction the QPU model.}
  \label{fig:qpu_abstraction}
\end{figure}

\medskip
\noindent
\textbf{Query interface.}
Query processing units expose an interface for receiving query requests.
This interface is common among all QPU classes.
As a result, a QPU can send query requests to other QPUs, regardless of their class.
This is the mechanism with which QPUs interoperate to perform query processing tasks,
and forms the basis of the query processing system's computation model (section~\ref{sec:computation_model}).

However different QPUs may be able to process only a subset of the queries that can be expressed by the interface's query
language.
For example, a QPU may only serve queries about a specific database table.
We call the set of queries that a unit can process its \textit{query processing capabilities}.

The QPU's query interface has \textit{streaming semantics}.
An invocation of the query interface initiates a stream between the QPU that sent the query request and the one that
received it;
the results of the query are sent as records through that stream.

We present the query interface in more detail in section \ref{ref:query_interface}.

\medskip
\noindent
\textbf{State.}
Each QPU maintains \textit{internal} state, that is accessible only by that QPU.

We distinguish the QPU state in three parts according to its functionality.
Each query processing unit maintains \textit{configuration state} that represents the query processing unit's
configuration parameters.
Moreover, QPUs that have downstream connections maintain information about these connections, which is
used for generating downstream query requests.
We call this part of the state \textit{local graph view}.
Finally, query processing units that implement derived state structures and QPUs that store intermediate query processing
results, for example in streaming join computations, maintain \textit{query processing state}.

\medskip
\noindent
\textbf{Initialization, query processing, and input stream callback function.}
The functionality of a query processing unit can be modeled using three functions:
\begin{itemize}
  \item The \textit{initialization function} (section~\ref{sec:initialization_func}), which is executed when the QPU is
  initialized.

  \item The \textit{query processing function} (section~\ref{sec:query_processing_func}), which is executed for each
  query request received by the QPU, and is responsible for processing the query and sending results to the response stream.

  \item The \textit{input stream callback function} (section~\ref{sec:callback_func}), which is executed for each record
  received through an input stream.

\end{itemize}

The QPU model defines the signatures of these functions, but a specific implementation.
Each QPU class provides implementations for these functions.

\bigskip

A conceptual depiction of the query processing unit model is shown in Figure~\ref{fig:qpu_abstraction}.
When the QPU's query API is called, an output stream ($R_A$) is established between the unit and the sender,
and the unit's query processing function is invoked for the given query.
The query processing function can read the QPU's state, and can perform downstream queries to other units.
For each downstream query, a corresponding stream is established ($Q_{A.1}$ and $Q_{A.2}$).
When a record is received from one of the streams, the QPU's callback function is invoked.
Each invocation of the callback function processes a received record, and returns the result to the query processing
function.
Upon receiving a result from a callback function,
the query processing function can potentially write to the QPU's state and/or emit a record to the output stream.


\subsection{Query Processing Unit component model specification}
\label{ref:specification}

In this section we present the detailed specification of the query processing unit component model.

\subsubsection{Query interface}
\label{ref:query_interface}
% we first present a high level overview of the QPU model's interface, and then describe each of its elements in more detail.

Query processing units expose an interface for receiving query requests:

\begin{displaymath}
  Query(QueryRequest) \rightarrow QueryResponse
\end{displaymath}

\begin{itemize}
  \item $QueryRequest$ specifies a predicate on the data items' \textit{attributes}, and a \textit{time interval}.

  \item $QueryResponse$ is a stream containing the query's results.
\end{itemize}

\noindent
\textbf{Query results as updates}.
$QueryResponse$ records represent \textbf{updates} to the corpus (an update can be a creation, modification, or deletion
of a data item).
Updates are represented as \textit{deltas}:
a delta representing an update $u$ to a data item $d$ contains the values of $d$'s attributes before $u$ is applied,
and those after $u$ is applied.

More specifically:
\[
  QueryResponse = [StreamRecord]
\]
where
\[
  StreamRecord =
\]
\[
  (DataItemID,~[(AttributeName,~AttributeValue_{old},~AttributeValue_{new})],~Timestamp)
\]

\begin{sloppypar}
An $update$ is a triplet containing
(1) the primary key of the data item ($DataItemID$) it refers to,
(2) a list of triplets of the form $(AttributeName, AttributeValue_{old}, AttributeValue_{new})$ that represent the
data item's attribute values before and after the update is applied,
and (3) the timestamp ($Timestamp$) assigned to the update.
\end{sloppypar}

\medskip
\noindent
\textbf{The query request's time interval.}
Given a $QueryRequest$ that specifies an attribute predicate $Pred$ and a time interval $T$ $=$ $[t_1, t_2)$
\begin{itemize}

  \item if $t_1$ $<$ $t_2$,
  $QueryResponse$ contains any update $u$ with $t_1$ $\leq$ $timestamp$ $<$ $t_2$,
  and for which one of the following is true:
  \begin{itemize}
    \item $u$ creates a data item $d$, and $d$'s attributes satisfy $Pred$.
    \item $u$ deletes a data item $d$, and $d$'s attributes satisfied $Pred$ before deletion.
    \item $u$ modifies a data item $d$, and $d$'s new attribute values after $u$ is applied satisfy $Pred$.
    \item $u$ modifies a data item $d$, and $d$'s old attribute values before $u$ is applied satisfy $Pred$.
  \end{itemize}
  We term this type of query \textit{an interval query}.

  \item If $t_1$ $=$ $t_2$ = $t$,
  then among the set of all updates that the condition stated above,
  $QueryResponse$ contains for each data item \textit{the update with the latest timestamp before $t$}, i.e:

  $\nexists$ $u':$ $timestamp'$ $>$ $timestamp$ $\land$ $timestamp'$ $\leq$ $t$

  We term this type of query a \textit{snapshot query}.

\end{itemize}

By returning the latest update before a given timestamp $t$,
a snapshot query effectively returns the \textit{state} of each data item that satisfies $Pred$ at $t$.
Therefore, a snapshot query refers to a \textit{snapshot} of the corpus that contains the effects of all update with
$timestamp$ $<$ $t$.

In contrast, an interval query returns may return multiple updates for a specific data item,
which have timestamps within the specified time interval and modify the data item so that either is starts of it stops satisfying $Pred$.

By using a upper bound ($t_2$) in the future (a timestamp that the system has not yet reached),
an interval query can continue receiving future updates that satisfy $Pred$.
This provides a mechanism for \textit{subscribing to notification for updates}.

\bigskip

\textbf{Query Language.}
% https://documentation.basis.com/BASISHelp/WebHelp/usr2/sql_grammar.htm
% http://www.h2database.com/html/grammar.html#expression

The query processing unit interface support a query language with an SQL-like syntax.
The query language supports point and range queries, logical operators, aggregation functions, and joins.
We argue that these is no inherent limitation in the QPU model that prevents it from supporting a more complex
query language (for example supporting nested queries).
However, we consider additional functionalities to be out of the scope of this work.
We believe this query language can effectively demonstrate that the QPU model can be used as building
block for constructing fully-functional query processing systems.

\medskip
\noindent
The QPU model's query language has following syntax:

{\obeylines\obeyspaces
\texttt{
QueryRequest          ::=  SELECT SelectExpression
~~~~~~~~~~~~~~~~~~~~~~~~~~~FROM TableExpression
~~~~~~~~~~~~~~~~~~~~~~~~~~~WHERE PredicateExpression
~~~~~~~~~~~~~~~~~~~~~~~~~~~TIMESTAMP TimestampExpression \todo{the timestamp can be just another attribute, but I think we need to have it separate for emphasis}
~~~~~~~~~~~~~~~~~~~~~~~~~~~[ GROUP BY attributeName ]
~~~~~~~~~~~~~~~~~~~~~~~~~~~[ ORDER BY OrderByExpression \{ ASC | DESC \} ]
}}

{\obeylines\obeyspaces
\texttt{
SelectExpression      ::=  ALL
~~~~~~~~~~~~~~~~~~~~~~~~~~~SelectExpressionItem , SelectExpression |
~~~~~~~~~~~~~~~~~~~~~~~~~~~SelectExpressionItem
}}

{\obeylines\obeyspaces
\texttt{
SelectExpressionItem  ::=  SUM(attributeName) | AVG(attributeName) |
~~~~~~~~~~~~~~~~~~~~~~~~~~~MAX(attributeName) | MIN(attributeName) |
~~~~~~~~~~~~~~~~~~~~~~~~~~~attributeName
}}

{\obeylines\obeyspaces
\texttt{
TableExpression       ::=  tableName JoinType tableName On tableName.attributeName = tableName.attributeName |
~~~~~~~~~~~~~~~~~~~~~~~~~~~tableName
}}

{\obeylines\obeyspaces
\texttt{
JoinType              ::=  \{ INNER | \{ LEFT | RIGHT \} OUTER \} JOIN
}}

{\obeylines\obeyspaces
\texttt{
PredicateExpression   ::=  PredicateExpression OR PredicateExpression |
~~~~~~~~~~~~~~~~~~~~~~~~~~~PredicateExpression AND PredicateExpression |
~~~~~~~~~~~~~~~~~~~~~~~~~~~NOT PredicateExpression |
~~~~~~~~~~~~~~~~~~~~~~~~~~~Term Op Term
}}

{\obeylines\obeyspaces
\texttt{
Term                  ::=  attributeName | Value
}}

{\obeylines\obeyspaces
\texttt{
Op                    ::= > | >= | < | <= | = | !=
}}

{\obeylines\obeyspaces
\texttt{
Value                 ::= stringValue | floatValue | intValue |
~~~~~~~~~~~~~~~~~~~~~~~~~~dateTimeValue | timestampValue
}}

{\obeylines\obeyspaces
\texttt{
OrderByExpression     ::= attributeName , OrderByExpression |
~~~~~~~~~~~~~~~~~~~~~~~~~~attributeName
}}

{\obeylines\obeyspaces
\texttt{
TimestampExpression   ::= FROM TimestampTerm [ TO TimestampTerm ]
}}

{\obeylines\obeyspaces
\texttt{
TimestampTerm         ::= SYSTEM START | LATEST | timestampValue
}}

~ \bigskip

In practice, each QPU instance can only process a subset of the queries that can be expressed by this query language,
according to the functionality supported by its class.
For example, a QPU class that implements a join operator can perform join over two input streams,
but not evaluate a $PredicateExpression$, or perform an $SUM$ on an attribute of the join result.
This can be achieved by connecting ``join'', ``filter'', and ``aggregator'' QPUs.
We present in detail these QPU classes in section~\ref{sec:qpu_classes},
and how they are composed to provide complex query processing functionalities in section~\ref{sec:query_processing_system}.

Moreover, instances of a certain class may support different parts of the query language according to their
configuration.
For example, two instances of a filter operator QPU class may support predicate queries for different tables.

\subsubsection{QPU State}

The query processing unit's state is divided into the following parts:

\medskip
\noindent
\textbf{Configuration state.}
Upon initialization, each QPU receives is given a set of configuration parameters.
These parameters are represented as key-value pairs.
The configuration state represents these parameters.

Configuration parameters can be divided to the following parts:
\begin{itemize}
  \item Topology configuration.
  This part of the configuration consists of the endpoints of the QPU's downstream connections in the QPU graph.

  \item Class-specific configuration.
  The part of the configuration specifies parameters such as cache size or the definition of a materialized view.
\end{itemize}

% \begin{lstlisting}[caption={Pseudocode for the QPU's configuration state},captionpos=b,label={lst:qpuconfigstate}]
% class ConfigurationState
%   function GetConfigParameter(Key) ConfigParameterValue
% \end{lstlisting}

\medskip
\noindent
\textbf{Local graph view.}
For each of its downstream connections, a query processing unit maintains a data structure that represents the
\textit{set of queries that this connection can process}, which we term \textit{query processing capabilities}.
It the query processing capabilities of its neighbors in order to: (1) validate if it can process a given
query, and (2) generate downstream queries for a given query.

The data structures for the query processing capabilities of a QPU's downstream connections compose it local graph view
state.

In section~\ref{sec:qpc_tree} we present query processing capabilities data structure as well as its use in detail.

\medskip
\noindent
\textbf{Query processing state.}
Query processing units that encapsulate derived state structures such as indexes, materialized views and
caches, store these data structures in the part of their state that we call \textit{query processing state}.
Additionally, streaming operator QPUs such as joins use this part of the state to store intermediate state.

We model the query processing state as set of key/value, ordered by key;
Keys are string values, and values are lists of data items, represented as:
\[
  (DataItemID, [(AttributeName, AttributeValue, AttributeValue)], Timestamp)
\]

Using pseudocode, we can represent the query processing state's API as follows:

\begin{lstlisting}[caption={Pseudocode for the QPU's query processing state},captionpos=b,label={lst:qpustate}]

type AttributeName string

type AttributeValue union {
  string
  float
  int
  dateTime
  timestamp
}

type StreamRecord {
  dataItemID  string
  attributes  [(AttributeName, AttributeValue, AttributeValue)]
  ts          timestamp
}

class QueryProcessingState
  function Get(Key_low, Key_high) [(Key, [StreamRecord])]
  function Put(Key, [StreamRecord])

\end{lstlisting}

$Get$ retrieves the query processing state entries with $Key_low$ $<$ $key$ $\leq$ $Key_high$.
$Put$ modifies the value of the query processing state for a given key.
Put can be used with a non-existing key to create a new entry, or with an empty value to delete an entry.


\subsubsection{Initialization function}
\label{sec:initialization_func}

Each QPU invokes an initialization function when it starts its execution.

\begin{lstlisting}[caption={Initialization function signature},captionpos=b,label={lst:init_func}]
type DownstreamConnection {
  endpoint sting
  capabilities QPCapabilities
}

function Init(QPState, [DownstreamConn])
\end{lstlisting}

\noindent
Listing~\ref{lst:init_func} shows the initialization function's signature.
$ProcessQuery$ receives as arguments:
\begin{itemize}
  \item $QPState$, which is a handler that enables $Init$ to read from and write to QPU's query processing state.

  \item A list of $DownstreamConn$, each consisting of the endpoint of a downstream connection and a data structure
  representing the connection's query processing capabilities (section~\ref{sec:qpc_tree}).
  $Init$ can therefore send downstream query requests.
\end{itemize}


\subsubsection{Query processing function}
\label{sec:query_processing_func}

The query processing function is responsible for processing a given query and emitting the results to the output stream.

For each query $q$ being processing, the QPU initiates an output stream, $R_q$,
and executed an instance of the query processing function, $QPF_q$.
$QPF_q$ is responsible for emitting records to $R_q$.
Moreover, if $QPF_q$ initiates input streams $I_{q-1}$, $I_{q-2}$, ..., then $QPF_q$ is responsible for handing the return
values of callback functions for records received through $I_{q-i}$.

\begin{lstlisting}[caption={Query processing function signature},captionpos=b,label={lst:query_processing_func}]
function ProcessQuery(QueryRequest, QPState, [DownstreamConn],
                      ResponseStream)
\end{lstlisting}

\noindent
\begin{sloppypar}
Listing~\ref{lst:query_processing_func} shows the query processing function's signature.
In addition to $QPState$ and $[DownstreamConn]$, $ProcessQuery$ receives:
\end{sloppypar}
\begin{itemize}
  \item $QueryRequest$, which represents the received query request.

  \item $ResponseStream$, a handler it can use to emit result to the output stream.

\end{itemize}

An instance of the query processing function is executed for each received query request,
therefore query processing unit can run multiple instances of the query processing function in parallel.

\subsubsection{Input stream callback function}
\label{sec:callback_func}

The input stream callback function is responsible for processing a record received through an input stream.
The callback function can read from and write to the query processing unit state,
and a (potentially empty) list of $StreamRecord$ to the query processing function.

\begin{lstlisting}[caption={Input stream callback function signature},captionpos=b,label={lst:callback_func}]
function ProcessInputRecord(StreamRecord, QPState, DownstreamConn)
          returns [StreamRecord]
\end{lstlisting}

\noindent
Listing~\ref{lst:callback_func} shows the query processing function's signature.

An instance of the callback function is executed for each record received through an input stream.


\subsection{QPU classes}
\label{sec:qpu_classes}

As described in the previous section, the query processing unit model has the role of ``template'',
defining unified semantics that every QPU conforms with.
This ensures that query processing units can be arbitrarily interconnected and interoperate to implement query processing
tasks.

A QPU class is an \textit{instantiation} of the query processing unit model, which defines implementations for
the \textbf{query processing function} and the \textbf{input stream callback function}.

In this section, we present a categorization of QPU classes according to their general characteristics,
and demonstrate some specific examples of QPU classes.
We intentionally do not present a more extended list of QPU classes as it is out of the scope of this chapter.
We present additional QPU classes in chapters~\ref{ch:case_studies}, \ref{ch:proteus}, \ref{ch:evaluation}.

We categorize QPU classes in three groups, according to their general characteristics:
\begin{itemize}
  \item \textbf{Relational operator QPUs}.
  Classes in this group can be viewed as \textit{streaming relational operators}.
  They receive input data streams and perform filtering or transformation of these streams.
  Every input stream record, results in the QPU emitting zero or one records in the output stream.

  Conforming to the QPU specification, every input stream is the output stream of another QPU,
  and every output stream is the response to a query request.

  Some examples QPU classes in this group are:
  \begin{itemize}
    \item \textbf{Filter:}
    A filter QPU implements a streaming filter operator.
    \todo{more details in example}
    \item \textbf{Join}:
    A join QPU implements a streaming join operator.
    Given a query with a join operator,
    the join QPU is responsible for initiating input streams by sending the appropriate query requests to downstream QPUs,
    performing a streaming join operation on the input streams, and emitting the result at its output stream.
    \item \textbf{Aggregator}:
    An aggregator QPU implements a aggregation function over an input stream, such as count, sum, average, min or max.
    The aggregator QPU emits an record at the output stream for each input record that changes the aggregation value.
  \end{itemize}

  \item \textbf{Derived state QPUs}.
  Classes in this group implement derived state structures, such as indexes, caches, and materialized views.

  QPU classes of this group make use of the same use of the ``query request - output stream'' semantics to implement
  their functionality:
  \begin{itemize}
    \item \textbf{Secondary index and Materialized View:}
    An secondary index (or materialized view) QPU initiates an input steam by sending a query request with an \textit{interval query without
    an upper bound timestamp}.
    In that way, the QPU effectively \textit{subscribes to notifications} for updates to the corpus.

    For each input record, the QPU's callback function updates the QPU's query processing state accordingly.
    When receiving a query request, the unit computes the results by reading from its query processing state,
    and emits them to the output stream.

    For simplicity we assume that a secondary index QPU maintains an index for a single attribute
    (and the same for materialized view QPUs respectively).

    
    \item \textbf{Cache:}
    A cache QPU stores query results at its query processing state.
    When receiving a query request, the QPU's query processing function first determines it has stored the query result,
    and if yes it retrieves and emits it at the output stream.
    Alternatively, the query processing function sends query request at a downstream connection, forwarding the same query.
    The callback function the stores each received record at the query processing state, and then emits it at the output
    stream.
    \end{itemize}

  \item \textbf{Routing QPUs}.
  Classes of this group are responsible for implementing higher level functionalities, such as coordinating partitioned or
  replicated derived data structures, or performing load balancing.

  Examples of classes in this group include:
  \begin{itemize}
    \item \textbf{Partition manager:}
    A partition manager QPU is responsible for managing access to set of QPUs that implement index or materialized view
    partitions.
    The unit has outgoing edges in the QPU graph to a number of partitions.
    It also maintains at its connection's capabilities state information about the partitioning scheme and the portion of
    the partitioned space that each of its connections corresponds to.

    When receiving a query request, the QPU's query processing function uses these information to determine which
    partitions need to be contacted for the given query,
    and sends query requests to the corresponding downstream connections.
    The QPU then combines the resulting input streams and emits the combined stream as its output stream.

    \item \textbf{Load balancing and replica manager:}
    QPUs of these classes have similar functionalities with the partition managers:
    given a query they select the most suitable among their downstream connections, according to a certain criterion specific
    to each class, forward the given query, and finally forward the resulting input stream to their output stream.

    \end{itemize}

  \item \textbf{Database driver QPUs}
  The database driver class is responsible for connecting the QPU graph with the corpus.
  It is particular class, as database driver QPU do not support downstream connections to other QPUs.
  As a result they are located at the leaves of the QPU graph.

  Database driver QPUs apply a restriction on the query language.
  Their query interface supports queries of the form:
  {\obeylines\obeyspaces
  \texttt{SELECT SelectExpression from TableName TIMESTAMP TimestampExpression}}

  Given a query, the query processing function uses the interface and mechanisms of they corpus database in order to generate and emit
  the output stream.

  Database driver QPUs acts as wrappers that expose a common interface and semantics --- those defined by the QPU model --- to the
  QPU graph, independent of how the corpus is stored and accessed.

  As a result database driver classes are database-specific.
  QPU-based query processing systems are compatible with any corpus database as long as there is a corresponding database driver class
  to provide the interface with that database.
\end{itemize}

\subsubsection{QPU class case studies}

\textbf{Filter}


% % \begin{algorithm}
% % \caption{Query processing function signature}
% % \label{func:query_processing}
% % \begin{algorithmic}
% % \Function{ProcessQuery}{QueryRequest, State, DownstreamNeigbors, ResponseStream}
% % \end{algorithmic}
% % \end{algorithm}
% Every QPU implements a function that is responsible for processing queries request received by the QPU.
% The query processing function is the core to the query processing unit's functionality.
% As different QPU classes have implement different functionalities, the query processing function's implementation is
% class-specific.



\bigskip
\noindent
\textbf{Secondary index}


\section{QPU-based query processing systems}
\label{sec:query_processing_system}

\subsection{Query processing system architecture}

\begin{figure}[t]
  \centering
    \includegraphics[width=0.4\textwidth]{./figures/design_pattern/qpu_graph_emergent_properties.pdf}
  \caption{QPU graph example}
  \label{fig:qpu_graph_emergent_properties}
\end{figure}

A \textit{QPU-based} query processing system is a directed acyclic graph (DAG).
Graph nodes are query processing unit instances.
Edges represent potential query request - response steam relations between QPUs.
A directed edge from $QPU_a$ to $QPU_b$ indicates that $QPU_a$ can send query requests to $QPU_b$.
When $QPU_a$ sends a $QPU_b$ a query request to $QPU_b$, a stream of query results with the opposite direction is established between them.

Leaf nodes (nodes with no outgoing connections) are always database driver QPUs.
Queries enter the QPU graph through root node (nodes with no incoming connections).
\todo{ref}

The capabilities of a QPU-based query processing system are emergent from the functionalities of the QPUs at its node, as well as the
graph topology.
For example, consider the QPU graph depicted in Figure~\ref{fig:qpu_graph_emergent_properties}:
\begin{itemize}

  \item $DB$ $driver_1$ can process queries with the form:
  {\obeylines\obeyspaces
  \texttt{SELECT SelectExpression from Patients TIMESTAMP TimestampExpression
  }}
  where $SelectExpression$ can contains one or more attributes of the $Patients$ table.

\item $DB$ $driver_2$ can process queries with the form:
{\obeylines\obeyspaces
\texttt{SELECT SelectExpression FROM KnowledgeBase TIMESTAMP TimestampExpression
}}
where $SelectExpression$ contains one or more attributes of the $KnowledgeBase$ table.

\item $Filter_1$ can send downstream queries to $DB$ $driver_1$
and apply a filter to its input stream.
Therefore, $Filter_1$ can process queries of the form:
{\obeylines\obeyspaces
\texttt{SELECT SelectExpression FROM Patients
        WHERE PredicateExpression
        TIMESTAMP TimestampExpression
        }}
where $SelectExpression$ and $PredicateExpression$ can use attributes of the $Patients$ table.

\item Similarly, $Filter_2$ supports queries of the form:
{\obeylines\obeyspaces
\texttt{SELECT SelectExpression FROM KnowledgeBase
        WHERE PredicateExpression
        TIMESTAMP TimestampExpression
        }}
where $SelectExpression$ and $PredicateExpression$ can use attributes of the $KnowledgeBase$ table.

\item $Join$ can send downstream queries to $Filter_1$ and $Filter_2$
and join the two input streams based on a given attribute.
Assuming that the $Patients$ and $KnowledgeBase$ tables have a common attribute, $symptoms$,
$Join$ can process queries of the form:
{\obeylines\obeyspaces
\texttt{SELECT SelectExpression
        FROM Patients JOIN KnowledgeBase ON Patients.symptoms = KnowledgeBase.symptoms
        WHERE PredicateExpression
        TIMESTAMP TimestampExpression
        }}
where $SelectExpression$ and $PredicateExpression$ uses attributes from both tables.

\end{itemize}

\todo{ref}

\subsubsection{QPU-graph topology rules}
topology cannot be arbitrary

rules that stem from the query processing units' functionalities

non-exhaustive list,
For each QPU class can be generalized to two types of rules:
(1) number of downstream connections, and (2) the \textit{query processing capabilities} of the downstream connection

\begin{itemize}
  \item All QPU graphs mush have database driver QPUs as their leaves.
  Database drivers generate the initial streams that any other QPU builds on.

  \item Most QPUs in the relational operator and derived state groups must have a single downstream connection.
  Exception consists relational operator QPUs that by definition operate on more that one input streams, such as join QPUs.

  \item A materialized view QPU must have a downstream connection that can provide a stream of updates on the query
  that defines the materialized view
  In more detail, the downstream connection of a materialized view QPU must be the root of a subgraph that
  \begin{itemize}
    \item Can process the query the is the materialized view's definition.
    \item Supports \textit{interval queries} on that query.
  \end{itemize}
  \todo{use existing example}

  \item Similarly, a secondary index QPU must have a downstream connection that can provide a stream of updates on the
  attribute that QPU is configured to index.

  \item A cache QPU must have at least on downstream connection.
  This connection can be towards any valid sub-graph.

  \item Similarly, a filter QPU can be connected to any valid sub-graph.

  \item A partition manager QPU must have one or more downstream connections to derived state QPUs that implement
  partitions of a derived state structure.
  In more detail:
  \begin{itemize}
    \item All downstream connection must be of the same QPU class

    \item They must be configured as partitions of a logical global derived structure, based on a single partitioning key.
  \end{itemize}

  \item A load balancer QPU must have one or more downstream connections, and the \textit{intersection} of the queries
  supported by these downstream QPUs must be non empty.
  Because any of the downstream QPUs can process the queries at the intersection of their supported queries, the load
  balancer QPU can perform distribute these queries among its downstream connections.
\end{itemize}

\subsection{Computation model}
\label{sec:computation_model}

% In this section we present how QPU-based query processing systems process queries.
% We first present an overview of the computation model \ref{sec:computation_model}, and then ...

The query processing unit computation model combines elements from microservice architectures and stream processing systems.
Similar to stream processing systems,
query processing units operate on input streams either to incrementally update their state, or to perform query processing
computations.
Similar to microservices architectures, any stream is initiates as a response to a \textit{service request}.

\medskip

The computation run by a QPU-based query processing system directly emerges from the QPU specification.
We describe this collective computation by describing the execution of a generic query $q$ by a QPU graph.
Given a query $q$, the query processing function of a QPU $Q$ at the root of a QPU graph sends query request to
downstream connection in order to initialize input stream required for processing the query.
The QPU performs a computation on the input streams, potentially also involving its state, and emits the results at its
output stream.
The same process is performed at each of the downstream connection of $Q$, and their downstream connection, propagating
downwards through the QPU graph.
This creates a \textit{query execution sub-graph} composed of the QPUs that participate in the processing of $q$.

The leaves of this sub-graph are QPU that can process their given queries without sending downward query requests.
This includes database driver QPUs, and derived state QPUs.
These QPUs become the leaves of $q$'s execution sub-graph.
Leaf QPUs produce output streams that are then received as input stream at \textit{upstream QPUs}.
Progressively, each non-leaf QPU in the query execution sub-graph receives an input stream for each query request sent,
and produces itself an output stream.
Finally, $Q$ calculates the response for $q$ and emits it as its output stream.
We call this type of QPU graph computation \textit{query execution mode}.

\medskip
our ````````````````````````''''''''''''''''''''''''
The same process is performed for incrementally updating secondary index and materialized view QPUs (\textit{state maintenance mode}).
There are the following differences between query execution mode and state maintenance mode:
\begin{itemize}
  \item Query execution happens in response to a client query,
  while state maintenance happens in response to the initialization of a secondary index of materialized view QPU.
  \item The goal of query execution is to process a client query,
  while the goal state maintenance is to establish a long running stream of notification for corpus updates.
  \item The root of a query execution sub-graph is one the QPU graph root nodes,
  while the root of a state maintenance sub-graph is a derived state QPU.
\end{itemize}

\medskip

Based on the above description, the computation run of a QPU-based processing system can be characterized as a
\textit{bi-directional dataflow computation}.
For a given query or state maintenance execution,
query requests flow downwards through the QPU graph, defining an execution sub-graph.
Response streams flow upwards through that sub-graph, each stream corresponding to an edge defined by a query request.

\medskip

Finally, query processing units are multi-threaded:
a QPU can process multiple queries in parallel.
Therefore, multiple different query execution and state maintenance sub-graph can co-exist in parallel in the same QPU
graph.

\subsection{Query execution}

\subsubsection{Query parse tree}

The first of a QPU's query processing function is to parse the query request to a form that can be used for query processing.
This form is a \textit{parse tree}.

A parse tree is constructed by applying the QPU query language's grammar to a given query string.
More specifically, a parse tree is composed to three types of nodes:
\begin{itemize}
  \item \textbf{Atoms}, which include keywords of the query language ($SELECT$, $FROM$ etc.),
  identifiers, such as table or attribute names, constants, operators and tokens.
  Atoms are leaf nodes of the parse tree.

  \item \textbf{Syntactic categories}, which are constructs built from other syntactic categories, or atoms,
  follow the query language's grammar rules ($SelectExpression$, $TableExpression$ etc.)
  Syntactic categories are internal nodes of the parse tree
\end{itemize}

Given a query string $s$, a parse tree is constructed by parsing $s$ using the query language grammar rules.
\todo{ref parsing algorithms}

\subsubsection{Query processing capabilities tree}
\label{sec:qpc_tree}

In this section we present the \textit{query processing capabilities tree} (QPC tree)
The QPC tree represents the \textbf{set of all query parse trees that a QPU can process}.

To achieve that, we introduce an additional typo of tree node, the \textbf{conjunction} \todo{better name??}.
Conjunction nodes essentially express the different branches of a syntax rule.
For example, the rule:
{\obeylines\obeyspaces
\texttt{
PredicateExpression  ::=  PredicateExpression OR PredicateExpression |
~~~~~~~~~~~~~~~~~~~~~~~~~~PredicateExpression AND PredicateExpression |
~~~~~~~~~~~~~~~~~~~~~~~~~~NOT PredicateExpression |
~~~~~~~~~~~~~~~~~~~~~~~~~~Term Op Term
}}

can be represented as \todo{do the tree}.


.. Present the trees in the orders example ..

Each QPU has a QPU tree, which is parts of its configuration state.
Stores the QPC trees of its neighbors as part of its local view state.

It is used for two purposes
\begin{itemize}
  \item The QPU uses determines if it can process a given query by the parse tree of the given query ``fits'' its QPC tree.
  \item The QPU generates downstream query requests using the QPC trees of its downstream connections, by performing a
  trimming operation.
\end{itemize}




% Given $q$, the query processing function of a QPU $Q$ at the root of the QPU graph sends query request to some its
% downstream connection, in order to initialize input stream required for processing the query.
% Upon receiving a query request from $Q$, each of its downstream connection performs the same process.
% In that way, query requests are propagated downwards through the QPU graph, creating a \textit{query execution sub-graph}
% composed of the QPUs that participate in the processing of $q$.

% As query requests requests are propagated downwards, expanding the $q$'s execution sub-graph, eventually some QPUs can
% process their queries without sending further query requests downwards.
% This occurs in database driver and derived state QPUs.
% The leaves of this sub-graph are QPU that can process 
% This includes database driver QPUs, and derived state QPUs.
% These QPUs produce output streams that are then processed as input stream at \textit{upstream QPUs}.

% Progressively, each non-leaf QPU in the query sub-graph receives an input stream for each query request sent,
% and produces it self an output 


% We describe how a QPU graph processes a given query using the example of Figure~\ref{fig:qpu_graph_emergent_properties}.

% Consider the query:
% {\obeylines\obeyspaces
% \texttt{Q = SELECT CustomerName, CustomerEmail, OrderID
%         ~~~~FROM Customers
%         ~~~~INNER JOIN Orders ON Customers.OrderID = Orders.OrderID
%         ~~~~WHERE Orders.OrderDate >= 2020-08-20 AND Orders.OrderDate < 2020-09-02
%         ~~~~TIMESTAMP FROM LATEST TO LATEST
%         }}





% - queries / control messages flow downwards.
% - responses flow upwards through the sub-graph defined by the queries
% - (each query establishes a sub-graph - actually a tree - to be used by that particular query);
% - Updates (independently) also flow upwards

% As described in the previous sections, query processing units can collaborate by invoking the query API of one another.
% QPUs can be composed in DAG hierarchies in which parent units can invoke the query API of their child units.
% Query responses for both snapshot and persistent queries are streams of results.
% Communication between QPUs thus uses a combination of the remote procedural call model (query API invocations) and the stream-processing model (query responses).

% Clients perform queries by invoking the query API of units at root nodes of the graph.
% Once a QPU receives a query, it determines whether it can directly process it, for example by performing lookups at its indexing structures,
% and if that is the case it responds with the query result.
% In case the query processing computation requires partial query results from its child units, the unit invokes the query API of those units with the appropriate sub-queries.
% Sub-query results are received and processed through the unit's callback function.

% This process is recursively performed at each query processing unit.

% Therefore, the computation that runs in QPU a graph can be modeled as a bidirectional data-flow computation.
% Queries flow downwards through the graph, are incrementally split into sub-queries, and processed across the graph.
% Sub-query results flow back upwards, are incrementally processed, and eventually produce the results to the initial query.

% The same computation model is used for index maintenance.
% We have extended the query interface semantics so that the query API can be used to subscribe to notifications for corpus updates, and query results can encode these updates.
% Using this mechanism, units with indexing functionalities can subscribe to corpus updates by invoking the query API of QPUs that provide this functionality.
% We describe this mechanism in more detail in Section \ref{subsec:query_classes}.


% The QPU graph runs a distributed bidirectional data-flow computation.

% A client performs a query $Q_c$ by invoking the query API of a query processing unit at the root of the graph.
% As described in Section~\ref{subsec:qpu}, the QPUs query processing computation can read from the unit's state,
% or perform downstream queries to QPUs at its child nodes.

% When a downstream query is performed, this process is recursively executed at each unit whose query API is invoked.
% Though this mechanism, $Q_c$ is incrementally transformed to sub-queries which flow downwards through the QPU graph,
% invoking computations at different nodes.
% Sub-query results are returned through the QPU streams established from query API invocations, and flow upwards
% through the graph.
% These results are incrementally processed, potentially updating the state of different QPUs, and eventually
% produce the initial query results, which are returned to the client.









% \section{QPU Composition: Constructing QPU-based query engines}
% Here, describe how query engines are constructed using instances of QPU classes.

% A query engine is a directed acyclic graph (why?) with QPUs as nodes.

% Describe the QPU-specific topology properties that the graph must satisfy in
% order to be functional:
% \begin{itemize}
%   \item All leaves must be QPUs of the datastore driver class.
%   Also, datastore driver QPUs cannot be nodes other than leaves.
%   \item TODO
% \end{itemize}

% \section{Computation Model}
% Describe the bidirectional data-flow computation.
% \begin{itemize}
%   \item Control messages (queries) flow downwards.
%   \item Responses flow upwards through the sub-graph defined by the control
%   (each query establishes a sub-graph - actually a tree - to be used by that
%   particular query);
%   \item Updates (independently) also flow upwards.
% \end{itemize}


% % \section{The consistency guarantees of QPU-based query engines}
% % TODO: What mechanism are needed so that QPUs can guarantee internal consistency?
% % What about session consistency guarantees?

% \section{Discussion}

% discussion: input output disconnected here
    % discussion: input output disconnected here

% % -- point to make -- unified subscription and snapshot and routing
% graph topology is distributed

% % It is known that query execution can be represented as a tree.
% % Base data are at the tree's leaves, tree nodes are relational operator such
% % as filter, joins and aggregations, and query result are at the root of the tree.
% % Each operator receives an input stream of records, performs a transformation on the, and emits a output stream of record.
% % This results to a data-flow computation in which data items flow upwards through the tree and are progressively transformed
% % to the query execution results.

% % Computation tasks vs architecture components

% % Our approach is based on three simple insights:
% % \begin{itemize}
% %     \item \textbf{Index-based distributed query processing is composed of basic, primitive tasks.}
% %     The most simple query engine design is a component that scans the corpus dataset and selects the data items which match a given query.
% %     When queries for certain attributes are frequent or require low response time, secondary indexes can be materialized for those attributes.
% %     Even when indexes are partitioned and distributed across the system for scalability, the basic components of an indexing system remains the same: index data structures that collaborate through appropriate index maintenance and query processing protocols to implement distributed indexes.
% %     Additional query processing functionalities such as multi-attribute queries, joins or federated queries across multiple corpus can be implemented using operators that build on top of these components.
% %     Finally, caching can be used to further improve response time for certain queries.

% %     \item \textbf{Primitive query processing tasks can be encapsulated by a common query processing component model.}
% %     Each of the described tasks can be encapsulated by a query processing component model with two properties: an API for responding to queries, and a callback function.
% %     For example, index data structures in general implement three basic functions: LOOKUP, INSERT, DELETE.
% %     The query API can encapsulate the LOOKUP function, while the callback function can express the task of index maintenance, receiving corpus updates and updating the index accordingly, and therefore encapsulate INSERT and DELETE.
% %     As another example, a cache on top of an index structure should be able to respond to the same queries as the underlying index, and therefore can be encapsulated by a query processing component with the same query API.
% %     Similarly, its callback function can express the task of receiving query results in case of a cache miss, and updating the cache.
% %     This concept can be generalized to represent other query processing components including bloom filters, materialized views, and streaming operators.

% %     \item \textbf{Query processing components need to cooperate to implement complex query processing tasks.}
% %     As an example, a caching component requires the ability to forward queries to other components, such as indexes, when cache misses occur.
% %     Similarly, a distributed indexing system, in which indexes are partitioned and distributed across system nodes, can be implemented using indexing components with the help of an additional component responsible for implementing a partitioned LOOKUP operation, by collecting and aggregating partial LOOKUP results.
% % \end{itemize}

% % Based on these observations, we introduce a query processing component model, called the \textit{Query Processing Unit} (QPU).
% % We have designed a generic query API and callback function interface with the aim of encapsulating multiple different query processing components.
% % Query processing units may have internal state for facilitating query processing, or be stateless.
% % Additionally, QPUs have the ability to invoke the query API of one another, and thus interoperate for query processing.


\bibliographystyle{plainnat}
\bibliography{refs}

\chapter{Case studies}
\label{ch:case_studies}
Having presented an approach for constructing query processing architectures through
assembly-based modularity,
in this chapter we demonstrate its flexibility by applying it to a number of use cases and applications.
More specifically:

\begin{itemize}
  \item We demonstrate how both a document and a term-partitioned secondary
  index can be implemented using QPU-based architectures (section~\ref{sec:cs_index_partitioning}).
  In addition,
  we use this case study to describe in detail the construction and functionality of QPU graph,
  including the configuration of query processing units, and the queries sent between units
  during initialization and query processing.

  \item We examine a state-of-the-art approach to providing federated metadata search over data spread
  across multiple private and public cloud storage platforms,
  and propose a decentralized approach that places index entries closer to the corpus in order to improve
  freshness (section~\ref{sec:zenko}).
  Moreover, we present a QPU architecture that implements partial index replication.
  This approach provides fine-grained control over the placement of index entries:
  Heavily queries index entries are placed closer to the users,
  while heavily update are placed closer to the corpus

  \item We examine a read-heavy news aggregator application,
  and propose a QPU-based query processing system for maintaining pre-computed state in order to improve the application's performance,
  while simplifying its code (section~\ref{sec:lobsters}).
  To achieve that, we present a QPU that implements partial materialization in materialized views:
  Only materialized view entries expected to be requested by the application are materialized.
  This reduces the view's memory footprint as well as the volume communication required for keeping view entries up-to-date
  with the corpus.
  Furthermore, we demonstrate how a QPU-based architecture can be distributed between data center and edge nodes,
  and use partial materialization to ensure that only heavily queried materialized view entries are placed at the edge.
\end{itemize}

\section{Flexible secondary index partitioning}
\label{sec:cs_index_partitioning}

As described in section~\ref{sec:index_partitioning_background}, the two main index partitioning schemes are:
\begin{itemize}
  \item \textbf{Partitioning by document}.
  Each index partition is responsible for the data items of a certain corpus partition,
  and is co-located in the same node as that corpus partition.
  \item \textbf{Partitioning by term}.
  Each index partition is responsible for a partition of the \textit{value space} of the indexed attribute;
  The placement of index partitions is independent from the placement of corpus partitions.
\end{itemize}

Experimental comparison of the two approaches \cite{dsilva:tworings, kejriwal:slik} has shown that there is no ``one-size-fits-all'' approach to secondary index partitioning.
Rather, each approach caters to different needs.
More specifically, partitioning by document is more suitable for:
\begin{itemize}
  \item Workloads with low selectivity queries, that return large result sets.
  \item Skewed distributions, in which a large number of data items correspond to a few attribute values.
  \item Write-intensive workloads that require low write latency.
\end{itemize}
\noindent
On the other hand, partitioning by term is more suitable for:
\begin{itemize}
  \item Large-scale systems with a large number of corpus partitions.
  \item High selectivity query workloads.
  \item Less skewed value distributions.
\end{itemize}

Because of that, applications can benefit from partitioning schemes adjusted to their data and workload characteristics.

In most existing query processing systems,
the choice of index partitioning scheme is made during the system's design, and is not configurable
by applications.
For example, MongoDB \cite{coubase:mongoindexes}, Cassandra \cite{cassandra:secondaryindexing} and Riak \cite{riakv:secondaryindexes}
only use the partitioning by document approach,
while HBase \cite{hbase:secondaryindexes} uses the partitioning by term approach.

\bigskip
\noindent
In this section, we demonstrate the flexibility of the QPU-based query processing architecture by showing how it can be used to
be used to express both index partitioning schemes.
More specifically,
we present a QPU architecture that implements a document-partitioned index,
and one that implements a term-partitioned index.

The design of the QPU architectures that we present is based on observations about the properties of the two index partitioning schemes.
We categorize these observations using the derived state read and write path framework presented in
section~\ref{sec:read_write_path}:
\begin{itemize}
  \item In the partitioning by document scheme,
  the index \textbf{write path} is \textit{local}:
  An index partition receives updates only from the corpus partition it is co-located with.
  On the other hand, in the partitioning by term scheme,
  the write path involves a many-to-one relationship:
  An index partition receives from all corpus partitions updates that belong in the value space it is responsible for.

  \item In the partitioning by document scheme,
  the \textbf{read path} is a \textit{broadcast} operation,
  while partitioning by term
  only index partitions with relevant index entries are involved in processing a given query.
\end{itemize}

We present QPU architectures for the two index partitioning approaches using the photo album example of section~\ref{sec:read_write_path} as a reference.
The corpus is composed of a set of image files.
Each image is identified by a primary key, and is associated with a set of user-defined tags.
The corpus is partitioned using a hash of the primary key as the partitioning key.
An application needs to create a secondary index on the $predominantColor$ tag, which can be assigned values in the range $[\#000000$, $\#FFFFFF$].

Figures~\ref{fig:index_partitioned_by_document} and~\ref{fig:index_partitioned_by_term} show the QPU graphs for a
document-partitioned index and a term-partitioned index respectively, and their placement across system nodes.
For simplicity we assume that the number of corpus partitions is equal to the number of system nodes,
and that a corpus partition is placed on each node.

The two architectures have a number of general characteristic in common.
In both architectures, a Corpus Driver QPU is placed on each node,
and is responsible for the corresponding corpus partition.
It provides the QPU graph with access to the data items in that corpus partition,
as well as the updates performed on those data items.
In addition, an Index QPU is used to represent each index partition;
A partition Manager QPU is connected to all index partitions,
and is responsible for coordinating query access to them.

In sections~\ref{sec:cs_index_partitioning_write_path} and~\ref{sec:cs_index_partitioning_read_path} we describe
how the two QPU architecture achieve the write and read path properties described above:

\subsection{Write path}
\label{sec:cs_index_partitioning_write_path}

\begin{figure}
  \begin{minipage}{.5\textwidth}
    \centering
    \includegraphics[scale=0.5]{./figures/case_studies/index_partitioned_by_document.pdf}
    \caption{QPU architecture for a document-partitioned secondary index.}
    \label{fig:index_partitioned_by_document}
  \end{minipage}%
  \begin{minipage}{.5\textwidth}
    \centering
    \includegraphics[scale=0.5]{./figures/case_studies/index_partitioned_by_term.pdf}
    \caption{QPU architecture for a term-partitioned secondary index.}
    \label{fig:index_partitioned_by_term}
  \end{minipage}
\end{figure}

The write path properties described above are achieved through (1) the QPU graph topology and the placement of graph vertices across system nodes,
and (2) the configuration of the Index QPUs (index partitions).

\subsubsection{Graph topology and placement}

\medskip
\noindent
\textbf{Partitioning by document.}
In the partitioning by document approach, we place an Index QPU on each system node,
and connect it to the corresponding Corpus Driver QPU.
As a result, each index partition has access corpus partition it is co-located with,
and therefore constructs and maintains an index containing the data items that belong in the corpus partition.

\medskip
\noindent
\textbf{Partitioning by term.}
In the partitioning by term approach,
we connect each Corpus Driver QPU to a Selection QPU,
and connect each Index QPU to all Selection units.
A Selection QPU forwards each update to the relevant index partition,
based on the attribute value in the update.
We describe in detail how this is achieved in the following section.
\subsubsection{Index partition configuration and graph initialization}

When the QPU graph for one of the partitioned index architectures is initialized,
the initialization function of each Index QPU sends a persistent query to the QPU's downstream connection,
in order to establish an input stream of updates (the initialization function of the Index QPU is presented in section~\ref{sec:qpu_class_examples}).
This downstream query is generated by the initialization function according to the QPU's configuration.
More specifically, each index QPU is configured with an attribute ($predominantColor$ in this example),
and an \textit{interval of values} of that attribute.
The index partition is responsible for index entries that correspond to attribute values in the specified interval.

\medskip
\noindent
In the partitioning by document approach,
each index partitions are configured to be responsible for the entire attribute value space,
which in the photo album example is $[\#000000$, $\#FFFFFF$].
As a result, upon initialization, each index partition sends the query: \todo{update query according to changes to QPU chapter}

\begin{lstlisting}[
          language=SQL,
          showspaces=false,
          basicstyle=\ttfamily,
          commentstyle=\color{gray},
          rulecolor=\color{black},
          stringstyle=\color{mymauve},
          frame=L,
          xleftmargin=\parindent
        ]{}
SELECT primaryKey, predominantColor
FROM photoAlbum
INTERVAL FROM LATEST
\end{lstlisting}

\noindent
to the Corpus Driver QPU it is connected to.
This initiates a stream between each Corpus Driver QPU and the corresponding Index QPU
The Corpus Driver initially sends a snapshot of the corresponding corpus partition to the
Index QPU, and then sends a stream record for each corpus update.

Conversely, in the partitioning by term approach, each index partition is responsible for a non-overlapping subset of the
attribute value space.
A simplified version of the configuration of the index partitions of the term-partitioned index QPU architecture
(figure~\ref{fig:index_partitioned_by_term}) is depicted below: \\

\begin{minipage}{.45\textwidth}
\begin{lstlisting}[
          showspaces=false,
          basicstyle=\ttfamily,
          commentstyle=\color{gray},
          rulecolor=\color{black},
          stringstyle=\color{mymauve},
          frame=L,
          xleftmargin=\parindent,
          caption={Configuration of the $index$ $partition_1$ in Figure~\ref{fig:index_partitioned_by_term}},captionpos=b
          ]{}
{
  // other configuration
  // parameters
  "indexConfiguration": {
    "table": "photoAlbum",
    "attribute":
        "predominantColor",
    "lower_bound": #000000,
    "upper_bound": #7FFFFF
  }
}
\end{lstlisting}
\end{minipage}\hfill
\begin{minipage}{.45\textwidth}
\begin{lstlisting}[
  showspaces=false,
          basicstyle=\ttfamily,
          commentstyle=\color{gray},
          rulecolor=\color{black},
          stringstyle=\color{mymauve},
          frame=L,
          xleftmargin=\parindent,
          caption={Configuration of $index$ $partition_2$ in Figure~\ref{fig:index_partitioned_by_term}},captionpos=b
          ]{}
{
  // other configuration
  // parameters
  "indexConfiguration": {
    "table": "photoAlbum",
    "attribute":
        "predominantColor",
    "lower_bound": #7FFFFF,
    "upper_bound": #FFFFFF
  }
}
\end{lstlisting}
\end{minipage}

\medskip
\noindent
As a result, upon initialization, $index$ $partition_1$ sends the query: \todo{udpate according to QPU chapter}

\begin{lstlisting}[
          language=SQL,
          showspaces=false,
          basicstyle=\ttfamily,
          commentstyle=\color{gray},
          rulecolor=\color{black},
          stringstyle=\color{mymauve},
          frame=L,
          xleftmargin=\parindent
        ]{}
SELECT primaryKey, predominantColor
FROM photoAlbum
WHERE predominantColor <= #000000 AND predominantColor < #7FFFFF
SNAPSHOT LATEST
INTERVAL FROM LATEST
\end{lstlisting}

\noindent
to each downstream Selection QPU, and $index$ $partition_2$ sends the query: \todo{udpate according to QPU chapter}

\begin{lstlisting}[
          language=SQL,
          showspaces=false,
          basicstyle=\ttfamily,
          commentstyle=\color{gray},
          rulecolor=\color{black},
          stringstyle=\color{mymauve},
          frame=L,
          xleftmargin=\parindent
        ]{}
SELECT primaryKey, predominantColor
FROM photoAlbum
WHERE predominantColor <= #7FFFFF AND predominantColor <= #FFFFFF
SNAPSHOT LATEST
INTERVAL FROM LATEST
\end{lstlisting}

\noindent
to each downstream Selection QPU.

\noindent
As a result of receiving one the above queries,
a Selection QPU sends to each downstream Corpus Driver QPU the query:

\begin{lstlisting}[
          language=SQL,
          showspaces=false,
          basicstyle=\ttfamily,
          commentstyle=\color{gray},
          rulecolor=\color{black},
          stringstyle=\color{mymauve},
          frame=L,
          xleftmargin=\parindent
        ]{}
SELECT primaryKey, predominantColor
FROM photoAlbum
SNAPSHOT LATEST
INTERVAL FROM LATEST
\end{lstlisting}

As a result, each Selection QPU receives an input stream from the corresponding Corpus Driver QPU,
for each of the two above queries,
and filters the received streams according to the predicate specified by each query.
Therefore, $index$ $partition_1$ receives from all three corpus partitions first a snapshot and then updates
corresponding to the $predominantColor$ in the interval $[\#000000$, $\#7FFFFF)$.
Similarly, $index$ $partition_2$ receives stream records corresponding to the $predominantColor$ in the interval
$[\#7FFFFF$, $\#FFFFFF)$.

For example, given a update that inserts to the corpus an image file with $predominantColor$ $=$ $\#613930$, assigned to to $corpus$
$partition_2$, $Corpus$ $Driver_2$ sends to $Selection_2$ a record encoding that update.
This input record matches the query:

\begin{lstlisting}[
          language=SQL,
          showspaces=false,
          basicstyle=\ttfamily,
          commentstyle=\color{gray},
          rulecolor=\color{black},
          stringstyle=\color{mymauve},
          frame=L,
          xleftmargin=\parindent
        ]{}
SELECT primaryKey, predominantColor
FROM photoAlbum
WHERE predominantColor <= #000000 AND predominantColor < #7FFFFF
SNAPSHOT LATEST
INTERVAL FROM LATEST
\end{lstlisting}

and thus the $Selection_2$  forwards the record only to $index$ $partition_1$.

\subsection{Read path}
\label{sec:cs_index_partitioning_read_path}

In both QPU architectures, the Partition Manager QPU is connected to all Index QPUs.
As described in section~\ref{sec:qpc_tree}, given a query,
the Partition Manager's query processing function determines which downstream connections need to be contacted.
It generates the corresponding downstream queries using the query processing capabilities trees of its downstream connections.
\todo{update the term ``query processing capabilities tree'' according to updates in the QPU chapter}

More specifically, given a query,
the query processing function, generates a query parse tree,
and performs an intersection between the query parse tree and the QPC tree of each of its downstream connections.
The result of each intersection operation is a query parse tree of the downstream query to be sent to the corresponding
connection.
If the intersection result is empty, then the QPU does not send a downstream query to that connection.

\begin{figure}
  \centering
    \includegraphics[width=\textwidth]{./figures/case_studies/qpt_index_partitioning_docs.pdf}
  \caption{Query processing capabilities tree of an index partition in the document-partitioned index QPU architecture.}
  \label{fig:qpt_index_partitioning_docs}
\end{figure}

\medskip
\noindent
\textbf{Partitioning by document.}
The QPC tree for an index partition in the document-partitioned index QPU architecture is shown in Figure~\ref{fig:qpt_index_partitioning_docs}.
As describe above, in the document-partitioned index, each index partition is responsible for the entire attribute value space of
the data items of one corpus partition.
The QPC tree of every index partition is thus are the same as the one depicted in Figure~\ref{fig:qpt_index_partitioning_docs}.
Moreover, because this tree corresponds to the entire attribute value domain,
the intersection with any query parse tree is an identity function:
Performing an intersection between that QPC tree a query parse tree leaves the parse tree unchanged.

Therefore,
given a query,
the Partition Manager QPU forwards the same query to each of its downstream connections.
This implements the required read path behavior.

\begin{figure}
\subfloat[]{%
\centering
  \includegraphics[width=0.6\textwidth]{./figures/case_studies/qpt_index_partitioning_terms_1.pdf}%
}

\subfloat[]{%
  \centering
  \includegraphics[width=0.6\textwidth]{./figures/case_studies/qpt_index_partitioning_terms_2.pdf}%
}
\caption{The query processing capabilities tree for $index$ $partition_1$ (a) and $index$ $partition_2$ (b) of the term-partitioned
index in Figure~\ref{fig:index_partitioned_by_term}}
\label{fig:qpt_index_partitioning_terms}
\end{figure}

\medskip
\noindent
\textbf{Partitioning by term}
Figure~\ref{fig:qpt_index_partitioning_terms} shows the $<predicateExpression>$ subtrees of the QPC trees
for $index$ $partition_1$ and $index$ $partition_2$ in the term-partitioned index architecture.
The QPC tree of $index$ $partition_1$ represents the set of queries in the interval $[\#000000$, $\#7FFFFF)$,
while the QPC tree of $index$ $partition_2$ represents the interval $[\#7FFFFF$, $\#FFFFFF]$.

We describe how the term-partitioned index architecture processes queries by running through an example query.
Given the query:

\begin{lstlisting}[
          language=SQL,
          showspaces=false,
          basicstyle=\ttfamily,
          commentstyle=\color{gray},
          rulecolor=\color{black},
          stringstyle=\color{mymauve},
          frame=L,
          xleftmargin=\parindent
        ]{}
Q = SELECT primaryKey, predominantColor
    FROM photoAlbum
    WHERE predominantColor >= #21B1FF
      AND predominantColor < #ff7b75
    SNAPSHOT LATEST
\end{lstlisting}

\noindent
the Partition Manager QPU calculates the intersection between the parse tree of $Q$ the QPC tree of $index$ $partition_1$.
This results to the downstream query:

\begin{lstlisting}[
          language=SQL,
          showspaces=false,
          basicstyle=\ttfamily,
          commentstyle=\color{gray},
          rulecolor=\color{black},
          stringstyle=\color{mymauve},
          frame=L,
          xleftmargin=\parindent
        ]{}
Q1 = SELECT primaryKey, predominantColor
     FROM photoAlbum
     WHERE predominantColor >= #21B1FF
       AND predominantColor < #7FFFFF
     SNAPSHOT LATEST
\end{lstlisting}

\noindent
Similarly, the downstream query for $index$ $partition_2$ is:

\begin{lstlisting}[
          language=SQL,
          showspaces=false,
          basicstyle=\ttfamily,
          commentstyle=\color{gray},
          rulecolor=\color{black},
          stringstyle=\color{mymauve},
          frame=L,
          xleftmargin=\parindent
        ]{}
Q2 = SELECT primaryKey, predominantColor
     FROM photoAlbum
     WHERE predominantColor >= #800000
       AND predominantColor < #ff7b75
     SNAPSHOT FROM LATEST
\end{lstlisting}

\noindent
The Partition Manager QPU sends $Q1$ to $index$ $partition_1$ and $Q2$ to $index$ $partition_2$.
It then merges resulting input streams, and emits the merged stream as its output stream.

In summary, in the term-partitioned architecture,
the Partition Manager generates and sends downstream queries to index partitions according their corresponding
value intervals, using the QPC tree mechanism.

\bigskip
\noindent
In conclusion, we have described how the QPU-based composable query processing architecture can be used to construct both
document and term partitioned indexes.
Using this flexibility, a database system can support both partitioning schemes and provide applications the ability to
select the partitioning scheme of each indexes according the data distribution characteristics and the expected workload.

\begin{figure}
  \centering
    \includegraphics[width=\textwidth]{./figures/case_studies/index_partitioned_two_level.pdf}
  \caption{QPU architecture for a two-tiered partitioned index, composed of a document-partitioned, and a term-partitioned tier.}
  \label{fig:index_partitioned_two_level}
\end{figure}

It is important to note that some existing databases,
such as Amazon DynamoDB \cite{dynamodb:secondaryindexes} and Apache Phoenix \cite{phoenix:secondaryidnexing}
support both index partitioning schemes.
However, we believe that the proposed approach provides a more structured approach to the design of query processing systems,
and enables additional flexibility.

\medskip
\noindent
For example, using the QPU-based architecture we can construct a hybrid, two-tiered partitioned secondary index,
consisting of a document-partitioned tier and a term-partitioned tier, as shown in Figure~\ref{fig:index_partitioned_two_level}.
We configure the document-partitioned tier to be strongly consistent with the corpus,
by configuring each connection between an Index QPU and the corresponding Corpus Driver to be synchronous.
We configure the term-partitioned tier to be eventually consistent with the corpus,
by configuring each Index --- Corpus Driver connection as asynchronous.

The document-partitioned trades consistent query results with more limited query load scalability,
as each query needs to be forwarded to every index partition.
The term-partitioned index trades higher query load scalability, with potentially stale query results.

Using this query processing architecture, individual queries are able to choose between the two tiers
according to their requirements.
To achieve this, we introduce the Index Selector QPU class.
The Index Selector is responsible for managing access to the two index tiers,
by forwarding each query to one of the Partition Manager QPUs it connected to.
It selects between its connections based on an indication by the given query:
Queries can select between the two partitioned index tiers using a special ``control'' attribute, called $tier$.
The query:

\begin{lstlisting}[
          language=SQL,
          showspaces=false,
          basicstyle=\ttfamily,
          commentstyle=\color{gray},
          rulecolor=\color{black},
          stringstyle=\color{mymauve},
          frame=L,
          xleftmargin=\parindent
        ]{}
SELECT primaryKey, predominantColor
FROM photoAlbum
WHERE predominantColor == #ff7b75
AND tier = sync
\end{lstlisting}

\noindent
will be forwarded to the document-partitioned index, while the query:

\begin{lstlisting}[
          language=SQL,
          showspaces=false,
          basicstyle=\ttfamily,
          commentstyle=\color{gray},
          rulecolor=\color{black},
          stringstyle=\color{mymauve},
          frame=L,
          xleftmargin=\parindent
        ]{}
SELECT primaryKey, predominantColor
FROM photoAlbum
WHERE predominantColor == #ff7b75
AND tier = async
\end{lstlisting}

will be forwarded to the term-partitioned index.

This architecture trades this functionality with the resources required for maintaining
two index replicas.

%%%%%%%%%%%%%%%%%%%%%%%%%%%%%%%%%%%%%%%%%%%%%%%%%%%%%%%%%%%%%%%%%%%%%%%%%%%%%%%%%%%%%%%%%%%%%%%%%%%%%%%%%%%%%%%%%%%%%%%%


\section{Federated secondary attribute search for multi-cloud object storage}
\label{sec:zenko}

While most enterprise data today originate from and is stored on-premises storage systems,
use cases for hybrid and multi-cloud storage are emerging in many industries.
For example, in the media industry, while the creation of content in on-premises private clouds is prevalent,
the use of public cloud services for content distribution and transcoding \cite{scality:bloomberg} is growing.
Moreover, organization increasingly choose to spread their data across multiple public cloud providers in order to avoid
dependence on a single provider, and improve their resilience against failures.

The advent of data distributed across multiple independent storage platforms has created the need for unified access to data across platforms.
In this section, we examine Zenko \cite{zenko:docs}, a multi-cloud data controller that aims to address this need.
Zenko provides a common namespace over a set of distinct object storage platforms,
including Amazon S3, Microsoft Azure Blob storage, and Google Cloud Storage.
It allows applications to access data in multiple storage locations using the AWS S3 API \cite{aws:s3}.

We focus on Zenko's federated metadata search functionality \cite{zenko:mdsearch}.
It is common for application to mark objects with metadata tags.
Zenko provides applications the ability to retrieve objects based on queries on their metadata tags,
independent from storage location.

Zenko uses a warehousing approach to provide federated metadata search:
It integrates metadata tags for objects stored on all storage platforms in a central \textit{metadata store} \cite{zenko:architecture, zenko:mongodb}.
More specifically, it uses a MongoDB deployment as this metadata store.
MongoDB provides the ability to create indexes on document attributes,
and to retrieve documents using queries on these attributes.
Zenko uses this functionality to implement metadata search:
Object metadata are stored as documents in MongoDB.
Queries on metadata tags are translated to MongoDB find requests.

A typical Zenko deployment consists of Zenko along with a private storage system deployed on-premises,
and multiple public cloud storage platforms, each on a different data center.
For this case study, we refer to each data center / storage system as a \textit{storage location}.
The primary data access method is through Zenko's API:
Application communicate with Zenko to read and write objects,
and Zenko is responsible for storing and retrieving each object from the corresponding storage location according to a specified policy.
In addition,
applications can write directly to some of the storage systems.
Zenko ingests updates performed directly to an underlying storage system through a mechanism called out-of-band updates \cite{zenko:outofband}.

\medskip
\noindent
In this case study,
we present an alternative approach for supporting federated metadata search on object storage platforms.
More specifically, we demonstrate how the QPU-based query processing architecture can be used to construct
a \textit{decentralized} secondary index that federates data stored on multiple storage locations;
The main idea is to partition the secondary index based on the storage location of each object,
and place each index partition close to the corresponding data.

There are two main advantages to this approach compared to the warehousing approach:
\begin{itemize}
  \item It ensures that the write path of the index does not require cross data center communication.
  This means that the system can update index partitions synchronously without the prohibitive overhead of
  cross data center communication.
  Alternatively, if index maintenance is asynchronous,
  the index can by updated in a more timely fashion, resulting in less stale index entries.
  In general, a decentralized index is better suited for applications that require up-to-date query results.

  \item By being distributed across multiple data centers,
  the query processing system can remain available in the face of a data center failure.
\end{itemize}

We make two assumptions:
The first is that storage platforms used in this case study have a common, object storage data model.
Second, a Corpus Driver QPU is available for each storage platform.
This assumptions ensure that a QPU graph can be connected with multiple underlying storage systems.

\medskip
\noindent
The QPU architecture for the multi-cloud index is shown in Figure~\ref{fig:federated_index} (a).
We built on the partitioned index QPU architectures presented in Section~\ref{sec:cs_index_partitioning}.
We construct a partitioned index QPU graph for each storage system, and co-locate it in the same data center as that
storage system.
We refer to each of these QPU graphs as an \textit{index location}.
Each index location is independent and can be constructed using either the partitioning by document or partitioning by
term approach.
In addition, a Partition Manager QPU is deployed on each storage location,
and connected to all index locations.
The Partition Manager QPUs consist the root nodes of the QPU graph.
Given a query, a Partition Manager QPU forwards the same query to every index location,
and then merges the resulting input streams.
This approach is equivalent to the document-partitioned index approach.

\begin{figure}
  \begin{minipage}{.5\textwidth}
    \centering
    \includegraphics[scale=0.24]{./figures/case_studies/federated_index.pdf}
  \end{minipage}%
  \begin{minipage}{.5\textwidth}
    \centering
    \includegraphics[scale=0.24]{./figures/case_studies/federated_index_remote.pdf}
  \end{minipage}
  \caption{QPU architecture for a secondary index that federates two storage locations.
  (a) Using a Partition Manager as a root at each storage location that forwards queries to every storage location.
  (b) Using \textit{partial index replicas} implemented by Partial Index QPUs.}
  \label{fig:federated_index}
\end{figure}

\medskip
\noindent
The downside of this approach is that the index query processing system needs to forward each query to every
storage location.

This can be addressed by replicating the entire index at each storage location,
but that would multiply the storage and memory footprint of the index.
In addition it would shift the requirement for cross data center communication to the write path;
Every update would need to be sent to every other data center.

We propose an alternative approach, based on partial index replication.
To achieve that, we introduce the Partial Index QPU class.
The goal of the Partial Index class is to implement secondary index in which only a subset of index entries are materialized,
essentially implementing a \textit{partial replica of a secondary index}.
The insight is that some index entries are accessed more frequently than others;
The Partial Index, similarly to a cache, materializes the most recently read index entries;
Additionally, it does not invalidate entries, but, similarly to an index, performs incremental updates to keep them up-to-date
with the corpus.
This has two benefits:
It bounds the memory footprint of the index, and reduces the cross data center communication needed by the index both in the read and the write path.
In the read path, queries that can be processed using materialized index entries do not need to be forwarded to
another storage location.
In the write path
Using the Remote Index QPU reduces the cross data center communication required for index processing by
replicating recently accessed index entries.

We use the Partial Index QPU to propose an improvement to the QPU architecture of Figure~\ref{fig:federated_index} (a).
We deploy Partial Index QPU on each storage location, and connected to the index sub-graph of the Partial index.
Each of these Partial Index QPUs functions as follows.
When initially deployed, it none of its entries materialized.
When it receives a query, its query processing function forwards the query to its downstream connection,
which in this QPU graph is an index location, as a persistent query.
It stores the received entries, incrementally updates them using the input stream of updates.
A difference with the Index QPU is therefore that the Partial Index establishes an input stream for each materialized
index entry, while the Index requires a single input stream.

\medskip
\noindent
While the partial index replication can potentially reduce the write path cross data center communication of an index replica,
it might still incur significant between storage locations,
in cases in which materialized index entries are updated frequently.
To address this issue,
we extend the Partial Index class with a mechanism for measuring the rate of updates to each index entry.
If the update rate for a certain entry crosses a threshold specified by the QPUs configuration,
then the QPU stops receiving updates for this entry and discards it.

The Partial Index QPU provides fine-grained placement of index entries.
Index entries that are queried more heavily are replicated, and thus placed closer to the clients,
while index entries that are update more heavily are placed closer to the corpus.
\todo{TODO: add paragraph discussing data transfer cost}

\medskip
\noindent
The QPU architectures described in this case demonstrate an important property of QPU-based query processing architectures:
A QPU graph can have a hierarchical structure,
in which lower layers are composed composed of multiple independent sub-graphs,
which are then connected by higher layers.
Th federated index architecture consists the partitioned index for each storage location (lower layer),
connected by Partition Manager QPUs (higher layer).
Moreover, the lower layer is transparent to the higher layer.
For example, the lower layer of the federated index architecture can used either the partitioning by term or
partitioning by document approach without affecting the system's functionality.

\subsection{Predicate-based indexing}

In this section, we extend the multi-cloud index case study by examining a use case inspired by one of Scality's customers.
We consider the example of a media organization that operates broadcasting and streaming services.
It stores media assets (video content, images) used by these services in an object storage system.
As described above, objects are tagged with metadata tags;
Applications use queries on metadata tags to provide recommendation and user search functionalities.
The organization operates in two geographic locations, and a separate storage platform is used for each location.

The metadata search requirement of this use case are as follows:
Applications on each geographic location need low latency metadata search on \textit{local} corpus (data stored on that geographic location),
as well as a \textit{subset} of corpus in the remote location.
This subset is specified by the application, and is expressed as predicates on metadata tags.
In other words, for an application that operates in $location_A$, there is an ``interest set'' in $location_B$:
a subset of the corpus $location_B$ that the application accesses frequently and needs low latency metadata search on.

\begin{figure}
  \centering
    \includegraphics[width=0.7\textwidth]{./figures/case_studies/predicate_based_index.pdf}
  \caption{QPU architecture for a predicate-based secondary index.}
  \label{fig:predicate_based_index}
\end{figure}

\medskip
\noindent
While the state-of-the-art warehousing approach of constructing a central index with metadata tags from all objects
on both locations can be applied to achieve these requirements,
that would potentially use significantly more resources that required,
if the interest set is small compared to the corpus.

Instead, we construct a secondary index that indexes only the interest set,
by filtering the corpus based on a given predicate before indexing.
Figure~\ref{fig:predicate_based_index} shows the QPU architecture used to achieve that.
We use a layer of Selection QPUs between the corpus (Corpus Driver QPUs) and the partitioned index.
The Selection QPUs apply a filter on the corpus based on the query that defines the interest set,
making only data items that satisfy it available to the index.
We refer to this architecture, as an \textit{predicate-based secondary index}.

A property of this architecture is that it is extensible:
Additional corpus subsets can be added to the filtered index by deploying additional Selection --- Index QPU subgraphs,
and connecting them through a Partition Manager QPU.

A federated index QPU architecture for this use case consists of two parts
follows the same logic as the one presented in Figures~\ref{fig:federated_index} (a) and (b),
with the addition that in one of the two storage locations we use a predicate-based index.


%%%%%%%%%%%%%%%%%%%%%%%%%%%%%%%%%%%%%%%%%%%%%%%%%%%%%%%%%%%%%%%%%%%%%%%%%%%%%%%%%%%%%%%%%%%%%%%%%%%%%%%%%%%%%%%%%%%%%%%%
% \section{Materialized views at the edge}
\section{Materialized view middleware}
\label{sec:lobsters}

In this case study, we examine an existing application that requires the use of pre-computed state,
and study how it can benefit by using a QPU-based architecture for query processing.

Lobsters \cite{lobste:rs} is a news aggregator web application.
In Lobsters, users post and comment on links (\textit{stories} in the Lobsters terminology).
In addition, users vote on stories and comments, and votes are used to rank stories.

Lobsters a is read-heavy application.
Traffic data for the production deployment of Lobster, provided by Lobsters' administrators \cite{lobste:stats} 88\% to 97\%
of the users' interactions with the application are operations tha perform reads to the application's backend.
These include viewing specific stories (55\%) and viewing the frontpage (30\%).

In applications such as Lobsters in which read performance is important, application developers often implement mechanisms to optimize it.
Lobsters, in addition to storing individual votes in a votes table, also stores per-story vote counts and story rankings as additional columns of other tables \cite{lobsters:schema}.
Pre-computing and storing vote counts and story rankings avoids the need re-compute them on every page load.
However, the application logic needs to explicitly update the pre-computed values every time a vote is casted.

Another approach that read-heavy applications employ to avoid expensive computation in read queries
is to use an in-memory key-value store, such Redis or memcached \cite{nishtala:memcachefacebook}
as a cache to speed up common-case queries.
The approach reduces load to the database,
as queries can be served from the cache when the underlying records are unchanged.
However, the application needs to explicitly invalidate and replace cache entries as database records change.
This requires complex application-side logic, and is error-prone.

\bigskip
\noindent
In this case study, we focus on a subset of Lobster's functionality that can de modeled as follows:

\begin{lstlisting}[
          language=SQL,
          showspaces=false,
          basicstyle=\ttfamily,
          commentstyle=\color{gray},
          rulecolor=\color{black},
          stringstyle=\color{mymauve},
          frame=L,
          xleftmargin=\parindent
        ]{}
TABLE users (id int, username text)
TABLE stories (id int, author_id int, title text, url text);
TABLE votes (user_id int, story_id int, vote int);
\end{lstlisting}

\noindent
We demonstrate a QPU architecture that expresses a materialized view which pre-computes the $vote\_count$ for each story.
The goal of this view is to facilitate the query that Lobsters executes for retrieving a requested story:

\begin{lstlisting}[
          language=SQL,
          showspaces=false,
          basicstyle=\ttfamily,
          commentstyle=\color{gray},
          rulecolor=\color{black},
          stringstyle=\color{mymauve},
          frame=L,
          xleftmargin=\parindent,
          commentstyle = \color{gray}
        ]
Q_story = SELECT id, author_id, title, url, vote_count
          FROM stories
          JOIN (
            SELECT story_id, SUM(vote) as vote_count
            FROM votes
            GROUP BY story_id
          ) view
          ON stories.id = view.story_id
\end{lstlisting}

\noindent
Figure~\ref{fig:lobsters_architecture_basic} shows the QPU architecture that implements this materialized view.
We pass $Q\_story$ as configuration to the Materialized view QPU.
Upon initialization, the Materialized view QPU sends the following as a downstream query to the Join QPU,
with the additional predicates:

\begin{lstlisting}[
          language=SQL,
          showspaces=false,
          basicstyle=\ttfamily,
          commentstyle=\color{gray},
          rulecolor=\color{black},
          stringstyle=\color{mymauve},
          frame=L,
          xleftmargin=\parindent
        ]
SNAPSHOT LATEST
INTERVAL FROM LATEST
\end{lstlisting}

\begin{figure}[t]
  \centering
    \includegraphics[scale=0.5]{./figures/case_studies/lobsters_architecture_basic.pdf}
  \caption{QPU architecture for the materialized view that integrates vote counts to stories.}
  \label{fig:lobsters_architecture_basic}
\end{figure}

The Join QPU in turn generates two downstream queries:

\begin{lstlisting}[
          language=SQL,
          showspaces=false,
          basicstyle=\ttfamily,
          commentstyle=\color{gray},
          rulecolor=\color{black},
          stringstyle=\color{mymauve},
          frame=L,
          xleftmargin=\parindent,
          commentstyle = \color{gray}
        ]
Q1 = SELECT id, author_id, title, url
     FROM stories
     SNAPSHOT LATEST
     INTERVAL FROM LATEST
\end{lstlisting}

\begin{lstlisting}[
          language=SQL,
          showspaces=false,
          basicstyle=\ttfamily,
          commentstyle=\color{gray},
          rulecolor=\color{black},
          stringstyle=\color{mymauve},
          frame=L,
          xleftmargin=\parindent,
          commentstyle = \color{gray}
        ]
Q2 = SELECT story_id, SUM(vote)
     FROM votes
     GROUP BY story_id
     SNAPSHOT LATEST
     INTERVAL FROM LATEST
\end{lstlisting}

\noindent
It sends $Q1$ to the stories Corpus Driver QPU and $Q2$ to the Sum QPU.
We note that it may send the two downstream queries to the corresponding Selection QPUs;
The two query plans are equivalent.

\smallskip
\noindent
When Join receives $Q1$, it initiates an output stream,
first sending a snapshot of all stories in the stories table,
and then sending an update for each write to the table.

\smallskip
\noindent
When Sum receives $Q2$, its query processing function sends a downstream query to the votes Corpus Driver QPU
in order to receive the most recent snapshot of the votes table and subscribe to subsequent writes to the table.
As a result, tt receives an input stream consisting of each record in the votes table snapshot,
and incrementally calculates the sum of the $vote$ attribute ($vote\_count$) for each distinct $story\_id$.
Its state is composed of $(story\_id$, $vote\_count)$ tuples.
When the Sum QPU completes processing the snapshot, it emits every computed $(story\_id$, $vote\_count)$ tuple
at its output stream.
In addition, the Sum QPU also receives an input record for each insert to the votes table.
Any update record received while still processing the snapshot, is treaded as a snapshot record;
It updates corresponding $vote\_count$, but does not result to an output record.
For each update received after the votes table snapshot has been processed,
the Sum QPU updates the $vote\_count$ for the corresponding $story\_id$,
and emits the updated $(story\_id$, $vote\_count)$ at its output stream.

\smallskip
\noindent
Join stores intermediate state for each of input stream.
When it receives an input record from one of the streams,
it matches it with the corresponding stored record for the other stream,
based on the $story\_id$ attribute.

In our implementation of this use case,
the Join QPU is merged with the Materialized View QPU so that the system does not need to maintain state
in order to perform the join operation, in addition to the materialized view state.
We present this in more detail in section \todo{add reference to imlementation part}.

\medskip
\noindent
In summary,
this QPU architecture provides the functionality of pre-computing story votes counts in order to avoid re-computation
when serving user read requests.
This is equivalent to the functionality already implemented in the Lobsters application.
However, using the QPU-based materialized view shifts the responsibility of updating the vote counts from the application logic,
to the query processing architecture, thus simplifying the application code.


\subsection{Partial materialization}

We expand the Lobsters case study by examining the operation of loading the application's frontpage.
The frontpage consists of the $K$ stories with the highest vote count ($K$ being a parameter).

The Lobsters implementation uses the following query to retrieve the information required for loading the frontpage:

\begin{lstlisting}[
          language=SQL,
          showspaces=false,
          basicstyle=\ttfamily,
          commentstyle=\color{gray},
          rulecolor=\color{black},
          stringstyle=\color{mymauve},
          frame=L,
          xleftmargin=\parindent
        ]
SELECT id, author, title, url, vote_count
FROM stories
JOIN (
  SELECT story_id, SUM(v.vote) as vote_count
	FROM votes
	GROUP BY story_id
) view
ON stories.id = view.story_id
ORDER BY vote_count DESC
LIMIT K
\end{lstlisting}

\begin{figure}[t]
  \centering
    \includegraphics[scale=0.4]{./figures/case_studies/lobsters_architecture_materialization.pdf}
  \caption{A vote triggers the materialization of an entry of the TopK materialized view.}
  \label{fig:lobsters_architecture_materialization}
\end{figure}

In this section, we show how the QPU architecture defined in the previous section can be extended to provide this
functionality.

To achieve that, we introduce an additional query processing unit class: the TopK Materialized View (TopK-MV) QPU.
The TopK-MV represents a materialized view which orders its entries based on a specified attribute.
In addition, it only materializes the $K$ entries with the highest (or lowest) values for that attribute.

We deploy a TopK-MV QPU in the Lobster application QPU architecture (Figure~\ref{fig:lobsters_architecture_materialization}),
and configure it as a materialized view with the same definition as the existing materialized view,
storying entries of the form $(story\_id$, $author\_id$, $title$, $url$, $vote\_count)$.
In addition, we configure TopK-MV QPU to order entries by $vote\_count$, and materialize the $K$ entries with the highest $vote\_count$.

The TopK-MV is connected to the Materialized view QPU and the Sum QPU.
Its initialization function sends a persistent query to the Sum QPU.
As a result, it first receives a snapshot containing the $vote\_count$ of every existing story in snapshot of the votes table;
It subsequently receives an update for each change to the $vote\_count$ of a story.
It stores the received snapshot as a list of $(vote\_count$, $story\_id)$ entries, ordered by $vote\_count$.
Then, for each of the $K$ entries with the highest $vote\_count$,
it sends a downstream query to the Materialized View QPU in order to retrieve the remaining attributes of the corresponding story.
Therefore, the TopK-MV \textit{materializes} a bounded number of entries, based on a given criterion.

Furthermore, the TopK-MV QPU continues receiving updates from the Sum QPU, and updating its
ordered list of $(vote\_count$, $story\_id)$ entries.
When the $vote\_count$ of a non-materialized entry becomes one of the top-K,
the unit's callback function discards the materialized entry with the lowest $vote\_count$,
and triggers the materialization of that entry.

The Top-K QPU uses the technique of \textit{partial materialization} in order to bound the size of pre-computed state.
This is a well known concept for materialized views in database systems \cite{zhou:partiallymaterialized, zhou:dynamicmaterialized},
and has been used by Noria \cite{gjengset:noria} in the context of data-flow systems.

\subsection{Placing materialized views at the edge}

% The ad-serving application's users are distributed worldwide, and therefore, the communication latency between user devices and the data center may be significant.
% Placing data geographically closer to end users is a common technique for reducing the large access latencies resulting from geo-distribution 

Query response time is crucial for user-facing read-heavy application, such as Lobsters.
As discussed in chapter~\ref{ch:background}, even small increases in user-perceived latency can result in significant drops in both web traffic and sales.
So far in this case study, we have examined how query response time can be decreased using pre-computed query results.
Another factor contributing to query response time is the communication latency between the application and the clients.
A client located in a different geographic region than the Lobsters application deployment might experience communication
latency in the order of hundreds of milliseconds.

A common solution to this latency problem is to place on servers geographically closer to clients using caches,
in order to avoid costly remote round-trips to data centers.
These servers, called edge nodes, are a crucial component in industry architectures.
For example, Google operates comparatively few data centers relative to edge nodes \cite{google:infra}.
In existing applications,
edge nodes are largely used for caching static data, such as images and video content, for example in content delivery network architectures.

\begin{figure}[t]
  \centering
    \includegraphics[scale=0.4]{./figures/case_studies/lobsters_architecture_edge.pdf}
  \caption{Distributing a QPU architecture between the data center and edge nodes.}
  \label{fig:lobsters_architecture_edge}
\end{figure}

\medskip
\noindent
Because of its modularity, the proposed query processing architecture can be used to place derived state, such as materialized views,
closer to client.
More specifically,
the query architecture designed for the Lobsters application (Figure~\ref{fig:lobsters_architecture_materialization}) can be
distributed among the data centers and edge nodes.
Considering a system architecture composed of a data center and multiple edge nodes,
we can place a TopK Materialized View QPU on each edge node, as shown in Figure~\ref{fig:lobsters_architecture_edge}.
Each TopK-MV QPU stored pre-computed state for the $K$ stories with the most votes.
It thus provides low latency access to the data required for loading the application's frontpage,
while maintaining a bounded memory footprint.
We connect each TopK-MV QPU to the complete Materialized View QPU and the SUM QPU placed in the data center,
In that way, TopK-MV QPUs can receive updates for updated vote counts,
and also request the materialize new stories as the reach the application frontpage.

We configure the connections between each TopK-MV QPU and the QPU architecture placed on the data center to be asynchronous,
as propagating each vote to the edge nodes synchronously adds a significant overhead to vote operations.
This choice makes a trade-off between write latency and the freshness of materialized views.
Propagating updates to materialized views asynchronously means that views are eventually consistent,
and might therefore return stale results.

In chapter~\ref{ch:evaluation}, we evaluate the effect of placing materialized view closer to clients on query response time and
and query result freshness.

%%%%%%%%%%%%%%%%%%%%%%%%%%%%%%%%%%%%%%%%%%%%%%%%%%%%%%%%%%%%%%%%%%%%%%%%%%%%%%%%%%%%%%%%%%%%%%%%%%%%%%%%%%%%%%%%%%%%%%%%

\bibliographystyle{plainnat}
\bibliography{refs}

% \chapter{Query processing as composition of query processing units}
% \label{ch:composition}
% Having presented the QPU characteristics, the principle of composing QPUs in a DAG, and the graph's computation model,
the goal of this chapter is to concretely.

\section{QPU query processing capabilities}

Each QPU is responsible/capable of processing queries in a specific space. These capabilities depend on:
\begin{itemize}
  \item The QPU's functionality (class).
  \item It's configuration.
  \item It's child connections in the graph.
\end{itemize}

QPUs maintain metadata (state) that describe their own capabilities, as well as their knowledge of the capabilities of
their child connections.

In this section we describe the format of the query processing capabilities state. We defer the description of how QPUs
populate this state for the following section.

\section{Distributed query processing protocol}
In this section we present the protocol that QPUs use to retrieve intermediate query responses for a given query by
sending sub-queries to child QPUs.

In detail, we present the algorithms for:
\begin{itemize}
  \item Determining given query a QPU can process locally and which it needs to forward to child connections.
  \item How a QPU uses its knowledge of the query processing capabilities of its child connections to generate and send
  sub-queries.
\end{itemize}

This protocol runs locally at each QPU and requires no central coordination.

\section{Configuration Dissemination}
In this section we describe how QPUs build/populate the query processing capabilities state.

Overview:
\begin{itemize}
  \item In the case of the index QPU, its configuration explicitly describes which queries can be processed locally
  (using the index).
  \item For other QPUs (e.g. filter, cache), their capabilities depend on the capabilities of their child connections.
  For example, if a cache QPU is connected to an index QPU, it inherits the query capabilities of that QPU.
\end{itemize}

Here, we also define the operation of combining the capabilities of more than one QPUs (a QPU uses that operation to
calculate its capabilities given its configuration and the capabilities of its children)

\subsection{Configuration propagation API}




\chapter{Proteus: Towards application-specific query processing systems}
\label{ch:proteus}
We have implemented the composable query processing architecture described in Chapter~\ref{ch:design_pattern} in the form
of a framework that facilitates the definition and deployment of QPU-based query processing systems.
We call the framework Proteus.
Proteus consists of:

\begin{itemize}
  \item A collection of implementations of QPU classes (Section~\ref{sec:proteus_qpu_architecture}).

  \item A service discovery mechanism that simplifies the configuration of a QPU architecture by allowing the QPU graph to self-organize
  with only local configuration input to each QPU instance (Section~\ref{sec:domain_dissemination}).

  \item An architecture description language that can be used to define the topology and configuration of QPU architectures
  (Section~\ref{sec:spec_language}).

  \item A mechanism for translating architecture descriptions to deployment plans (Section~\ref{sec:proteus_deployment}).
\end{itemize}

\noindent
In Section~\ref{sec:implementation}, we present the implementation details of Proteus.


\section{Query processing domain dissemination}
\label{sec:domain_dissemination}
The set of queries that a query processing unit can process constitutes its query processing \textit{domain}.
As presented in Section~\ref{sec:qpc_tree},
we encode the query processing domain using a tree data structure, called domain tree.

The domain of a query processing unit depends on its functionality (class), its configuration, and, in most cases,
on the domains of its downstream connections as well.
This is because QPU classes such as Join and Partition Manager process a given query by breaking it down to sub-queries,
sending these sub-queries to their downstream connections,
and then performing a computation over the returned results.
Essentially, a QPU of this type de-composes queries into simpler tasks, and delegates some of these tasks to their downstream connections.
Therefore, the set of queries that it can process depends on the types of tasks that its downstream connections can perform.

Because of that, a query processing unit requires the domain tree of each of its downstream connections for its operation.
More specifically, as presented in Section~\ref{sec:qpc_tree}, a QPU uses the downstream domain trees in order to:
(1) compute its own domain tree, and (2) generate downstream queries during query processing.
In this section, we describe the mechanism through which a query processing unit acquires the domain trees
of its downstream connections.

\subsection{Domain interface}
We extend the query processing unit specification with a mechanism that allows a QPU to request the domain trees of its connections.
For that, we define an additional interface, as follows:

\begin{displaymath}
  GetDomain() \rightarrow [(QPUClass, DomainTree)]
\end{displaymath}

An invocation of $GetDomain$ returns a stream that contains $DomainTree$ records, where $DomainTree$ is a serialization of a QPU's domain tree.
When a QPU receives an invocation of its domain interface, it establishes an output stream with the caller.
It then sends a $DomainTree$ record through the stream, representing its current domain tree.
Each time there is a change to the QPU's domain tree, it sends an additional $DomainTree$ record through the stream,
representing the updated domain tree.

Similarly to the query interface, $GetDomain$ requests follow the QPU graph topology:
An edge in the QPU graph from $QPU_A$ to $QPU_B$ indicates that $QPU_A$ can invoke the domain interface of $QPU_B$.

We define the domain interface response as a stream rather than an one-time response in order to enable propagation of
configuration, domain and topology changes through the QPU graph.
This is a first step towards enabling \textit{dynamic reconfiguration} of QPU architectures.
Examples of dynamic reconfiguration might include the creation new indexes or materialization of views at runtime.
We consider dynamic reconfiguration of QPU architecture out of the scope of this work.
It is, however, a direction for future work.

In order to support the domain interface, we extend the query processing unit specification with two additional methods:
a domain processor and a domain response processor.
These methods are analogous to the query interface methods.
The domain processor method is responsible for processing a GetDomain request,
and the domain response processor method is executed for each received domain response,
and is responsible for updating the QPU's local view accordingly.

\subsection{Query processing domain discovery mechanism}

The first task of the constructor of each query processing unit class is to initialize the QPU's domain tree.
For the Corpus Driver class this is performed using the input configuration (\S\ref{sec:qpc_tree}).
Every other QPU class, as discussed above, requires the domain trees of its downstream connections in order to compute its own domain tree.
Because of that, the constructor method of every QPU class except of the Corpus Driver starts by sending a $GetDomain$ request to each
of its downstream connections.
After receiving the corresponding responses, the constructor computes the QPU's domain tree, using the rules presented in Section~\ref{sec:qpc_tree}.

If a QPU receives a $GetDomain$ request while still this process is being performed, it creates the response stream,
but sends the response only after it has finished computing its domain tree.

We illustrate this process using the architecture shown in Figure~\ref{fig:domain_example_graph} as an example.
Figure~\ref{fig:domain_sequence_diag} depicts a sequence diagram that describes the message exchanges for domain discovery
in that QPU graph.
Upon initialization, the $Join$, $Selection_1$ and $Selection_2$ send a $GetDomain$ request to each of their downstream
connections,
$Corpus$ $Driver_1$ and $Corpus$ $Driver_2$ compute their query processing domains using their input configuration.
Once each $Corpus$ $Driver$ QPU computes its domain tree, it responds to the corresponding $GetDomain$ request.
When each $Selection$ QPU receives the $GetDomain$ response, it calculates its domain tree, and then it responds to the request from $Join$.
Finally, $Join$ receives the responses from $Selection_1$, $Selection_2$, and computes its domain tree.

The goal of the domain dissemination mechanism is to reduce the input configuration required by QPU instances,
in order to simplify the process of configuring and deploying a QPU architecture.
Thanks to this mechanism, the configuration of a query processing unit consists of ``local'' configuration
(configuration about that specific unit), as well as the endpoints of its downstream connections.
The QPU can then use the domain interface of its connections to obtain information about them.
For example, a Partition Manager connected to a number of index QPU instances requires only the endpoints of its
connections as configuration.

\begin{figure}
    \centering
    \begin{minipage}{.3\textwidth}
        \centering
        \includegraphics[scale=0.5]{./figures/design_pattern/qpu_graph_emergent_properties.pdf}
        \caption{An example QPU architecture.}
        \label{fig:domain_example_graph}
    \end{minipage}%
    \begin{minipage}{.7\textwidth}
        \centering
        \includegraphics[scale=0.4]{./figures/proteus/domain_sequence_diag.pdf}
        \caption{Sequence diagram for the domain discovery of the architecture in Figure~\ref{fig:domain_sequence_diag}.}
        \label{fig:domain_sequence_diag}
    \end{minipage}
\end{figure}


\section{Query processing unit service}
\label{sec:proteus_qpu_architecture}
Proteus provides an implementation of the query processing architecture presented in Chapter~\ref{ch:design_pattern}.
We have implemented the query processing unit component as a service, i.e. a daemon process.
In this section we describe the system design and architecture of this service.

The design of the QPU service is guided by the principles of the query processing unit model.
Instead of implementing a separate QPU service for each QPU class,
we design the query processing unit service as a \textit{polymorphic} service:
Different instances of the same service implement different QPU classes, based on their configuration.
To achieve that, we separate the components that are common to every QPU class,
such as the configuration and query processing domain component,
and the components that are class-specific.
For the class-specific components, we provide implementations for different classes in the form a library.
We ensure that components are separated and communicate through well-defined APIs.

Our goal with this design is to make the query processing service \textit{extensible}.
Implementing an additional QPU class consists of extending the QPU class library with component implementations
for the additional class.

\subsection{QPU service architecture}

\begin{figure}
  \centering
    \includegraphics[width=0.7\textwidth]{./figures/proteus/QPU_architecture.pdf}
  \caption{An overview of the QPU service architecture.}
  \label{fig:qpu_arch}
\end{figure}

\medskip
\noindent
Figure \ref{fig:qpu_arch} depicts the components of the query processing unit service in Proteus.

A configuration file is passed to the QPU service during initialization.
The \textbf{Configuration} component is responsible for parsing the configuration file into a set of configuration parameters.
It exposes an API that other components can use to retrieve these configuration parameters.
Configuration parameters include the QPU class to be provided by the service,
the endpoints of downstream connections, and class-specific configuration parameters.
Some examples of class-specific configuration parameters are:

\begin{itemize}
  \item The Corpus Driver class configuration specifies endpoints of storage tier components that the Corpus Driver communicates with to provide its functionality,
  the table it is responsible for, and, optionally, the table's schema.

  \item The Index class configuration, as described in Section~\ref{sec:cs_index_partitioning},
  specifies an attribute name and an interval of values for that attribute.

  \item The Aggregator class configuration specifies an aggregation function, and the aggregation and grouping attributes.
\end{itemize}

\medskip
\noindent
The \textbf{Query Processing Domain} component is responsible for storing and providing access to QPU's domain tree as well
as the domain tree of downstream connections.
The QPU's domain tree is initialized by the QPU class Constructor method.
The domain trees of the QPU's downstream connections are initialized and subsequently updated by the QPU class Domain Response Processor.
The Query Processing Domain component provides two interfaces:
\begin{itemize}
  \item An interface for providing a serialization of the domain tree to the QPU class Domain processor, and
  \item An interface for computing downstream queries based on a given query parse tree.
\end{itemize}

\bigskip
\noindent
The \textbf{State} component is responsible for the QPU service's internal state.
It provides a key-value interface which other components use for storing and retrieving state entries.
The State component is used by derived state QPU classes, such as the Index and Materialized View classes,
for storing their derived state structures.

We have defined a key-value interface for the State component,
and have implemented multiple versions of the component using different backend stores:
an in-memory implementation using an ordered map data structure,
an implementation that uses MariaDB \cite{mariadb:docs} as backend store,
and one that use AntidoteDB \cite{antidotedb:docs}.
The State component to be used by a QPU service instance is controlled by a configuration parameter.

\bigskip
\noindent
The \textbf{API Processor} is responsible for receiving and processing incoming requests for the two open interfaces of
the query processing unit:
the Query and the Domain interface.
It provides a Query Processor method which is responsible for processing query requests,
and a Domain Processor method, responsible for domain requests.

When the Query Processor method is executed for query request received by the API processor,
it first parses the given query to a query parse tree \ref{sec:query_parse_tree} using the Query Language Parser.
If the query parse tree is successfully created, the Query Processor initiates a response stream,
and invokes the Class Query Processor of the QPU class specified in the configuration.
After executing the query, the Class Query Processor sends query result records back to the API Processor,
which emits them at the response stream.

Similarly, when the API Processor receives a GetDomain request,
it executes the Domain Processor, which passes the request to instance of the Class Domain Processor,
and emits each received response as a stream record.

The API Processor can process multiple requests concurrently.
For each received request, it creates an output stream,
and spawns the corresponding Processor method in a new thread.

\bigskip
\noindent
The \textbf{QPU Class} component is responsible for providing the methods defined by the query processing
unit specification (Sections~\ref{sec:QPU_model} and~\ref{sec:domain_dissemination}).
It is implemented as a library that provides QPU class method implementations.
Each class in the library provides five method definitions:
a Constructor method, a Class Query Processor method, a Domain Processor method, a Query Response Processor method,
and a Domain Response Processor method.
In our prototype, we have implemented the following QPU classes:
\begin{itemize}
  \item \textbf{Selection, Aggregation, Secondary Index, Partition Manager, Join, and Materialized view} as defined in previous chapters.

  As an optimization, we have integrated the functionality of the Selection class to other classes,
  such as the Corpus Driver, Secondary Index and Cache.
  This is because the Selections QPUs are often connected to these classes in QPU architectures.
  By integrating the selection functionality with these classes, we simplify QPU architectures
  and reduce the overhead of message serialization and de-serialization.

  Furthermore, in our current implementation, a Secondary index QPU service is responsible for a single attribute in a single
  continuous value interval, specified by its configuration.
  A multi-attribute query is decomposed into a conjunction of multiple single attribute queries.

  \item \textbf{Cache}.
  We have implemented a cache QPU class that stores query responses in an in-memory ordered map data structure.
  It uses a Least Recently Used eviction policy.
  Moreover, we defined the cache size in terms of \textit{query result records}.
  That is, a query result that contains $N$ data items occupies $N$ units of cache size.

  Moreover, we have implemented a time-based and an notification-based invalidation policy.
  In the time-based policy, cache entries are invalidated after an amount of time specified by the QPU's configuration.
  In the notification-based policy, the cache QPU performs, for each query,
  a downstream persistent query to subscribe to notifications for updates that modify the query result;
  When a query result changes, the cache QPU invalidates the corresponding cache entry.
  The invalidation policy is controlled by a configuration parameter.

  \item \textbf{Aggregator.}
  We have implemented a generic Aggregator Class that divides an input set of data items into groups based on a grouping attribute,
  and applies a function over an aggregation attribute to the data item of each group.
  The grouping and aggregation attributes as well as the aggregation function are configuration parameters.
  An aggregation operation fails if any data item in the input does not have a value for  the grouping or the aggregation attribute.
  Our current implementation provides the sum and count functions.

  \item \textbf{Corpus Driver.}
  We have implemented three versions of the Corpus Driver class, corresponding to different storage systems.:
  a relational database (MySQL \cite{mysql:docs}), a key-value store (AntidoteDB \cite{antidotedb:docs}),
  and an object storage system (Scality's CloudServer \cite{cloudserver:github}).
  We present more details on the implementation of each Corpus Driver in Section~\ref{sec:implementation}.

  \item \textbf{Network.}
  We have defined and implemented an additional class, called Network.
  A Network QPU has a single downstream connection.
  It forwards every received query request to that connection, and then forwards the input stream as its output stream.
  Moreover, it can be configured to delay, re-order or drop stream records, based on a specified distribution.
  The goal of this class is to simulate various network conditions for testing purposes.

  % \item \todo{TODO: Query Router}

\end{itemize}

The QPU class library is extensible;
Additional classes can be implemented by providing the corresponding method definitions.

\bigskip
\noindent
The \textbf{Downstream client} component is responsible for sending downstream requests to other QPU services.
It exposes an interface other components can use to submit query and domain requests.
The Downstream clients sends these requests to the corresponding connections,
and for each received response records it invokes the corresponding Class Response Processor method.

Moreover, the Downstream client component is responsible for delivering the records of a each stream exactly once and in-order
to the QPU Class.
To achieve this, the API processor assigns a sequence number to each emitted stream record.
The Downstream client uses these stream numbers to determine if records have been lost, already received or re-ordered.
Moreover, the Downstream client can request from downstream connections to re-transmit records that it has missed,
using control messages in the stream.
The API Processor in turn stores stream records after sending them in a buffer of configurable length.


\section{Architecture specification language}
\label{sec:spec_language}
In this section, we present a simple architecture description language for describing QPU-based query processing architectures.

The goal of a an architecture description is to describe a QPU graph,
including the class and configuration of the QPUs at its vertices and the connections among them,
as well as the placement of the graph vertices across system nodes.

An architecture description comprises a series of placement context assignments,
a series of QPU instance declarations, and a series of declarations of connections between QPU instances.

A placement context is denotes a node or collection of nodes.
It has the following syntax:

\begin{lstlisting}[
  language=qpulang,
  caption={Syntax used for placement context assignment.},
]
PlacementCtx([endpoint], PlacementCtxName)
\end{lstlisting}

\noindent
where $[endpoint]$ is a collection of system nodes, expressed as hostnames, or IP addresses,
and $PlacementCtxName$ is the name of the context assigned to these nodes.
For example, the description:

\begin{lstlisting}[
  language=qpulang,
  caption={Assigning placement contexts to four nodes.},
]
PlacementCtx([10.200.4.56], "node_1")
PlacementCtx([10.200.3.45], "node_2")
PlacementCtx([10.200.4.56, 10.200.3.45], "dc_eu")
\end{lstlisting}

\noindent
assigns the node with address 10.200.4.56 to the context ``node\_1'' ,
the the node with address 10.200.3.45  to context ``node\_2'',
and both nodes to the context ``dc\_eu''.

\medskip
\noindent
A QPU context declaration has the following syntax:

\begin{lstlisting}[
  language=qpulang,
  caption={Syntax used for specifying the configuration of a query processing unit.},
]
Configuration = {
  config_parameter_1: value_1,
  config_parameter_2: value_2,
  ...
}

qpu_object = <QPU_class>(Configuration)
\end{lstlisting}

\noindent
where $<QPU\_class>$ denotes one of the classes available in the QPU library.
The configuration specifies the configuration parameters passed to the QPU during initialization.

\medskip
\noindent
Specifying a connection between QPUs has the following syntax:

\begin{lstlisting}[
  language=qpulang,
  caption={Syntax used for specifying a connection between two query processing units.},
]
qpu_object_1.connectTo(qpu_object_2)
\end{lstlisting}

\noindent
This defines that $qpu\_object\_1$ has a downstream connection to $qpu\_object\_2$.

\medskip
\noindent
Finally, the following syntax can be used to describe the placement of QPUs across system nodes:

\begin{lstlisting}[
  language=qpulang,
  caption={Syntax used for specifying the placement of a query processing unit},
]
qpu_object.place([PlacementCtxName])
\end{lstlisting}

\noindent
The $place$ method assigns a set of \textit{placement constraint} to query processing unit.
The QPU can be placed on any node that satisfies the constraint.
For example, based on the description:

\begin{lstlisting}[
  language=qpulang,
  caption={Creating two query processing units, and specyfing their placement on system nodes.},
]
PlacementCtx([10.200.4.56], "node_1")
PlacementCtx([10.200.4.56, 10.200.3.45], "dc_eu")

q1 = Selection(config_selection)
q2 = Cache(config_cache)

q1.place("node_1")
q2.place("dc_eu")
\end{lstlisting}

\noindent
$q1$ is placed on the node with address 10.200.4.56,
while $q2$ can be placed on either of the two nodes.

\bigskip
\noindent

A limitation of this configuration language is that it does not explicitly express the placement of a QPU graph
relative to the corpus and the client,
but rather it expresses placement on system nodes.


\section{Query processing system deployment}
\label{sec:proteus_deployment}

We have designed a Deployment Generator component in Proteus for translating QPU architectures defined using the configuration
language presented in the previous section into deployment plans.

Each QPU service runs in a Docker \cite{docker} container.
The Deployment Generator translates an architecture description into (1) a Docker Swarm stack file
and (2) a configuration file for each QPU service.
The architecture is then deployed using Docker Swarm \cite{docker:swarm}

Docker Swarm uses Compose files \cite{docker:composefile} as deployment specification files.
A Compose file is a YAML \cite{yaml} file defines a set of services, a service being defined as a set of replicas of container that share the same image.
The Deployment Generator creates a Compose file that defines a service for each QPU instance defined in the architecture description.
A configuration file is passed to each service, using a shared volume.
Moreover, the Deployment Generator specifies a common network that all QPU services are connected to.

In order to generate the configuration file to be passed to each QPU service, the Deployment Generator translates the configuration parameters
of each QPU instance in the architecture specification to a TOML \cite{toml} file.
Moreover, it adds to the configuration parameters the QPU instance's downstream connections as defined in the architecture specification.

% \section{Fault tolerance}
% Describe the mechanisms provided by Proteus for tolerating faults during query
% processing.
% \begin{itemize}
%   \item Lost or re-ordered stream records.
%   \item Mid-query QPU crash.
%   \item Partitions.
% \end{itemize}

% \section{Implementation}
% Report on the implementation of different components of Proteus, including:
% \begin{itemize}
%   \item The framework used for communication between QPUs.
%   \item How streams are implemented.
%   \item ...
% \end{itemize}

\section{Implementation}
\label{sec:implementation}

The prototype implementation of Proteus consists of 13k lines of Go (v1.14).
The source code is available at \url{https://github.com/dvasilas/proteus}.

% It can operate on a single server or across a cluster of servers.

\bigskip
\noindent
\textbf{Interface.}
Applications interface with Proteus through a protocol buffer \cite{protobuf:docs} interface over gRPC (v1.31) \cite{grpc:docs}
gRPC is an remote procedural call (RPC) framework.
Using gPRC, a client process can directly call a method on a server process, on a different machine, as if it were a local object.
The workflow for using gPRC consists of defining a service, specifying its interface; methods that can be called remotely
and their parameters and return types.
gRPC uses protocol buffers as its interface description language.
We selected gRPC due to its support for streaming and bidirectional streaming RPCs,
and because it provides compiler plugins that can generate client- and server-side code based on a service definition.

Each query processing unit includes a gRPC server, as well as a gRPC client for communicating with other QPUs.
Client applications also need to implement gRPC clients.

\medskip
\noindent
We have additionally implemented two clients libraries, in Go and Java.
In the Go driver library, the operation of opening a connection to QPU actually opens and maintains a pool of connections
to that QPU.
Performing a query operations consists of requesting an idle connection, potentially opening a new connection if no idle
connections are available, or, if the maximum number of connections has been reached,  waiting until a connection becomes
available, performing the operation, and returning the connection to the pool.

\bigskip
\noindent
\textbf{Storage.}
Overall, query processing units maintain state either in memory, or persistently in AntidoteDB \cite{antidotedb:docs} or
MariaDB (v.10.3.27) \cite{mariadb:docs}.
However, currently not all QPU implementations support all three state backends.
For example, our index QPU implementation supports both in memory and AntidoteDB,
while the Aggregator and join QPU implementations support only MariaDB.
For QPU implementations that support more that one backends,
the backend by a QPU instance is specified by a configuration parameter.

\bigskip
\noindent
\textbf{Subscribe API (see models chapter for terminology).}
As described in Section~\ref{sec:storage_tier_api}, the data storage tier exposes a logical Subscribe API that can be
used for subscribing to notifications for data items updates.
We have implemented the Subscribe API in three data stores: MySQL/MariaDB, CloudServer, and AntidoteDB \cite{scality:cloudserver},
a REST server that compatible with the AWS S3 API, implemented by Scality.
As a results, Proteus currently can be used with a data storage tier provided by one of these data stores.

We note that MySQL/MariaDB and AntidoteDB can be used both as a QPU state backend and as a storage tier.
These two roles are independent:
A QPU graph may be used to provide materialized view on top of AntidoteDB,
and independent instances of AntidoteDB also be used for maintaining the state of QPUs in the graph;

\textbf{MySQL/MariaDB:}
We implement the Subscribe API using a user-defined function \cite{mariadb:udfs}, executed in response to a trigger \cite{mariadb:triggers}.
One the same machine as the MySQL instance, we run an executable, termed notification server,
that 1) implements a gRPC server, and 2) listens for socket connections.
For each MySQL table associated with Proteus, we install triggers that call a user-defined function for each modification to
the table.
For each record that is inserted or updated, the user-defined function encodes and sends a message to the notification
server through a socket connection.
corpus driver QPUs can subscribe to this update stream using the notification server's gRPC service.
The notification server is schema-agnostic; however, making a table available to Proteus requires installing the corresponding
triggers for that invoke the user-defined function.

\textbf{CloudServer}
CloudServer does not provide trigger or user-defined function mechanisms analogous to MySQL.
We have thus modified CloudServer adding a notification service provided by a gRPC server.
Our implementation consists of 300 lines of Node.js.
The source code is available at \url{https://github.com/dvasilas/cloudserver/tree/proto/proteus}.

\textbf{AntidoteDB}
Similarly to CloudServer, we have modified AntidoteDB adding a simple notification service that can be used by corpus driver QPUs.
The source code is available at \url{https://github.com/dvasilas/antidote/tree/proteus/log_propagation}.

\bigskip
\noindent
\textbf{Operator implementation.}
The implementation of the core QPU functionalities,
including relational operators (e.g. join, aggregation) and query processing data structures (e.g. secondary indexes,
materialized views) are fairly straightforward.
This is because update operations work in an incremental fashion,
processing records as they arrive.


In addition, we note that, despite the fact that the $Query()$ API allows for queries with fine-grained timestamps.
most QPU implementations currently support three more coarse query modes:
snapshot queries, persistent (or subscribe queries), and a combination of the two.
Snapshot queries are executed on the most recent snapshot, in a best-effort fashion,
and submitting a query request for an earlier snapshot returns an error.

A notable exception is the index QPU, in which we have implemented a versioned secondary index.
In the versioned secondary index,
each update creates a new version of the corresponding posting list.
The index maintains a configurable number of versions for each data item;
old versions are deleted to reduce memory/storage resource consumption.
The index QPU maintains its state either in memory or in AntidoteDB.

\bigskip
\noindent
\textbf{Multi-threading.}
Each QPU maintains two thread pools: one for executing query requests and one for processing input stream records.
Query requests and processing of input records are submitted as tasks to be executed by available threads.
Our benchmarking showed that using worker pools results in improved performance and stability,
especially in the implementations with MySQL backend,
We attribute this to having a bounded number of threads performing database transactions,
since our previous implementation was creating a new thread for each incoming request.

\bigskip
\noindent
\textbf{Packaging and deployment.}

As mentioned above, the query processing unit service is packaged as a Docker container.
This container is common for all QPU classes:
the QPU's class configuration is specified by a configuration file passed to the service at initialization,
Deploying a QPU service consists of deploying the QPU container
and providing the configuration file as a shared volume.

We use Docker Swarm for deploying a QPU graph:
A Docker compose file specifies the QPU services to be deployed, the configuration file to be passed to each service,
and the placement of each service.


\chapter{Evaluation}
\label{ch:evaluation}
In this chapter, we aim to validate our analysis of the trade-offs involved in query processing state placement decisions,
and to evaluate the effectiveness of our design and prototype implementation in providing an effective mechanism to applications
for navigating these trade-offs.

This evaluation consists of two parts.
The first part (\S\ref{sec:eval_1}) is based on the case study of a read-heavy web application, presented in Section~\ref{sec:lobsters}.
It focused on the placement of a materialized view,
and examines two placement options: in the data center, close to the corpus, or at the edge close to the clients.
The aim of the first part is to answer the following questions:

\begin{itemize}

  \item What is the overhead of materializing state in Proteus, directly over the data storage tier?
  (Section~\ref{sec:eval_query_processing_perf})

  \item What improvements in query processing performance can be achieved by placing materialized state close to clients?
  (Section~\ref{sec:eval_query_processing_perf})

  \item What are the penalties in freshness incurred when placing materialized state away from the data storage tier
  (Section~\ref{sec:eval_freshness})?
  How is freshness affected by load (\S\ref{sec:eval_freshness_throughput}) and round-trip time between
  sites (\S\ref{sec:eval_freshness_rtt})?

  \item What additional data transfer costs result from materialized state placed close to clients receiving updates?
  (Section~\ref{sec:eval_data_transfer})

\end{itemize}

The second part (\S\ref{sec:eval_2}) of the evaluation is based on the case study of federated query processing on a multi-cloud
corpus, presented in Section~\ref{sec:zenko}.
In this part we examine three partitioning and placement configurations for a multi-cloud secondary index.
We compare the three configurations under different types of workloads, and evaluate three metrics:
query processing performance, freshness and data transfer costs.

The aim of the second part is to (1) demonstrate the expressiveness of the QPU approach,
by deploying and comparing three query engine configurations across 3 data centers without any changes to the client application and the storage tier,
and (2) validate that by controlling the configuration of QPU-based query engines,
applications can navigate the trade-offs of geo-distributed query processing and optimize for different criteria according to their needs.


\section{Placing materialized views at the edge}
\label{sec:eval_1}

\subsection{Experimental scenario}
\label{sec:eval_scenario}

The evaluation in this section is based on the case study presented in Section~\ref{sec:lobsters},
which describes the Lobsters \cite{lobste:rs} web application.
We choose this application for the following reasons:
\begin{itemize}
\item It is characterized by a query-heavy workload that requires the materialization of derived state,
and derived state is updated by a stream of small updates.
This make Lobsters suitable for evaluating the efficacy of our design and prototype implementation in
navigating the trade-offs of query processing state placement,
by examining the effect of different placement schemes in query performance and query result freshness.

\item It open-source \cite{lobsters:source},
allowing us to examine the application's interaction with the database in which it stores its state,
and statistics about the application's data usage patterns are available \cite{lobste:stats}.

\item It resembles a class of popular large-scale web applications, such as Reddit and Hacker News.
\end{itemize}

\bigskip
\noindent
In Lobsters, users post, comment, and vote on ``stories''.
Each story is associated with a ``hotness'' value that indicates how popular it is.
Stories are ranked by hotness;
the stories with the highest hotness value appear on the front page.
The hotness value of a story depends on parameters such as the number of votes for the story,
the number of comments, and the hotness of those comments.
Various operations, such as voting or commenting on a story, modify the hotness value.
Computing the hotness value when it is queried would impose a prohibitive delay on queries
% It is prohibitively expensive \cite{gjengset:noria} to compute the hotness value of stories during queries.
In particular, serving the Lobsters' front page requires computing the hotness of every story in order to rank them.
That is why the Lobsters application adds an additional column to the $stories$ table which stores a pre-computed
hotness value for each story.
The application updates the value of the hotness column when operations are performed,
such as upvoting or downvoting a story, or adding a comment to a story.

For this evaluation, we consider a version of the Lobsters application, that consists of two operations:
voting for stories, and requesting the front page.
We choose this simplified model of the application because it gives us better control over the properties of the workload
for the purposes of this evaluation,
while also capturing the aspects of the Lobsters application that make is suitable for this evaluation.

In particular, we consider the following database schema:

\begin{lstlisting}[caption={Simplified Lobsters schema used in this evaluation.}]
TABLE users (id bigint, username varchar(50))
TABLE stories (id bigint, user_id bigint, title varchar(150), description mediumtext, short_id varchar(6));
TABLE votes (id bigint, user_id bigint, story_id bigint, vote tinyint);
\end{lstlisting}

The front page is a listing of the 25 most highly ranked stories, including their title, author, and vote count.
In the statistics provided by the Lobsters administrators, the front page operation constitutes 30.1\% of client requests,
and voting on stories constitutes 0.5\% of client requests.
The workload used for this evaluation consists of 95\% front page operations, and 5\% voting operations, unless otherwise specified.

\begin{table}[H]
\centering
\begin{tabular}{|c||c|c|c|c|c|c|}
\hline
Number of votes & [0-10) & [10-20) & [20-30) & [30-40) & [40-50) & [50-60) \\
\hline
\% of stories & 41.1 & 40.3 & 11.3 & 4.2 & 1.6 & 1.3 \\
\hline
\end{tabular}
\caption{Distribution of votes to stories in the Lobsters statistics \cite{lobste:stats}.}
\label{tab:votes_per_story}
\end{table}

According to the available data \cite{lobste:stats}, most stories (81.4\%) receive between 0 and 20 votes,
while about 7\% receive 30 votes or more.
Our experiments show that when votes follow this distribution,
very few votes are performed on the stories on the front page to have meaningful impact on query result freshness.
To address that, we use a more skewed distribution:
We configure 60\% of votes to target the 25 stories in the front page, and 40\% to follow the distribution shown in Table~\ref{tab:votes_per_story}.

\begin{figure}[H]
  \centering
    \includegraphics[scale=0.5]{./figures/evaluation/lobsters_architecture_eval.pdf}
  \caption{QPU graph used for this evaluation. The Materialized view QPU computes the vote count for each story,
  and joins it with the corresponding record from the $stories$ stable.
  The ``GET /'' request indicates the front page operation, and the ``POST /stories/X/upvote'' request indicates the operation of voting up story X.}
  \label{fig:eval_lobsters_qpu_arch}
\end{figure}

\bigskip
\noindent
The evaluation uses the QPU graph depicted in Figure~\ref{fig:eval_lobsters_qpu_arch} to maintain a materialized view that
pre-computes the vote count of each count and joins it to the corresponding record of the $stories$ table.
The materialized view is defined by the following query:

\begin{lstlisting}[caption={Definition of the materialized view maintained by the QPU graph shown in Figure~\ref{fig:eval_lobsters_qpu_arch}.}]
SELECT id, author_id, title, url, vote_count
FROM stories
JOIN (
  SELECT story_id, SUM(vote) as vote_count
  FROM votes
  GROUP BY story_id
) view
ON stories.id = view.story_id
\end{lstlisting}

This simplifies both the vote and front page operations compared to the baseline Lobsters implementation:
the vote operation does not need to explicitly update the vote count of the given story, as this
is performed by the QPU graph,
and the front page operation can be served from the materialized view.

In the actual QPU graph deployed for the experiments, each Selection QPU is merged with its downstream connection,
in a single unit that performs both functionalities.
In addition, the Materialized View is merged with the Join QPU.

The Materialized view QPU implementation used in these experiments stores its state in a MariaDB instance.
This is a separate instance from the one that is used the Lobsters database.
It is deployed alongside and only accessible by the Materialized view QPU.
The choice of using the same database both as the baseline and for storing the materialized view is aimed at eliminating
the effect of the database performance from the evaluation results,
providing a comparison that isolates the effects of placement.

Finally, we create an index on the table that implements the materialized view in order to support efficient retrieval of
the stories with the highest vote count.

\bigskip
\noindent
We consider a system topology consisting of two geographically distant sites:
The Lobsters application is deployed on one site, called server site,
and clients are located on another site, called client site.
Round-trip time between these two sites is 80 ms.
This corresponds to a scenario in which the Lobsters web application is deployed in a data centers in North America,
and clients are located in Europe \cite{pbailis:hats}, or vice versa.

Our experiments  make use of the ability to place QPUs strategically in the system topology.
We use two placement schemes:
One in which the Materialized view QPU is placed at the server site, and one in which it is placed at
the client site.
We evaluate the effects and trade-offs of materialized view placement by comparing these two placement schemes.


\subsection{Experimental Setup}
\label{sec:eval_setup}

While the actual Lobsters Ruby-on-Rails application is open-source,
we use a simplified version of it in order to isolate its interaction with the database.
We implement an adapter that translates front page and vote requests to the queries that the real
Lobsters application would issue, and issues those queries either to the Lobsters database (MariaDB),
or MariaDB and Proteus, depending on the experiment configuration.
This allows us to isolate the interaction between the application and the database, which is the focus of this work,
and remove other tasks that the Lobsters application performs, which quickly become a bottleneck.

We implement this adapter as a server-side component:
it is deployed along with the database on the server site, and plays the role of a simplified web server:
It exposes a gRPC endpoint, similar to the QPU gRPC server, and clients issue operations to it as RPC requests.
We choose this configuration because our initial experiments showed that issuing a large number of concurrent transactions
to MariaDB under a 80 ms client-server round-trip time results in errors both in the database,
and the Go MySQL library.

\bigskip
\noindent
\textbf{Workload generation.}
We implement a workload generator \cite{lobsters:bench} that is responsible for issuing
requests to the Lobsters adapter.
The workload generator uses an open-loop model \cite{schroeder:cautionarytale}:
it creates requests based on a target load (requests submitted per second) value;
each request is executed by a separate thread (creating and destroying threads are low-cost operations because threads are
implemented as Goroutines).

In addition, the workload generator measures throughput and response time.
We define response time as the delay that a client application experiences between issuing a request and receiving the
corresponding response.
To capture the variance of response time, we use a histogram data structure provided by the Go implementation of gRPC \cite{grpcgo:histogram}
that accumulates values in a histogram with exponentially increasing bucket sizes, in order to compute response time percentiles.

Our experiments show that while the open-loop design is effective at generating the target load
both under low and high client-server round-trip times (in closed-loop designs, high round-trip times result in lower offered load),
it results in a positive feedback loop effect above a certain load threshold:
When the system starts not being able to handle the offered load, response time increases.
However, the workload generator keeps generating requests, and, because of the increased response time,
more requests are ongoing concurrently.
This puts more load in the gRPC server, increasing the response time even more.
Even a small initial increase in response time triggers this feedback loop,
which eventually increases response time more and more during the duration of the benchmark.
To address this problem, we extended the workload generator with a mechanism that limits the overall number of requests
(and thus threads) that can be ongoing at a given point in time, to a configuration-specified bound.
When the bound is reached, additional requests need to wait for ongoing requests to be completed.

The concurrency bound mechanism is effective in avoiding the positive feedback loop effect.
However, it also means that when running a benchmark for a given target load we need to specify a bound in the number of
concurrent threads that is sufficient for reaching the target load.
If the bound is too low, then the workload generator cannot generate enough load, but and but response time does not
increase because the delay caused by waiting for available threads is outside of the response time measurement.

\bigskip
\noindent
\textbf{Freshness.} The materialized state maintained by the Materialized view QPU updated asynchronously.
As a result, queries served from the materialized view might reflect state that is stale relative to the database state.
The staleness of the materialized state is impacted by its placement:
Placing the Materialized view QPU at the client side entails a minimum 40 ms communication latency to the materialized view
(80 ms round-trip time between sites).
Throughout the rest of this chapter, we consider the terms freshness and staleness equivalent and use them interchangeably.

One of the aims of this evaluation is to quantify how stale query results become.
To achieve that, we measure the staleness of the results returned by the front page operation,
using the following metrics:
\begin{itemize}
  \item \textbf{Update latency:} The delay between a vote being committed in the database, and the materialized view being
  updated with the new vote count.
  \item \textbf{Returned version:} The difference, in number of versions, between the version of a story record returned by a query,
  and the version that would have been read by querying the Lobsters database instead of the materialized view.
\end{itemize}

We collect these metrics as follows.
For each vote operation, the Lobsters database logs the timestamp at which the corresponding transaction commits;
the database then includes the commit timestamp the update record that it publishes to the QPU graph.
When the update record reaches the Materialized view QPU, the QPU stores the commit timestamp in an
``update log'' table.
Moreover, the Materialized view QPU logs the timestamp of each view update in the update log,
and the timestamp at the start of each query, in a ``query log''.

At the end of a benchmark run, the Materialized view QPU performs a post mortem analysis:
The update latency for each vote is computed by subtracting the update timestamp from the commit timestamp.
The returned version for each front page story is computed by comparing the update and query logs.

A limitation of this mechanism is that it requires comparing timestamps taken on different servers.
To address that, in the benchmarks in which we take freshness measurements,
we deploy all system components (Lobsters database, QPU graph, workload generator) as containers,
on a singly physical machine.
Because they share a single OS kernel, all timestamps used for computing freshness metrics are based
on the host operating system's clock, and thus can be meaningfully compared.
We use the Linux tc utility \cite{tc} to simulate the 80 ms round-trip time between sites,
despite all containers being deployed on a single server.

\bigskip
\noindent
\textbf{Hardware.}
Experiments were run on a cluster provided by the Laboratoire d'Informatique de Paris 6 (LIP6).
Each server consists of 2 Intel Xeon E5645 CPUs, each with 6 cores, 64 GB RAM, an 128 GB SSD disk, and a 4 TB HDD disk.

\bigskip
\noindent
\textbf{Configuration.}
The average ping latency between machines in the cluster is less than 1ms.
We simulate the two geographically distant sites by using the Linux tc utility \cite{tc} to add delay to outgoing packets.

For response time measurements, the Lobsters MariaDB instance and the QPUs, except the Materialized view QPU are deployed
on a single server on the server site, and the workload generator is on a server in the client site.
The Materialized view QPU is deployed on a separate dedicated server, either on the application or the client site,
according on the placement scheme being tested.
As described above, for the freshness measurements, all components are deployed on a single server.

Experiments run for 5 minutes unless otherwise specified, and we start taking measurements after an initial
``warmup period'' of 30 seconds.
Repeated runs have shown that results are stable and consistent across runs.


\subsection{Query processing performance}
\label{sec:eval_query_processing_perf}

\begin{figure}[H]
        \centering
        \begin{subfigure}[b]{0.24\textwidth}
            \centering
            \includegraphics[width=\textwidth]{./figures/evaluation/evaluation_deployments_baseline_remote.pdf}
            \caption{Baseline/remote}
            \label{fig:deployments_baseline_remote}
        \end{subfigure}
        \hfill
        \begin{subfigure}[b]{0.24\textwidth}
            \centering
            \includegraphics[width=\textwidth]{./figures/evaluation/evaluation_deployments_proteus_remote.pdf}
            \caption{Proteus/remote}
            \label{fig:deployments_proteus_remote}
        \end{subfigure}
        \hfill
        \begin{subfigure}[b]{0.24\textwidth}
            \centering
            \includegraphics[width=\textwidth]{./figures/evaluation/evaluation_deployments_proteus_local.pdf}
            \caption{Proteus/local}
            \label{fig:deployments_proteus_local}
        \end{subfigure}
        \hfill
        \begin{subfigure}[b]{0.24\textwidth}
            \centering
            \includegraphics[width=\textwidth]{./figures/evaluation/evaluation_deployments_internal.pdf}
            \caption{Materialized-view-internal}
            \label{fig:deployments_mv_internal}
        \end{subfigure}
        \caption{Deployments used in this evaluation.}
        \label{fig:eval_deployments}
    \end{figure}

We compare three deployments that differ in how they store and calculate the per-story vote count,
and how they distribute computations and state across the nodes of the system (Figure~\ref{fig:eval_deployments})
\textbf{Baseline/remote} (Fig.~\ref{fig:deployments_baseline_remote}) is equivalent to the real Lobsters application:
it pre-computes and stores vote counts in a column of the Lobsters $stories$ table.
This serves as the baseline approach.
\textbf{Proteus/remote} (Fig.~\ref{fig:deployments_proteus_remote}) consists of the QPU graph shown in Figure~\ref{fig:eval_lobsters_qpu_arch},
deployed on the server site.
In \textbf{Proteus/local} (Fig.~\ref{fig:deployments_proteus_local}),the Materialized view QPU is deployed on the client site.
This is intended to evaluate the effect of placing the materialized view close the client.
We also compare with a deployment (\textbf{Materialized-view-internal}, Fig~\ref{fig:deployments_mv_internal}) in which the workload generator
directly issues request to the Materialized view QPU's state, bypassing the QPU's gRPC server
(the workload generator and materialized view are co-located on the same server).
This aims at providing an indication of the best performance the Materialized view QPU can achieve,
without taking into account its gRPC server (which we have identified as a performance bottleneck).

\begin{figure}[H]
\centering
  \includegraphics[width=0.7\textwidth]{./figures/evaluation/responseTime.png}
  \caption{Throughput vs 95th percentile query response time.}
  \label{fig:responseTime}
\end{figure}

Figure~\ref{fig:responseTime} shows throughput--query response time plots of these deployments.
The ideal throughput--response time curve would be a horizontal line with low response time.
The lower bound response time for Baseline/remote and Proteus/remote is 80 ms as this is the round-trip time between sites.
In reality, all systems' plots have a ``hockey stick'' shape:
latency remains relatively low until a point in which the system fails to keep up with the offered load.
After that point, the system cannot achieve additional throughput, and response time increases.

Materialized-view-internal scales up to 9800 requests/second,
outperforming both Proteus deployments.
This deployment is intended to evaluate the performance of the core functionality of the Materialized view QPU,
by directly translating front page and vote operations to accesses to the QPU's state using a thread pool,
bypassing the client-server gRPC communication.
Because of that, we conclude that the gRPC communication incurs a significant overhead in the throughout that
can be achieved by the system.

Proteus/remote scales to 4200 requests/second, which is a 24\% overhead compared to the baseline deployment (Baseline/remote),
which scales to 5500 requests/second.
Both those deployments eventually read and write state to a MariaDB instance.
The difference is how they translate requests to database accesses.
The adapter used in the Baseline/remote deployment uses a simple logic that translates vote and front page request to database transactions.
Conversely, the Materialized view QPU used in the Proteus/remote deployment contains more complex logic,
such as parsing received queries in SQL form, and receiving records from its input stream and updating the materialized view.
We attribute the observed 24\% overhead to the more complex logic.

In the Proteus/local deployment, query response time is significantly lower.
This is achieved because moving the Materialized view QPU to the client site removes the need for a costly round-trip to
the server site, and thus removes the 80 ms lower bound.
In addition, Proteus/local achieves a 28\% increase in achieved throughput compared to the Baseline/remote deployment
(we consider the maximum achieved throughput for which response time does not exceed 20ms and 100ms respectively).
We attribute this improvement to the lower concurrency required to generate the same load in Proteus/local
compared to Proteus/remote and Baseline/remote.
In more detail, offering a certain load (volume of requests/second) requires creating a number of concurrent client threads,
each performing a request.
When the round-trip time between sites is 80 ms, each of these threads executes significantly longer compared to
when the round-trip is just a few milliseconds.
As a result, offering a given amount of load in the Proteus/remote setup results in a significantly greater number
of threads, and thus open connections to the QPU's gRPC server, than in the Proteus/local setup.
If the number of connections that can be opened is not bounded by a connection pool, this overloads the QPU's gRPC server,
significantly increasing response times.
When a connection pool is used, each requests needs to wait for an available connection,
again increasing the end-to-end response time experienced by the client.

\medskip
\noindent
\textbf{Conclusion.} Our experiments confirm that placing materialized views closer to the client benefits read-heavy applications
by removing costly round-trip communication across sites, and achieves scalability improvements.
This results is expected, and shows the benefits that can be achieved by enabling this placement.


\subsection{Freshness}
\label{sec:eval_freshness}

\subsubsection{Freshness vs Throughput}
\label{sec:eval_freshness_throughput}

In the actual Lobsters application (and the Baseline/remote deployment),
the vote count is maintained in the $stories$ table.
Because of the consistency guarantees of MariaDB,
the vote count of a story, is always up-to-date with the state of the $votes$ table.
However, maintaining a materialized view placed at a remote site synchronously adds prohibitive overhead to write operations.
Because of that the QPU graph in this evaluation maintains the materialized view asynchronously, and, as a result,
its state might be stale relative to the state of the corpus.

In this section, we present, for the experiments describe in the previous section,
the measurements for the freshness metrics presented in Section~\ref{sec:eval_setup} (update latency and returned version).
Our aim is to examine the effect of asynchronous derived state maintenance in the freshness of query results.
We present results for both placement schemes (Proteus/local and Proteus/remote).
Query results for the Baseline/remote deployment are always up-to-date,
and update latency is 0.


\begin{figure}[H]
\centering
  \includegraphics[width=0.7\textwidth]{./figures/evaluation/fr_latency_throughput.png}
  \caption{Throughput vs 95th percentile update latency.}
  \label{fig:fr_latency_throughput}
\end{figure}

Figure~\ref{fig:fr_latency_throughput} shows 95th percentile update latency as throughput increases,
for the Proteus/remote and Proteus/local deployments.
As described above, update latency is the delay between committing a vote in the Lobsters database,
and the corresponding vote count update in the materialized view.
The ideal throughput--update latency curve would be a horizontal line with latency close to the lower bound defined by
the communication latency.
The lower bound for Proteus/remote is 40 ms while for Proteus/local it is less than 1ms.
In reality, latency remains low as long as the system can keep up with the offered vote request load,
and then increases.

Results show the Proteus/local setup scales well; Update latency remains within 5-10 ms of the lower bound.
The Proteus/remote setup exhibits a higher update latency:
For 4250 requests/second, the update latency in Proteus/remote is 88\% higher than in Proteus/local,
relative to the lower bound.
This can be attributed to the same reasons as the scalability difference between the two deployments:
The Materialized view QPU in Proteus/remote is more loaded because more connections are concurrently ongoing
for the same load value, compared to Proteus/local,
leading to increased latency for applying updates to the view.

\begin{figure}[H]
\centering
  \includegraphics[width=0.7\textwidth]{./figures/evaluation/fr_latency_throughput_breakdown.png}
  \caption{Breakdown of 95th percentile update latency in the Proteus/local deployment
  (as shown in Figure~\ref{fig:fr_latency_throughput}) as throughput increases.}
  \label{fig:fr_latency_throughput_breakdown}
\end{figure}

Each vote committed in the database triggers a record that flows through the QPU graph, and eventually updates the corresponding
vote count in the Materialized view QPU.
Figure~\ref{fig:fr_latency_throughput_breakdown} shows a breakdown of the update latency in the Proteus/local
deployment.
It depicts the delay at each step that vote records follow through the QPU graph.

A vote record's path consists of the following steps:
\begin{enumerate}
  \item When the transaction that inserts a vote record into the Lobsters database commits,
  a trigger sends a message to the notification server (Section~\ref{sec:implementation}).
  The notification server then constructs an update record and sends it to the Corpus Driver QPU through a gRPC stream.
  The duration of this step is shown as \textbf{notification server} in Figure~\ref{fig:fr_latency_throughput_breakdown}.

  \item The Corpus driver receives an update record, and forwards it to the Sum QPU (\textbf{corpus driver}).

  \item The Sum QPU receives an update record, computes an updated vote count,
  and sends a corresponding record to the Join QPU (\textbf{sum}).

  \item \textbf{Cross-site communication} in Figure~\ref{fig:fr_latency_throughput_breakdown} corresponds to the delay for
  sending an update record from the Sum QPU, located at the server site, to the Join QPU, located at the client site.

  \item Finally, the Join QPU receives a record and updates the materialized view accordingly (\textbf{join-MV}).

\end{enumerate}

We observe that:
\begin{itemize}
  \item Update latency is dominated by cross-site communication  (up to 89\%).
  \item Latency at the Corpus driver and Sum QPUs is low: 2ms and 6ms at most.
  This is expected: the Corpus driver simply forwards records upstream;
  The Sum QPU, for each record, updates a vote count stored in memory, and sends a record upstream.
  \item Latency at the notification server and Materialized View QPU increases as the offered load increases.
  However, both scale well as the load increases.
  For the Materialized view QPU this is due to the increasing load in the database that stores the materialized view.
  The latency increase in the notification server can be attributed to the increased load in the database,
  resulting in more triggers being executed concurrently.
\end{itemize}

\begin{figure}[H]
\centering
  \includegraphics[width=0.7\textwidth]{./figures/evaluation/fresh_reads_throughput.png}
  \caption{Throughput vs percentage of query results and that return a fresh version.
  A returned result is considered fresh if 1) it is the most up-to-date version committed in the database at that time (k=1),
  2) it is amongst the two most up-to-date versions committed in the database at that time (k$\leq$2).}
  \label{fig:fresh_reads_throughput}
\end{figure}

\bigskip
\noindent
Figure~\ref{fig:fresh_reads_throughput} shows the freshness of query results as throughput increases,
measured as the percentage of query results that returned fresh versions.
We define a returned result as a single story with its vote count;
Each front page request returns 25 stories, and each is considered separately.
We consider a scenario in which only the most latest version is considered fresh (k=1),
and one in which the two most recent versions are considered fresh (k$\leq$2).

We observe that:
\begin{itemize}
  \item Proteus/remote has better freshness than Proteus/local, as expected.
  For the k=1 scenario, over 95\% of reads observe the latest versions under the highest load.
  For the k$\leq$2 scenario, freshness remains nearly constant at over 99\%.
  \item Freshness in Proteus/local decreases as throughput increases.
  Proteus/local suffers from up to 80\% stale query results (for k=1), and freshness decreases constantly as load increases.
  However, most stale results observe the second most up-to-date version:
  in the k$\leq$2 scenario over 97\% of query results are fresh.
\end{itemize}

These results can be explained using Figure~\ref{fig:fr_latency_throughput}.
In Proteus/local, it takes at least 45 ms for an updated vote count to be reflected in the materialized view,
but a front page request reaches the view with significantly lower delay.
When load is low, this does not lead to stale query results because there are a few vote requests (5\%).
However, as load increases, queries observe increasingly more stale materialized view  entries.

This is not the case for Proteus/remote.
There, both types of requests reach the materialized view with similar delay,
and because of the query-heavy nature of the workload,
most queries observe tha latest materialized view entries.

\begin{figure}[H]
\centering
  \includegraphics[width=0.7\textwidth]{./figures/evaluation/readV_cdf_throughput.png}
  \caption{CDF of returned version at 4250 requests/second for the Proteus/remote and Proteus/local deployments.}
  \label{fig:readV_cdf_throughput}
\end{figure}

Figure~\ref{fig:readV_cdf_throughput} displays a CDF showing how stale a result of a front page request is
(measured in number of versions),
under a load of 4250 requests/second, for the Proteus/remote and Proteus/local deployments.
Results show that:
\begin{itemize}
  \item In both deployments, most queries return the latest or second most recent version.
  \item Proteus/remote has better freshness than Proteus/local:
  Under the same conditions, Proteus/remote exhibits 4.5\% stale returned results,
  while Proteus/local 20\% (4.4X).
  However, in Proteus/local only 2.2\% of queries results are more stale than the second most recent version.
\end{itemize}

\medskip
\noindent
\textbf{Conclusion.}
Placing the materialized view close to the client, and thus away from the underlying datastore incurs a freshness penalty:
queries return stale results relative to the results that would have been obtained by querying the database.
However, for the workload characteristics in these experiments,
query results are rarely more stale than the second most recent version.
Moreover, update latency and versions freshness scale well as the system's load increases.
We can argue that the level of freshness shown in these experiments to be achievable when placing materialized views close
to the client is sufficient for many query-heavy applications that tolerate eventual consistency.

\subsubsection{Freshness vs round-trip latency}
\label{sec:eval_freshness_rtt}

The experiments in the previous section evaluated the effect of the query processing state placement under a constant
round-trip delay, as the load offered to the system increases.
In this section, we invert these two variables:
we measure freshness under a constant load,
as the (simulated) round-trip time between the application and client site increases,
for the Proteus/local deployment.
The aim of this experiment is to examine the effect of round-trip delay in freshness.

Experiments are performed under a load of 2000 and 4000 requests/second.
We have selected these values based on the results shown in Figure~\ref{fig:fr_latency_throughput}:
Under a load of 2000 requests/second both deployments are able to keep up with the offered load,
while under 4000 requests/second the Proteus/remote scheme exhibits increased update latency.

\begin{figure}[H]
\centering
  \includegraphics[width=0.7\textwidth]{./figures/evaluation/fr_latency_net_latency.png}
  \caption{Round-trip time between sites vs 95th percentile update latency, for the Proteus/local deployment.}
  \label{fig:fr_latency_net_latency}
\end{figure}

Figure~\ref{fig:fr_latency_net_latency} shows the update latency as the round-trip time between the two sites increases,
under 2000 and 4000 requests/second.
We observe that for both loads, update latency scales linearly with the round-trip time.
Under 2000 requests/second, update latency is at most 5ms above the lower bound set by the one-way network latency
between the two sites, while under 4000 requests/second it is at most 11ms (2.2X).

\begin{figure}[H]
  \begin{subfigure}{0.5\textwidth}
    \includegraphics[width=\linewidth]{./figures/evaluation/readV_freshness_netLatency_200.png}
    \caption{2000 requests/second.}
    \label{fig:readV_freshness_netLatency_200}
  \end{subfigure}%
  \hspace*{\fill}
  \begin{subfigure}{0.5\textwidth}
    \includegraphics[width=\linewidth]{./figures/evaluation/readV_freshness_netLatency_400.png}
    \caption{4000 requests/second.}
    \label{fig:readV_freshness_netLatency_400}
  \end{subfigure}%
\caption{Distribution of returned version vs round-trip time between sites.}
\label{fig:readV_freshness_netLatency}
\end{figure}

Figures~\ref{fig:readV_freshness_netLatency_200} and~\ref{fig:readV_freshness_netLatency_400}
show the distribution of returned version (which version relative to the most recent one was returned by a query)
for 2000 and 4000 requests/second respectively.
We observe that under both loads freshness decreases as round-trip time increases;
Increasing the load of the system increases the gradient of this decrease.
However, in both cases, queries, generally, observe at most the forth most recent version:
Only 0.06\% and 0.8\% of query results are more stale than the forth most recent version.

\medskip
\noindent
\textbf{Conclusion.}
Update latency, and the freshness of query results are primarily affected by round-trip time between sites,
and to a lesser degree by the system's load.

\subsection{Data transfer between sites}
\label{sec:eval_data_transfer}

\begin{table}[H]
\centering
\begin{tabular}{|c||c|c|c||}
\hline
Deployment & Baseline/remote & Proteus/remote & Proteus/local \\
\hline
Cross-site data transfer (MB) & 0 & 0 & 7.7 \\
\hline
Data transfer out to internet (MB) & $\approx$7700 & $\approx$7700 & $\approx$7700 \\
\hline
\end{tabular}
\caption{Measured data transfer for a 5 minute benchmark with a load of 4000 requests/second.}
\label{tab:data_transfer}
\end{table}

Distributing a query engine across multiple sites entails data transfer between sites.
If the system is deployed on a public cloud platform this incurs an additional cost
because data transfer between data centers is part of public clouds' pricing models.
For example, on AWS EC2, data transfer costs \$0.02 per GB \cite{aws:pricing}.

In the scenario used for these experiments, placing the Materialized view QPU at the client site entails that
update records from the Sum to the Materialized view QPU are sent between sites.

To measure the amount of inter-site data transfer,
we have implemented a mechanism for measuring and aggregating the size of outgoing messages at each QPU.
For a 5 minute benchmark, with 4000 requests/second (200 votes/second), 7.7MB of data were transferred between data centers.
This is because only 5\% of requests are votes, and the size of an update record is small (around 90 bytes),
as it only contains the id and vote count of a story.

In contrast, in the same benchmark, 7.5GB of data were sent as query responses.
This is because the size of a query response is around 4MB
(it contains the records of 25 stories), and 95\% of requests are queries.

We conclude that, in this the evaluation scenario, the materialized view can be placed in the client site without incurring
significant data transfer costs.

\subsection{Conclusion}

The evaluation presented in this section demonstrates that there are benefits and drawbacks to both placement options:
Placing materialized views close to the client results in improvements in response time and throughput,
at the expense of freshness.
Conversely, placing materialized views close to the corpus ensures fresh query results,
but brings limitations to response time and throughput.
As a result, client-site placement of materialized views is more suitable for applications that require low query response times or high
query load, and can tolerate stale query results;
server-site placement is better-suited for applications for which query processing performance is not critical,
but require up-to-date query results.
In addition, evaluation results show that Proteus can to efficiently implement both placement schemes.


\section{Federated metadata search for multi-cloud object storage}
\label{sec:eval_2}

\subsection{Experimental scenario}

The evaluation presented in this section is based on the case study presented in Section~\ref{sec:zenko}.
We consider a multi-cloud data serving system, composed of three storage locations.
A storage location may be a public cloud storage platform (e.g. Amazon S3, Microsoft Azure Blob storage, Google Cloud Storage),
or an on-premise storage system.
Each storage location is independent (not aware of the other locations),
and a multi-cloud data controller (\S\ref{sec:zenko}) is responsible for providing a common namespace across storage locations.
The controller implements an object storage API, such the AWS S3 API:
objects are composed of a primary key, a set of metadata attributes, and content.
This evaluation focuses on providing support for federated queries on metadata attributes.

We consider a system model composed of the 3 storage locations, in 3 geographically distant data
centers.
Each storage location stores a disjoin subset of the dataset.
An instance of the multi-cloud controller is deployed on each storage location,
and users are served by the controller that is geographically closer to their location.
An overview of the system model is shown in Figure~\ref{fig:eval_part2_overview}.

We consider an application that consists of two types of operations:
updating the metadata attributes of a given object,
and performing queries on metadata attributes.

\begin{figure}[H]
\centering
  \includegraphics[width=0.7\textwidth]{./figures/evaluation/eval_part2_overview.pdf}
  \caption{An overview of the system model in \S\ref{sec:eval_2}.}
  \label{fig:eval_part2_overview}
\end{figure}

\subsection{Methodology}

As discussed in Section~\ref{sec:zenko}, there are alternative approaches for designing a multi-cloud query engine.
The aim of this evaluation is to demonstrate the need for flexibility in the query engine's design for addressing the
needs of different applications,
and validate that QPU-based query engines can provide the required flexibility.

To achieve that, we consider 3 query engine configurations (QPU graphs), 3 workload types with different query-update ratios,
and 3 metrics (query processing performance, freshness, and data transfer between storage locations);
We experimentally determine which query engine configuration is better-suited for each combination of workload type and target metric.

\bigskip
\noindent
\textbf{Query engine configurations.}
The main functionality of the query engine in this scenario is to maintain secondary indexes for accelerating queries on
metadata attributes.
We consider the following approaches for partitioning and placing a multi-cloud secondary index across the system.

\begin{itemize}
  \item \textbf{Replicated global indexes (rg-index):}
  An index responsible for indexing data from all storage locations is deployed on each location.
  This approach has the advantage that queries are served by the local index.
  However, each index needs to receive update notifications from the two remote storage locations.
  Depicted in Figure~\ref{fig:rg_index}.

  \item \textbf{Partitioned index (p-index):}
  In this configuration, the index on each storage location is responsible for the local corpus.
  The system forwards each query to all 3 storage locations, and combines the retrieved results.
  This approach requires a third of the storage space for indexes compared to rg-index,
  and ensures that update notifications are sent only to the local index,
  at the expense of requiring cross-site communication for serving queries.
  Depicted in Figure~\ref{fig:p_index}.

  \item \textbf{Partitioned index with caching (p-index-cache):}
  This configuration is an extension to p-index that uses a caching layer with the aim of reducing access to remote indexes.
  For each index, a cache responsible for caching sub-query results from it is deployed on the two remote storage locations.
  Depicted in Figure~\ref{fig:p_index_cache}.
\end{itemize}

\begin{figure}[H]
  \begin{subfigure}{0.5\textwidth}
    \includegraphics[width=\linewidth]{./figures/evaluation/rg_index.pdf}
    \caption{}
    \label{fig:rg_index}
  \end{subfigure}%
  \hspace*{\fill}
  \begin{subfigure}{0.5\textwidth}
    \includegraphics[width=\linewidth]{./figures/evaluation/p_index.pdf}
    \caption{}
    \label{fig:p_index}
  \end{subfigure}%
  \caption{The (a) replicated global indexes (rg-index) and (b) partitioned index (p-index) QPU graph configurations.}
  \label{fig:p_rg_index}
\end{figure}

\begin{figure}[H]
\centering
  \includegraphics[width=0.7\textwidth]{./figures/evaluation/p_index_cache.pdf}
  \caption{The partitioned index with caching (p-index-cache) QPU graph configuration.}
  \label{fig:p_index_cache}
\end{figure}

\bigskip
\noindent
\textbf{Workload types.}
Overall, we consider a query-heavy application that requires the use of secondary indexes for query processing.
We examine 3 workload types, each with a different mix of query and update operations:
95\% queries - 5\% updates (w95/5), 80\% queries - 20\% update (w80/20),
60\% queries - 40\% update (w60/40).

\bigskip
\noindent
\textbf{Evaluation metrics.}
For this evaluation, we use the metrics discussed in Section~\ref{sec:eval_1}:
query processing performance, freshness and data transfer cost for data transferred between storage locations.

\subsection{Experimental Setup}
On each storage location, data is stored on an instance of MongoDB.
We use MongoDB as an object store:
an object is represented by a MongoDB document, with document fields representing the object's metadata attributes.
At the start of each experiment, we preload each storage location with with 33k objects,
so that in total the system stores 100k objects.

The Index QPU implements an in-memory B-tree secondary index.
The Cache QPU implements a cache with a least recently used (LRU) eviction policy,
and a time-to-live (TTL) based invalidation policy.
We set the size of caches so that each cache can hold 50\% of the index entries it is responsible for,
and configure the TLL value to 5 seconds.

We configure the system so that there is an 80sm round trip time between storage locations in order to
simulate a multi-cloud system deployed over distant geographic locations.

\bigskip
\noindent
\textbf{Workload generation and measurements configuration.}
We use the Yahoo! Cloud Serving Benchmark (YCSB) \cite{ycsb} for generating workload and performing measurements.
The core operation of the YCSB framework is that it drives a number of client threads,
each executing a sequential series of operations by making calls to the underlying system,
and measures the latency and achieved throughput of their operations.
At the end of an experiment, a statistics model aggregates the measurements and reports the achieved throughput and
the measured latency percentiles.

We have modified YCSB's MongoDB driver to send query operations to Proteus.
We deploy a YCSB instance on each storage location.
On each location, YCSB client threads send update operations to the local MongoDB instance,
and query operations to the local Partition Manager QPU.
At the end of an experiment, we gather and aggregate measurements from all three storage locations,
and compute the total achieved throughput and latency percentiles.

Each object in the dataset has multiple, randomly generated metadata attributes.
We ensure that a numeric attribute with a specified key existing in every object.
Query operation in the workload are point queries that refer to this attribute.
Both the attribute's values and query predicated follow a uniform distribution.
We use the mechanism presented in Section \ref{sec:eval_setup} for measuring freshness.

Experiments run for 5 minutes unless otherwise specified, and we start taking measurements after an initial
warmup period of 30 seconds.

\subsection{Query processing performance}

\begin{figure}[H]
\centering
  \includegraphics[width=0.9\textwidth]{./figures/evaluation/ycsb_responseTime.png}
  \caption{Throughput vs query response time. Plots for the rg-index and p-index configuration show the 90th percentile response time;
  Plots for the p-index-cache configuration show average response time in order to capture the effect of caching in reponse time.}
  \label{fig:ycsb_responseTime}
\end{figure}

Figure~\ref{fig:ycsb_responseTime} shows throughput--query response time plots for the 3 QPU graph configuration and 3 workload
types.
We observe that:
\begin{itemize}
  \item Response time in the p-index configuration is 80ms higher than in the rg-index configuration.
  This is because in p-index the query engine forwards queries to indexes across storage locations while
  in rg-index queries are served by the local index.
  \item Average response time in the p-index-cache configuration is around 80ms when load is low,
  and then decreases as load increases.
  This effect is the result of caching:
  In low load values, caches are not filled, and most queries result in cache misses;
  As load increases, more queries are served from the cache, decreasing the average query response time.
  \item The p-index configuration achieves around 50\% throughput compared to rg-index.
  This can be attributed to the closed loop workload generation mechanism of YCSB:
  Given a certain number of client threads, in rg-index each query operation has a takes less than 25ms
  when the system is not saturated, while in p-index each query operation take 85-90ms because of the 80ms
  round trip time between storage locations.
  Therefore, in p-index each client thread cannot offer the same load as in rg-index.
  \item The p-index-cache configuration achieves 21\% lower throughput (for the w95/5 workload).
  This is expected as each cache miss adds an 80ms overhead to response time,
  resulting in client threads being able to generate less load.
\end{itemize}

We conclude that the rg-index configuration is better-suited for achieving the best query processing performance.
However, this comes at the expense of memory overhead for maintaining a global index at each storage location.
The rg-index-cache configuration requires achieve query processing performance comparable to rg-index,
and requires 66\% the memory of rg-index (because each cache is configured to 50\% the size of an index).

\subsection{Freshness}

In this section, we examine the query result freshness achieved by the alternative QPU graph configurations.
In order to evaluate the freshness of the different configurations,
we break them down into \text{placement patterns}, we evaluate the freshness of each placement pattern,
and reason about how placement pattern freshness contributes to the overall freshness of QPU graph configurations.
The placement patterns present in the three QPU graph configurations are shown in Figure~\ref{fig:placement_patterns}.

\begin{figure}[H]
  \begin{subfigure}{0.24\textwidth}
    \includegraphics[width=\linewidth]{./figures/evaluation/ycsb_freshness_local.pdf}
    \caption{}
    \label{fig:ycsb_freshness_local}
  \end{subfigure}%
  \hspace*{\fill}
  \begin{subfigure}{0.24\textwidth}
    \includegraphics[width=\linewidth]{./figures/evaluation/ycsb_freshness_remote_corpus.pdf}
    \caption{}
    \label{fig:ycsb_freshness_remote_corpus}
  \end{subfigure}%
  \hspace*{\fill}
  \begin{subfigure}{0.24\textwidth}
    \includegraphics[width=\linewidth]{./figures/evaluation/ycsb_freshness_remote_client.pdf}
    \caption{}
    \label{fig:ycsb_freshness_remote_client}
  \end{subfigure}%
  \hspace*{\fill}
  \begin{subfigure}{0.24\textwidth}
    \includegraphics[width=\linewidth]{./figures/evaluation/ycsb_freshness_remote_client_cache.pdf}
    \caption{}
    \label{fig:ycsb_freshness_remote_client_cache}
  \end{subfigure}%
  \caption{Placement patterns. (a) Corpus, index and clients are located on the same storage location (local).
  (b) The corpus is located on a remote storage location (remote-corpus). (c) The index is co-located with the corpus;
  clients are located on a remote storage location (remote-client). (d) A cache is used to reduce cross-location communication (remote-client-cache).}
  \label{fig:placement_patterns}
\end{figure}

\begin{table}[H]
\centering
\begin{tabular}{|c||c|c|c|c||}
\hline
& local & remote-corpus & remote-client & remote-client-cache \\
\hline
rg-index & $\blacksquare$ & $\blacksquare$ &  & \\
\hline
p-index & $\square$ & & $\blacksquare$ & \\
\hline
p-index-cache & $\square$ & &  & $\blacksquare$ \\
\hline
\end{tabular}
\caption{Placement patterns that QPU graph configurations are composed of.
$\blacksquare$ $\blacksquare$ indicates that both patterns contribute to the configuration's freshness.
$\square$ $\blacksquare$ indicated that only the pattern with $\blacksquare$ contributed to the configuration's freshness.}
\label{tab:placement_patterns}
\end{table}

Table \ref{tab:placement_patterns} shows the placement patterns used for each QPU graph configuration.
In rg-index, each index is connected to both local (\ref{fig:ycsb_freshness_local}) and remote (\ref{fig:ycsb_freshness_remote_corpus}) corpus.
As a result, query result freshness is affected by the freshness characteristics of both patterns.
In p-index, each Partition Manager QPU is connected to the the local index (\ref{fig:ycsb_freshness_local}),
and two remote indexes (\ref{fig:ycsb_freshness_remote_client}).
The Partition Manager forwards a given query to all three indexes and waits to receive all responses before responding to
the client.
Because of that, the freshness of p-index is determined by the freshness of the remote-client pattern.
Similarly, the freshness of p-index-cache is determined by the freshness of the remote-client-cache pattern
(\ref{fig:ycsb_freshness_remote_client_cache}).


\begin{figure}[H]
  \begin{subfigure}{0.5\textwidth}
    \includegraphics[width=\linewidth]{./figures/evaluation/ycsb_update_latency_local.png}
    \caption{}
    \label{fig:ycsb_update_latency_local}
  \end{subfigure}%
  \hspace*{\fill}
  \begin{subfigure}{0.5\textwidth}
    \includegraphics[width=\linewidth]{./figures/evaluation/ycsb_update_latency_remote.png}
    \caption{}
    \label{fig:ycsb_update_latency_remote}
  \end{subfigure}%
  \caption{Throughput vs 95th percentile update latency for the Index QPU in the local (a) and remote-corpus (b) placement patterns.}
  \label{fig:ycsb_update_latency_local_local_remote}
\end{figure}

Figures~\ref{fig:ycsb_update_latency_local} and~\ref{fig:ycsb_update_latency_remote} show the 95th percentile update latency as throughput increases,
for local (\ref{fig:ycsb_update_latency_local}) and remote-corpus (\ref{fig:ycsb_update_latency_remote}) placement patterns.

We implement the storage tier's Subscribe API (\S\ref{sec:storage_tier_api}) using MongDB's change stream functionality \cite{mongo:changestreams}.
Our results show that, with the configuration used for these experiments, the change stream client receives a notification with a 50-60ms delay.
We indicate this as the baseline update latency in the plots (notification).

In addition, we note that our mechanism for measuring update latency includes in update latency
the response time of the update operation performed by the client.
Because update operation response time increases with load, part of the increase in update latency is due to the update operation response time.
Update operation response time is shown in Figure~\ref{fig:ycsb_write_latency}.

\begin{figure}[H]
\centering
  \includegraphics[width=0.7\textwidth]{./figures/evaluation/ycsb_write_latency.png}
  \caption{Throughput vs client update response time. Client updates are always local,
  as in all configurations clients write to the local MongoDB instance.}
  \label{fig:ycsb_write_latency}
\end{figure}

\noindent
For the update latency results in Fig.~\ref{fig:ycsb_update_latency_local_local_remote}, we observe that:
\begin{itemize}
  \item In remote-corpus, update latency is at least 40ms higher that the baseline
  due to the 40ms network latency between storage locations.

  \item Update latency increases with the ratio of updates.
  This is due to contention, as each update needs to acquire a write lock on the index.
  For both placement patterns, for the w80/20 and w60/40 workloads,
  update latency does not scale to loads larger than 13k operations/second.
\end{itemize}

The evaluation results shown in Figures~\ref{fig:ycsb_responseTime} and~\ref{fig:ycsb_update_latency_local_local_remote} demonstrate the trade-off betwee
query performance and query result freshness.
The rg-index configuration, in which queries are served locally, achieves better query processing performance at the expense of increased update latency;
The update latency overhead is determined by the network communication latency between storage locations.
Conversely, the p-index configuration ensures low update latency at the expense query processing performance.
\begin{figure}[H]
  \begin{subfigure}{0.5\textwidth}
    \includegraphics[width=\linewidth]{./figures/evaluation/ycsb_readV_freshness_throughput_9505.png}
    \caption{}
    \label{fig:ycsb_readV_freshness_throughput_9505}
  \end{subfigure}%
  \hspace*{\fill}
  \begin{subfigure}{0.5\textwidth}
    \includegraphics[width=\linewidth]{./figures/evaluation/ycsb_readV_freshness_throughput_6040.png}
    \caption{}
    \label{fig:ycsb_readV_freshness_throughput_6040}
  \end{subfigure}%
  \caption{Throughput vs percentage of queries that returned the most recent version in the w95/5 (a) and w60/40 (b) workload.}
  \label{fig:ycsb_readV_freshness_throughput}
\end{figure}

Next, we examine how update latency affects query result freshness.
Figures~\ref{fig:ycsb_readV_freshness_throughput_9505} and~\ref{fig:ycsb_readV_freshness_throughput_9505} show
the percentage of query operations that return the most recent version of an object as load increases,
for the w95/5 and w60/40 workloads respectively.
We observe that:
\begin{itemize}
  \item For the w95/5 workload, all patterns except remote-client-cache return fresh result for close to 100\% of queries.

  \item For the w60/40 workload, local and remote-corpus return stale results for loads greater than 10k operations/second
  (8\% and 23\% stale query results respectively).

  \item For both workloads, the remote-client-cache pattern results in significantly more stale query results than the other
  patters.
  This is due to the time-based invalidation policy, and the TTL value of 5 seconds.
\end{itemize}

We conclude that, for query-dominated workloads, all pattern exhibit high freshness.
For workloads with higher update rations,
caching with time-based invalidation offers a balance between memory resource overhead and query processing performance,
at the expense of lower query result freshness.

\subsection{Data transfer between storage locations}

\begin{table}[H]
\centering
\begin{tabular}{|c||c|c|c|c||}
\hline
& \textbf{w95/5} & \textbf{w80/20} & \textbf{w60/40} & \textbf{w5/95}\\
\hline
\textbf{rg-index} & 64MB & 213MB & 467MB & 1.37GB \\
\hline
\textbf{p-index} & 18GB & 5.95GB & 5.85GB & 1.25GB \\
\hline
\textbf{p-index-cache} & 4.95GB & 5.58GB & 4.97GB & 947MB \\
\hline
\end{tabular}
\caption{Amount of data transfer between sites for a 5 minute benchmark with 256 YCSB client treads at each storage location.}
\label{tab:ycsb_data_transfer}
\end{table}

In this section, we measure the amount of data sent across storage location for different QPU graph configurations and
workload types.

In the rg-index configuration, cross-location data transfer occurs for sending update notifications from the corpus
to remote Index QPUs:
Each update operation generates three update notifications, one for each storage location.
As a result, data transfer increases as update ratio increases.

In the p-index configuration, cross-location data transfer occurs for retrieving query results from remote Index QPUs.
Partition Manager QPUs forward every query to all three storage locations, and retrieve the results.
As a result, data transfer decreases as query ratio decreases.

In the p-index-cache configuration, cross-location data transfer occurs on cache misses.
When an index entry is not present in a cache, the Cache QPU forwards it to the corresponding Index QPU, and retrieves the
results.
Similarly to the case of p-index, data transfer decreases as query ratio decreases.

These results demonstrate another aspect of the trade-off between the rg-index and p-index configurations.
In general, p-index results in significantly higher data transfer than rg-index.
Our results show that the rg-index configuration results in more data transfer than the p-index configuration for workloads with at least 95\% updates (w5/95).
This because the size of an update notification is significantly smaller than the size of a query response.
Given an update, the update notification contains the object's primary key, the updated attribute(s) and a timestamp;
Given a query, the response contains the primary keys, attributes and timestamps of all objects that match the query predicate.

Furthermore, we observe that using a caching layer results in a 72\% data transfer reduction for the w95/5 workload.
Therefore, using the p-index-cache configuration can result in significant data transfer cost reduction compared to
the p-index configuration.

\subsection{Conclusion}

\begin{table}[H]
\centering
\begin{tabular}{|c||c|c|c||}
\hline
& \textbf{w95/5} & \textbf{w80/20} & \textbf{w60/40}\\
\hline
\textbf{Query processing performance} & rg-index & rg-index & rg-index \\
\hline
\textbf{Query result freshness} & p-index & p-index & p-index \\
\hline
\textbf{Data transfer cost} & rg-index & rg-index & rg-index \\
\hline
\end{tabular}
\caption{Summary of experimental comparison between the three QPU graph configurations.}
\label{tab:ycsb_summary}
\end{table}

Table~\ref{tab:ycsb_summary} summarizes the evaluation results presented in this section
by showing which QPU graph configuration is better-suited for each combination of target metric
and workload type.

However, applications are rarely interested in optimizing a single metric,
but rather finding the right balance in the trade-offs between metrics.
Our evaluation results indicate that, for query heavy workloads,
the p-index-cache configuration offers a balance between query processing performance,
freshness, and memory resource and data transfer overhead.

The evaluation presented in this section demonstrates the trade-offs between
different index and cache partitioning and placement approaches,
and shows that Proteus can be efficiently used to tune the trade-offs between query processing
performance, freshness and operational costs by controlling the query engine's architecture.

\section{Conclusions}

The evaluation results presented in this chapter confirm our analysis of the trade-offs
involved in the placement of derived state.
This validates the central argument of this thesis:
No single derived state partitioning and placement approach is optimal for all needs,
and flexibility is required for addressing different needs.

Moreover, the evaluation demonstrates the expressiveness of the proposed approach:
QPU-based query engines enable the flexibility for configuring derived state partitioning and placement
with simple configuration changes to the query processing middleware, without requiring changes to the application.
QPU-based query engines can be used to flexibly partition derived state, and place state and query processing computation
freely across the system.
THanks to that they can be employed to address diverse application characteristics and requirements.


\chapter{Related work}
\label{ch:related_work}
% \section{Secondary attribute querying in NoSQL databases}

% Many large-scale key-value storage systems sacrifice features like secondary indexing and/or consistency in favor of scalability or performance. This limits the ease and efficiency of application development on such systems.

% \subsection{Secondary index based approaches}

% \textit{SLIK: Scalable Low-Latency Indexes  for a Key-Value Store}

% SLIK \cite{kejriwal:slik} is a secondary indexing system designed with the design goals of
% (1) low latency,
% (2) horizontal throughput scalability,
% (3) consistency, specifically the requirement of providing the same strong consistency as a centralized system,
% and (4) availability.

% SLIK partitions secondary indexes using the scheme which we refer to as partitioning by term
% (in \cite{kejriwal:slik} the term independent partitioning is used.
% Index partition (referred to as an indexlets) are represented by B+trees.
% Index entries store hashes of data items' primary keys.
% Therefore, after an index lookup each data item contains the result need to be retrieved from the storage engine.

% To achieve consistent index lookups, SLIK using two techniques:
% \begin{itemize}
%   \item \textit{Ordered writes}.
%   Index entries are created before updates to the corresponding data items are applied,
%   and old index entries are removed asynchronously.
%   This guarantees that if a data item contains a secondary attribute value, then an index lookup for that value will return the data item,
%   by ensuring that the lifespan of each index entry spans that of the corresponding data item.
%   \item \textit{Treating data items as ground truth and index entries as hints}.
%   The system verifies the results of index lookups by checking the corresponding data items.
%   This guarantees that if a data item is returned by an index lookup then this data item contains the requested secondary key.
% \end{itemize}

% Moreover, SLIK performs long-running bulk operations such as index creation, deletion and migration in the background,
% in order to avoid blocking blocking normal operations.

% SLIK is implemented in RAMCloud \cite{ousterhout:ramcloud}, a distributed in-memory key-value storage system.

% This work analyses alternative approaches available in a number of aspects in the design of a secondary indexing system, and discusses the tradeoffs these approaches.
% They make specific design decisions guided by their design goals, for example the requirement for consistency for index lookups.
% Moreover, this work does not consider a geo-distributed setting.

% \textit{Secondary Indexing Techniques for Key-Value Stores: Two Rings To Rule Them All} \cite{dsilva:tworings}

% % In this paper, we explore the challenges associated with indexing modern distributed table-based data stores and investigate two secondary index approaches  which  we  have  integrated  within  HBase.
% % Our detailed analysis and experimental results prove the bene-fits of both the approaches.  Further, we demonstrate that such  secondary  index  implementation  decisions  cannot  be made in isolation of the data distribution and that different indexing approaches can cater to different needs.

% % We discuss two indexing strategies for distributed key-value  stores:  one  based  on  distributed  tables  that  is able to exploit the table model of the underlying sys-tem for index management, the other using a co-location approach allowing for efficient main-memory access

% % Both  strategies  are  implemented  and  integrated  intoHBase in a non-intrusive way.

% % We provide an enhanced client interface to query HBase tables  using  secondary  indexing  that  supports  both point queries and range queries.

% % We present a detailed performance metrics on various database operations with secondary indices and a comparative analysis of the different approaches

% % We present a thorough analysis on the effects of data distribution on different indexing approaches.

% % we provided a very detailed analysis of how different data distributions warrant different indexing approaches and demonstrated a case for both implementations.
% % Our results show that there is clearly  a  benefit  to  having  secondary  indices  in  HBase,and  that  they  can  be  often  built  with  reasonable  performance overhead.
% % Although there has been some prior works to achieve secondary indexing in HBase, our work have been more detailed and insightful about the various alternatives and clearly shows that there is no one-stop solution to secondary indexing needs in HBase.



% \textit{A Comparative Study of Secondary Indexing Techniques inLSM-based NoSQL Databases} \cite{qader:comparativestudy}

% Qader et al. study the secondary indexing techniques used in state-of-the-art commercial and research NoSQL databases.
% More specifically, they categorize secondary indexes in (1) stand-alone indexes, where indexing structures are built and maintained
% and (2) filter indexes, where there is no separate secondary index structured, but secondary attribute index information is stored inside the original data blocks.
% Stand-alone indexing techniques are further categorized to those that perform in-place update (i.e. for each write the index structures are accessed, updated and stored back to disk),
% and those that perform append-only updates.
% The authors implement a number of different secondary indexing techniques on top of LevelDB [12] and study the trade-offs between different indexing techniques on various workloads.
% This work is mainly focused on a single server instance of LevelDB and does not consider a distributed setting.


% In this paper, we present a taxonomy of NoSQL secondary indexes,broadly split into two classes:Embedded Indexes(i.e. lightweight filters embedded inside the primary table) andStand-Alone Indexes(i.e. separate data structures).

% we built LevelDB++, on top of Google’s popular open-source LevelDB key-value store.
% There, we implemented two Embedded Indexes and three state-of-the-art Stand-Alone indexes, which cover most of the popular NoSQL databases.
% Our comprehensive experimental study and theoretical evaluation show that none of these indexing techniques dominate the others:
% the embedded indexes offer superior write throughput and are more space efficient, whereas the stand-alone secondary indexes achieve faster query response times.
% Thus, the optimal choice of secondary in-dex depends on the application workload.
% This paper provides an empirical guideline for choosing secondary indexes.

% \textit{Diff-Index: Differentiated Index in Distributed Log-Structured Data Stores} \cite{tan:diffindex}

% \textit{Schema-Agnostic Indexing with Azure DocumentDB} \cite{shukla:schemaagnostic}


% \subsection{Secondary index partitioning schemes}
% \cite{dsilva:tworings}

% \cite{kejriwal:slik}
% SLIK achieves high scalability by distributing index entries independently from their objects rather than co-locating them

% % SLIK: Cassandra [20], DynamoDB [3] and Phoenix [7] on HBase [4] provide local secondary indexes which are partitioned using the co-location approach
% % Some of the systems above like DynamoDB [3] and Phoenix [7] on HBase [4] also provide global secondary indexes, but they are only eventually consistent.

% \subsection{Approaches not based on secondary indexing}

% \textbf{HyperDex: A Distributed, Searchable Key-Value Store} \cite{escriva:hyperdex}

% % SLIK : HyperDex [17] is a disk-based large-scale storage sys- tem that supports consistent indexing. It partitions data using a novel hyperspace hashing scheme by mapping objects’ attributes into a multidimentional space. As the number of attributes increase, the number of hyperspaces increases dramatically. HyperDex alleviates this by par- titioning tables with many attributes into multiple lower- dimensional hyperspaces called subspaces. HyperDex also replicates the entire contents of objects in each index. This means that while HyperDex provides an ef- ficient mechanism for search, it uses more storage space for the extra copies of objects. While this is acceptable for disk based systems, it would be very expensive for main-memory based systems.

% \textbf{Replex: A Scalable, Highly Available Multi-Index DataStore} \cite{tai:replex}


% \section{Multi-site Web search}

% \textbf{On the Feasibility of Multi-Site Web Search Engines} \cite{baezayates:multisitefeasibility}

% \textbf{Quantifying Performance and Quality Gains in Distributed Web Search Engines} \cite{cambazoglu:multisitequantifying}

% \textbf{Query Forwarding inGeographically Distributed Search Engines} \cite{cambazoglu:multisiteforwarding}

% \textbf{Improving the Efficiency of Multi-site Web Search Engines} \cite{frances:multisiteimprovingefficiency}

% \section{Modular - Flexible architectures}

% \textbf{The Click Modular Router} \cite{kohler:click}

% Click is a new software architecture for building flexible and configurable routers.

% A Click router is assembled from packet processing modules called elements.
% Individual elements implement simple router functions like packet classification, queueing, scheduling, and interfacing with network devices.
% A router configuration is a directed graph with elements at the vertices;
% packets flow along the edges of the graph.
% Several features make individual elements more powerful and complex configurations easier to write, including pull connections,
% which model packet flow driven by transmitting hardware devices, and flow-based router context, which helps an element locate other interesting elements.

% This paper presents Click, a flexible, modular software architecture for creating routers.
% Click routers are built from fine-grained components;
% this supports fine-grained extensions throughout the forwarding path.
% The components are packet processing modules called elements.

% A Click element represents a unit of router processing.
% An element represents a conceptually simple computation,such as decrementing an IP packet’s time-to-live field,
% rather than a large, complex computation, such as IP routing.
% A Click router configuration is a directed graph with elements at the vertices.
% An edge, or connection, between two elements represents a possible path for packet transfer.

% \bibliographystyle{plainnat}
% \bibliography{refs}

% \section{Distributed query processing}

% \subsection{Query processing in Peer-to-Peer systems}
% DHT stuff.

% \subsection{Query processing in distributed databases}


% \section{Modular architectures}


% \section{Stream processing and Data-flow systems}

\chapter{Conclusion and Future Work}
\label{ch:conclusion}
Query processing is a crucial component of data serving systems.
Nowadays, applications distribute data across multiple geographically distributed data centers in order to
efficiently serve users worldwide.
Moreover, organizations spread their data and processing between on-premise and cloud environments,
as well as between multi-cloud cloud providers, in order to improve fault tolerance and decrease costs.
As a result, efficient geo-distributed query processing is essential for addressing the needs of today's internet-scale
applications.

In this thesis, we have studied the design decisions and trade-offs involved in the design of geo-distributed query engines
that maintain derived state for speeding-up query processing.
We have shown that, in the presence of these trade-offs, the placement of query processing state across the system,
and the communication patterns involved in query processing and state maintenance are crucial aspects that affect
the query engine's performance, effectiveness and operational costs.
However, existing systems lack support for configurable placement of query processing state.
To address this problem, this thesis has presented a query engine architecture model aimed at enabling flexible and
configurable placement of query processing state and computations,
and an implementation of that model in Proteus, a framework for constructing application-specific, geo-distributed
query engines.
The core contribution of this thesis is a query processing component abstraction that combines microservice and
stream processing semantics, called Query Processing Unit.
This abstraction serves as the building block for composing modular query engine architectures.
The characteristics of the query processing unit enable flexibility in the design of the query engine's architecture,
and the placement of the query engine's state across the system.
This flexibility is essential for navigating the trade-offs of geo-distributed query processing,
and adjusting to the requirements and characteristics of different applications.


\bibliographystyle{plainnat}
\refstepcounter{chapter}
\bibliography{refs}

\end{document}
