Query processing is a crucial component of data serving systems.
Nowadays, applications distribute data across multiple geographically distributed data centers in order to
efficiently serve users worldwide.
Moreover, organizations spread their data and processing between on-premise and cloud environments,
as well as between multi-cloud cloud providers, in order to improve fault tolerance and decrease costs.
As a result, efficient geo-distributed query processing is essential for addressing the needs of today's internet-scale
applications.

In this thesis, we have studied the design decisions and trade-offs involved in the design of geo-distributed query engines
that maintain derived state for speeding-up query processing.
We have shown that, in the presence of these trade-offs, the placement of query processing state across the system,
and the communication patterns involved in query processing and state maintenance are crucial aspects that affect
the query engine's performance, effectiveness and operational costs.
However, existing systems lack support for configurable placement of query processing state.
To address this problem, this thesis has presented a query engine architecture model aimed at enabling flexible and
configurable placement of query processing state and computations,
and an implementation of that model in Proteus, a framework for constructing application-specific, geo-distributed
query engines.
The core contribution of this thesis is a query processing component abstraction that combines microservice and
stream processing semantics, called Query Processing Unit.
This abstraction serves as the building block for composing modular query engine architectures.
The characteristics of the query processing unit enable flexibility in the design of the query engine's architecture,
and the placement of the query engine's state across the system.
This flexibility is essential for navigating the trade-offs of geo-distributed query processing,
and adjusting to the requirements and characteristics of different applications.