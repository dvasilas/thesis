The goal of this chapter is to motivate and define the research problem that
this thesis addresses, and to introduce its contributions.

Idea 1: Start by discussing the importance of design trade-offs in distributed
systems in general.
Almost every aspect of a distributed system faces trade-offs.
Due to that, the design of each system needs to be informed by application/user
requirements and workload characteristics.
A lot of work has been done to address this.
\begin{itemize}
  \item One size does not fit all. Specialized database designs for specific use
  cases.
  \item Online database reconfiguration.
  \item Using learned models (machine learning for various database components).
\end{itemize}

The same is true for query processing.
There have been approaches for "tunning" query processing for the needs for different applications:
\begin{itemize}
  \item Selecting which indexes/materialized view to materialize in databases.
  \item Optimizing distributed query execution (finding the best place to execute each operator).
\end{itemize}

Limitation of this state-of-the-art:
\begin{itemize}
  \item Current approaches do not provide flexible placement of query processing state:
  \begin{itemize}
    \item This is definitely the case for sharded NoSQL databases.
    \item TODO: determine how relational dabasese hanlde index placement.
    \item Haven't seen works considering flexible cache or materialized view placement.
  \end{itemize}
\end{itemize}

Idea 2: An argument for our choice of separating the storage and query processing tier is that disaggregation of storage
and query processing is actually a trend in cloud systems.
One example to use is AWS Athena that provides queries on top of AWS S3 with a pay-per-query pricing scheme.


\section{Problem Statement}

\section{State-of-the-art}

% For motivating the work, we will state the relevance and timeliness of two
% application needs / market trends / ... : attribute-based data retrieval, and
% geo-distribution of data.

% Attribute-based data retrieval: \\
% Retrieving data based on descriptive attributes attached to them (as opposed to
% retrieval based on primary key or content-based retrieval).

% Data geo-distribution: \\
% At the same time, data is increasingly geo-distributed (and federated).

% Problem definition: \\
% This thesis studies how attribute-based retrieval can be performed efficiently
% on geo-distributed data.


\section{Contributions}
\subsection{A design pattern for configurable query processing architectures}
Chapter~\ref{ch:design_pattern}

\subsection{Proteus: Towards application-specific query processing systems}
Chapter~\ref{ch:proteus}