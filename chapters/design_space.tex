The goal of this chapter is to draw the design space of geo-distributed query processing.

This includes:
\begin{itemize}
  \item Listing the axes of the design space, and the options for each axis.
  \item Describing the expected performance and efficiency characteristics
  of each point in the design space.
  \item Describe the trade-offs that occur from the above analysis.
  \item Discuss the additional challenges/constraints that occur from the geo-
  distribution of data.
  \item Present state-of-the-art approaches and their limitations.
\end{itemize}

\section{Design choices}
Describe the design choices involved in building a geo-distributed query engine.
For each, list the possible options and comment on their effect on the engine's
performance and efficiency.

\subsection{Query processing techniques}
The choice between (1) performing query processing by iterating over the corpus
(filtering, aggregation without materialized views), (2) using caches on top of
that, or (3) maintaining derived state (indexes, materialized views) updated in
response to changes to the corpus data.

\subsection{State maintenace}
When maintaining derived state, the choice between synchronously versus
asynchronously updating it for each change to the corpus.

State maintenance schemes have two characteristics:
\begin{itemize}
  \item How much of the state is materialized.
  When maintaining an index we assume that every index entry is materialized.
  In contrast, when maintaining a cache, only a subset of entries are
  materialized
  (a subtlety here is how an ''entry'' is defined, it might be different between
  index and cache).
  There is a finite storage area for materializing entries.
  When it is filled, some entries are evicted to make space for new entries to
  be materialized.
  \item When is the state being updated.
  The options for this can be:
  \begin{itemize}
    \item Update-triggered - Synchronously. When an update to the corpus
    occurs, it is blocked until the derived state is up-to-day.
    \item Update-triggered - Asynchronously. Updating the derived state is done
    outside of the critical path of the corpus update.
    \item Read-triggered. The derived state mays updated as a response to the
    state being read (cache misses).
    This is used in caching.
  \end{itemize}
\end{itemize}
  Based on the above analysis, we can argue that caching is just another state
  maintenance scheme, and therefore could be treated as a configuration
  parameter when deploying a QPU.
  (Note: this is just a way to unify and simplify things by merging caching
  QPUs with index QPUs, materialized view QPUs etc.)

\subsection{Component placement}
There are two types of computational query processing operator can be
involved in:
\begin{itemize}
  \item Update-triggered: For example, updating an index as a result to a corpus
  update
  \item Query-triggered: For example, index/cache lookup as a result to a query.
\end{itemize}
This can apply to any query processing operator, not only stateful ones.

The placement determines how close (in terms of network round trip time) the
operator is placed related to its sources of updates and queries.

Deciding on placement schemes can lead to trade-offs cause by:
\begin{itemize}
  \item The benefits and costs of placing operators close to update sources
  versus close to query sources.
  \item An operator might have multiple update and/or query sources.
\end{itemize}

We can define ``levels`` of placement: on the same physical machine, on the same
cluster, in the same DC.

\section{Trade-offs}
Summarize how the previously described design choices result to performance and
efficiency trade-offs.

Note: This may be redundant depending on how detailed those trade-offs are
discussed in the previous section, but it is central to the thesis to make the
argument that it is because of these trade-offs that we propose the QPU
approach.

An idea is to here focus on examples of applications that might choose different
points in these trade-offs.

\subsection{Response time vs Freshness}

\subsection{Response time \& Freshness vs Cost}


\section{Constraints of geo-distributed data}
Discuss the implication of performing query processing on geo-distributed data,
and of replicating the derived data used for query processing.

\subsection{Implications of replicated data in derived state}
Implication of query processing over AP storage systems:
\begin{itemize}
  \item If not carefully desgined, concurrent non-conflicting updates may
  result to confliced on related state.
  Example: write1: recordX:attributeA=1, write2: recordY:attriuteA=1.
  This results to: index\_update1: add recordX to (attributeA, 1),
  index\_update2: add recordY to (attributeA, 1).
  Therefore a ''set CRDT'' behavior is needed for indexes.
  Generalize to state other than indexes.
  \item Conflicting updates to the corpus data.
  Ensure that conflict resolution on corpus data is reflected on derived data.
\end{itemize}

\subsection{CAP}
Describe how query processing performance-availability is constraint by
CAP in the case of replicated derived state.