The goal of this chapter is to prepare the ground so that the reader has the
mental framework needed to understand the rest of this document.

This includes:
\begin{itemize}
  \item Implicitly establishing a terminology to be used throughout the
  document.
  \item Establishing a ''world view'' by describing the system and data model
  that we consider, and the assumptions that we make
  \item Establishing concepts that will guide our design decisions in the
  following chapters (see Section~\ref{sec:requirements}).
  \item Introducing technical background that will be needed for the following
  chapters.
\end{itemize}

\section{Models}
\subsection{System model}
We consider a data serving system with a two-tiered architecture: a data
storage tier and a query processing tier.


\subsubsection{Data storage tier}
The data storage tier has the role of storing data (we term the data stored by
the data storage tier as the corpus). \\

\textbf{Data (corpus) model} \\
In short, the corpus is a collection of records grouped in tables.
Each record is identified by a unique key (can be a composite key).
Each record also has associated attributes. An attribute is a key-value pair.
Schema can be pre-defined or not.

Argue that this data model can express a variety of data models (relational,
wide-column, object storage. \\

Logically, we consider this tier as a component that may expose some of the
following APIs:
\begin{itemize}
  \item An API for iterating over the corpus data.
  \item An API for subscribing to notification for changes to the corpus data.
  \item An API for querying the corpus data.
\end{itemize}
This component can be implementing by an actual storage system that stores the
corpus data (in that case it might provide all three APIs), or by an event
stream providing only the subscription API.

We also consider this tier from a distribution point of view (how data is
placed in a physical system infrastructure)
We consider the data storage tier as being deployed across multiple
geographically distant data centers.
Therefore, the data may be distributed and/or replicated across multiple
geographic locations.

Also:
\begin{itemize}
  \item Describe our assumptions on the consistency guarantees of the storage
  tier.
  \item State that we consider this tier ''imposed''; we build on top of it.
\end{itemize}

\subsubsection{Query processing tier}

\begin{itemize}
  \item Descibe the intended functionality.
  \item Sketch the query language.
\end{itemize}


\subsubsection{Consistency between corpus data and query responses}

\section{Requirements}
\label{sec:requirements}

\subsection{End-user requirements}
In this section, present the factors that affect the experience and satisfaction
of end-users of the query processing tier.

\textbf{Response time} \\
Present evidence of the negative effects of users perceiving high response times.

\textbf{Freshness}
\begin{itemize}
  \item Define freshness intuitively, as query results being up-to-date with the
  corpus data.
  \item Present evidence of use cases in which getting fresh results is
  important.
\end{itemize}

\textbf{Correctness} \\
  Define correctness using recall and precision.

\textbf{Availability} \\
  Present evidence of the negative effects of downtimes.

\textbf{Consistency}

\subsection{System provider requirements}
In this section, we present factor that are indirectly affect end-users, but are
important for the system operator.

\textbf{Scalability} \\
Present evidence of about the scale of internet services nowadays.

\textbf{Operational Cost} \\
The cost of operating the query processing system.
Can split to fixed (monthly) cost, and per-query cost.


\section{Query Processing Techniques}

\subsection{Secondary indexing}

\subsection{Distributed Query planning}

\subsection{Caching}