\section{Navigating the design space of geo-distributed query}
(Note: The title might be a bit too fancy. To re-think.)

Question to answer:
Can the QPU approach be used to adjust to navigate the design space
geo-distributed query, making different trade-offs depending on the requirements
and characteristics of specific applications.

Hypothesis to validate:
For a given pair of workload type (eg. query-heavy) and requirement (expressed
as performance/efficiency metric) there is a query engine configuration that
''optimizes'' the given metric.

(Idea on how to visualize this: 2D matrix ''workloads - metrics'':
for each cell, find which configuration is best for this workload and metric.

High-level plan:
\begin{itemize}
  \item Run a set of different workloads, measuring different metrics (query
  performance, freshness, cost)
  \item Repeat this for a set of query system configurations.
  \item Examine for each worload-metric pair, how changing the query engine
  configuration affects the target metric.
\end{itemize}

\section{Application benchmark}
(Note: I ran out of inspiration for titles)

Question to answer:
What performance gains can the QPU approach deliver to an application, and how
it Proteus compare against state-of-the-art systems?

Hypotheses to validate
\begin{itemize}
  \item 1. The QPU approach (Proteus) can provide the performance comparable to
  that of state-of-the-art systems when used with a similar configuration.
  \item 2. The QPU approach (Proteus) can improve the application's performance
  (or improve freshness or cut cost) by enabling query engine designs (placement
  schemes, ...) not possible with current query systems.
\end{itemize}

(This is currently being designed and implemented.)