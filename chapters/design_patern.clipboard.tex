% The goal of this chapter is to present the first contribution of the thesis:
% An abstraction of a query processing operator that enables constructing query
% engines by composing instances of this abstraction.

% \begin{tcolorbox}
% Note:

% The goals of the structure of this chapter are
% \begin{enumerate}
%   \item Give the reader a clear understanding of what a QPU is.
%   It may be frustrating to go into too much detail in something as vague and
%   abstract as the ``QPU abstraction'' without showing how it is actually used.
%   \item Explain \textbf{why} we came up with it (Themis has already pointed out
%   that this was not clear to him).
%   My solution is to try to do a ``step-by-step walk-through'' of the design:
%   ``This is what we are trying to achieve (objective, requirements)'', ``this
%   is why'', ``these are the observations that guide our design'', and, finally,
%   ``here is what we came up with and why it works''.

%   It is probably a good idea to have ``toy examples'' of QPU usage here to make
%   things more clear.
% \end{enumerate}
% \end{tcolorbox}

% We need to start by motivating the contribution:
% Building on the trade-off discussion of the previous chapter, argue that query
% processing systems need to be able to be adjusted to the characteristics and
% requirements of different applications.


In this section we present our first contribution,
an architectural design pattern for constructing and deploying geo-distributed query processing systems.

This pattern
and shifts the task of designing and implementing it from the database designer to the application architect.

\section{Design goals}
We first define 
% In particular, our design is based on the following objectives:
% \vspace{-2.5mm}
% \begin{itemize}
%   \item \textbf{Declaratively defined architecture.} The query system architecture should not be predefined.
%   Rather, system architects should be able to construct query processing architectures in a per use case basis, and
%   have control over the query processing techniques that the system employs.
%   The query system design process should include decisions such as selecting the query processing state
%   partitioning and replication schemes, and whether to use caching.
%   In addition, the architecture should not make assumptions about the distribution scheme of the base data.
%   \item \textbf{Flexible component placement.} The system designer should have fine-grained control over the placement
%   of the query system's state and computations.
% \end{itemize}


\section{Query processing as a composition of modular operators}
% Point out the inherent characteristics of query processing that make the QPU
% approach possible.
% \begin{itemize}
%   \item Query processing can be broken down to primitive (cannot be broken down
%   any more) self-contained tasks.
%   \item These tasks can fit in categories/class (there can be many
%   different indexes but they all be expressed as specializations of a general
%   ``indexing'' task).
%   \item Task classes are composable and their interaction has certain
%   characteristics.
%   \begin{itemize}
%     \item They all expose a similar ``query'' interface.
%     \item Describe how data and control message flow among them.
%   \end{itemize}
% \end{itemize}


% % \subsection{Designing a query processing component abstraction}
% % Νο 4.1.2
% Present the objective: design a query processing operator abstraction ...

% Requirements:
% \begin{itemize}
%   \item Generic and Expressive. Must to be able to express multiple different
%   query processing operators (it is probably better to have explicitly listed
%   them by this point in the text).
%   \item Extensible. Must be able to express other operators, or enable
%   additional functionality.
%   \item Modularity, Composition. Must enable composition/interoperability.
%   \item Encapsulation(?). Example: A ``partitioned index'' might be a
%   complicated sub-graph of QPUs.
%   Higher-level QPUs don't need to know the details, just interface with the
%   QPU(s) at the root of the sub-graph.
% \end{itemize}




% The characteristics and properties of the system are emergent from the properties of its building block and the interactions
% between them.
% Because of that, we first present these building blocks.


The query processing unit abstraction encapsulates relational operators such as filters, joins 

polymorphic, set of common properties that enable composability, specialized implementations

In that way, complex query processing tasks can be expressed through composition of simple building blocks.
To enable this architecture design pattern, we introduce an architecture component abstraction, termed Query Processing
Unit (QPU).

% The QPU abstraction has the role of an architecture component template: it defines a set of properties including
% interfaces, functionalities, and communication patterns.
% Different instantiations (classes) of the QPU abstraction can be constructed, but all need to conform
% to the properties defined by the abstraction.
% Instances of these QPU classes are the concrete building blocks that can be composed to construct query processing
% systems.

The key idea is that query processing systems can be constructed by assembling composable building blocks that
encapsulate primitive query processing tasks.

The key idea is that query processing systems can be constructed by assembling composable building blocks that
encapsulate primitive query processing tasks.
In that way, complex query processing tasks can be expressed through composition of simple building blocks.
To enable this architecture design pattern, we introduce an architecture component abstraction, termed Query Processing
Unit (QPU).




This pattern decouples the query system from the storage architecture,
and shifts the task of designing and implementing it from the database designer to the application architect.

This architecture is aimed at enabling

polymorphic, set of common properties that enable composability, specialized implementations


an architectural design pattern

modular query system architecture
constructed by assembling composable building blocks that encapsulate primitive query processing tasks

% such as indexing and caching, and share a uniform interface.

key idea is that index-based query processing can be decomposed into primitive tasks that can be encapsulated by simple components.
key idea is that query processing can be decomposed into basic tasks that can be encapsulated by independent components.

emergent properties -> first present
Having a common abstraction ensures that ...

architecture component abstraction, called Query Processing Unit (QPU), that embodies this principle.

The QPU abstraction defines a set of interfaces and a communication protocol.

Different instantiations of the abstraction implement different functionalities while conforming to the common specification.
Query processing systems are constructed by interconnecting QPU instances in a modular microservice-like architecture.

To address this challenge, we propose a modular query system architecture.


Components are designed to be composable:
a single unified interface is used for all components encapsulating different query processing tasks, and they are able to collaborate using the interface of one another.
Since components communicate through a well-defined interface, they can be flexibly placed across the system.
This abstraction, called a Query Processing Unit (QPU), can express query processing components such as indexes, caches and Bloom filters.

A database administrator can deploy a query subsystem by interconnecting a collection of query processing units into a directed acyclic graph, and deploying them across the system infrastructure.
Different points in the design space between response time, consistency and resource utilization can be occupied by adjusting the graph topology and its placement across the system.

An individual QPU represents a unit of query processing computation, such as performing an index or cache lookup, or selecting the data items satisfy a given query by scanning a given set of data items.
Units with different functionalities but uniform interface and semantics can be composed to implement higher order functionalities.
Using this insight, we utilize the QPU abstraction as a building block for constructing query engines.




In more detail ...

\label{subsec:qpu}
% We enable these objectives using an assembly-based design strategy.
% The key idea is that query processing systems can be constructed by assembling composable building blocks that
% encapsulate primitive query processing tasks.
% In that way, complex query processing tasks can be expressed through composition of simple building blocks.
% To enable this architecture design pattern, we introduce an architecture component abstraction, termed Query Processing
% Unit (QPU).

% The QPU abstraction has the role of an architecture component template: it defines a set of properties including
% interfaces, functionalities, and communication patterns.
% Different instantiations (classes) of the QPU abstraction can be constructed, but all need to conform
% to the properties defined by the abstraction.
% Instances of these QPU classes are the concrete building blocks that can be composed to construct query processing
% systems.
% In the rest of this section we use the terms \textit{query processing unit}, \textit{QPU}, and \textit{unit}
% interchangeably.


% The goal of this section is to do a clean presentation of the QPU abstraction.

% \begin{tcolorbox}
% Note:

% Not sure how to transition from the ``step-by-step'' approach of the
% previous section to a ``definitions'' approach for this section.
% \end{tcolorbox}

% Note: Mostly keeping the outline for the rest of this chapter short as it has
% been written in various paper drafts and LightKone deliverables.

% TODO: subsection: Graph of QPUs (?)
