Today's internet-scale applications must handle large volumes of requests from users worldwide.
The database systems that these applications rely on distribute data across multiple geographically distributed sites
in order to offer fast responses and remain available in the presence of failures.
Query processing is an essential component of these systems.
In this thesis, we have studied the problem of building query engines that maintain derived state,
such as secondary indexes and materialized views, for speeding-up query processing, in the context of geo-distribution.
We have presented an analysis of the design decisions and trade-offs involved in geo-distributed query processing,
and have shown that, in the presence of these trade-offs, the placement of query processing state across the system,
and the communication patterns involved in query processing and state maintenance are crucial aspects that affect
the query engine's performance and effectiveness.
However, existing systems lack support for configurable placement of query processing state.
To address this problem, this thesis has presented a query engine architecture model aimed at enabling flexible and
configurable placement of query processing state and computations,
and an implementation of that model in Proteus, a framework for constructing application-specific, geo-distributed
query engines.
